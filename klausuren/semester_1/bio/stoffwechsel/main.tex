\documentclass{article}
\usepackage[a4paper]{geometry}
\usepackage{amsmath} \usepackage[version=4,arrows=pgf-filled,
textfontname=sffamily,
mathfontname=mathsf]{mhchem}
\usepackage{darkmode}
\enabledarkmode
\usepackage{fancyhdr}
\pagestyle{fancy}
\lhead{Stoffwechsel}
\rhead{Dezember 2024}  
 
\begin{document}
 
\section{Formel} 
\ce{C6H12O6 + 6O2 -> 6CO2 + 6H2O}, mit \ce{C6H12O6} als Glucose
  
\section{Anabol und Katabol} 
Bei anabolen (aufbauenden) Reaktionen werden einfachere Moleküle zu komplexeren Stoffen zusammengesetzt. Bei katabolen (abbauenden) Reaktionen werden komplexe Verbindungen zu einfachen Molekülen umgebaut.
 
\section{APT}
APT, Adenosintriphosphat, kann exergonisch zu ADP reagieren, sodass ein Phosphatmolekül für nutzbare Energie ausgewechselt wird. Andersherum kann ADP + P zusammen mit Energie auch endergonisch zu ATP reagieren, sodass die frei gewordene Energie für andere Reaktionen verwendet werden kann. Somit kann Energie sozusagen in ATP gespeichert werden.
 
\section{Energie in Reaktionen}
 
\begin{center}
\begin{tabular}{ |c|c| }
\hline
 exergonisch & setzt Energie frei \\
\hline
 endergonisch & benötigt Energie \\
\hline
\end{tabular}
\end{center}
nach einer exergonischen Reaktion wird Energie frei gesetzt, welche dann in einer endergonischen Reaktion genutzt werden kann.
 
\section{Mitrochondrien} 
Mitochondrien bestehen aus einer \textbf{äußeren Membran} und einer \textbf{inneren Membran}. Zwischen den beiden Membranen ist der \textbf{Intermembranraum}. Innerhalb der inneren Membran ist die \textbf{Mitochondrienmatrix}, mit dem \textbf{Plasmid} in der Mitte. Die gewölbten Teile der inneren Membran ist die \textbf{Christae}.
In der inneren Membran ist die ATP-Synthase.
 
\section{ATP-Synthase}
Innerhalb des Intermembranraums sind mehr \ce{H+}-Ionen vorhanden, weshalb sie durch die APT-Synthase in die Mitrochondrienmatrix diffundieren. Dadurch wird APT synthetisiert.  
 
\section{Redoxreaktionen}
Eine Redoxreaktionen ist eine Reaktion, bestehend aus sowohl einer Oxidation und einer Reduktion. \textbf{Eine Reduktion} ist eine Reaktion, bei welcher die Ladung reduziert wird, also negativ geladene Elektronen \textbf{dazukommen}. Eine Oxidation ist das Gegenstück, heißt eine Reaktion, bei welcher Elektronen abgegeben werden. \newline
Gegeben zwei Produkte, welche zu einander reduziert und oxidiert werden können, und der Frage, in welche Richtung die Reaktion tatsächlich ablaufen würde, ist das Redoxpotential der Elemente zu beachten. Ein höheres Redoxpotential bedeutet, dass es eher reduziert, bzw. Elektronen aufnimmt. 
 
\section{NAD}
NAD existiert als \ce{NAD+} (oxidiert) und \ce{NADH + H+} (reduziert).
Ein anderes Stoff, \ce{AH2} kann das \ce{H2} und die darin enthaltenen Elektronen bei einer Oxidation abgeben, sodass das \ce{NAD+} zu \ce{NADH + H+} rediziert werden kann. Später kann das \ce{NADH + H+} wieder zurück oxidieren und das \ce{H+} an einen anderen Stoff abgeben, damit dieser reduziert werden kann. So kann das \ce{H2} und die damit enthaltenen Elektronen auch über längere Distanzen transportiert werden.
 
\section{Atmungskette} 
Die Komplexe sind da einfach, unbeweglich. Ubichinon und Cytochrom C können sich bewegen. 
Über Komplex I und II werden Elektronen von jeweils \ce{NADH + H+} und \ce{FADH2} an Ubichinon abgegeben, welche dabei beide oxidieren, zu jeweils \ce{NAD+} und \ce{FAD}. Mit diesen reduziert dann Ubichinon. Diese Elektronen gibt Ubichinon über Komplex III an Cytochrom C ab, von wo aus sie dann über Komplex IV auf Sauerstoff übertragen werden, so dass zusammen mit 2 \ce{H+}-Ionen Wasser entsteht.
\subsection{Energetische Kopplung}
Alle Redoxreaktonen laufen exergonisch, freiwillig, ab. Die dabei frei gewordene Energie wird genutzt um \ce{H+} über Komplexe I, II und IV entgegen des Konzentrationgefälles aus der Mitrochondrienmatrix in den Intermembranraum zu pumpen. 
 
\end{document} 
 
