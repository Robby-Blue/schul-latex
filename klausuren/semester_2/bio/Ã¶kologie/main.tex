\documentclass{article} 
\usepackage{darkmode}
\enabledarkmode
\usepackage[a4paper]{geometry}
\usepackage{fancyhdr}
\pagestyle{fancy}
\lhead{Ökologie}
\rhead{Februar 2025}
\begin{document}
 
\section*{Kurzfassung}
Die wichtigsten Begriffe auf einen Blick \newline
\begin{center}
\begin{tabular}{ |c|c| }
\hline
 \multicolumn{2}{|c|}{\textbf{Ökosystem}} \\
\hline
 \textbf{Biotop} & \textbf{Biozönose} \\
\hline
 nicht lebendig & lebendig, biologisch \\
\hline
 \textbf{a}biotische Umweltfaktoren & biotische Umweltfaktoren \\
\hline
\end{tabular}
\end{center} 
 
\begin{center}
\begin{tabular}{ |c|c| } 
\hline
 \multicolumn{2}{|c|}{\textbf{Potenzen}} \\ 
\hline
 \textbf{physiologisch} & \textbf{ökologisch} \\
\hline
 experimentell & realitätsnah \\
\hline
 optimale Lebensbedingung & weitere Umweltfaktoren \\
\hline
 keine Konkurrenz & Konkurrenz \\
\hline
\end{tabular}
\end{center}
 
\begin{center}
\begin{tabular}{ |c|c| } 
\hline
 \multicolumn{2}{|c|}{\textbf{Toleranzkurven}} \\ 
\hline
 \textbf{Begriff} & \textbf{Erklärung} \\
\hline
 \textbf{Toleranzkurven} & Kurve der Potenz nach einem Umweltfaktor \\
\hline
 \textbf{Minimum / Maxiumum} & Minimale / Maximale belebbares vorkommen der Umweltfaktors \\
\hline
 \textbf{Optimum} & Punkt der höchsten Potenz \\
\hline
 \textbf{Präferenzbereich} & Bereich um das Optimum \\
\hline
 \textbf{Pessimum} & Bereich um enden, belebbar, nicht fortpflanzbar \\
\hline
\end{tabular}
\end{center}  
Eine 2D Toleranzkurve ist ein Ökogramm.   
 
\section{Ökosystem}  
\begin{description} 
\item[dynamisch] Ökosysteme konenn durch Einflüsse, den \textbf{Umweltfaktoren}, sowohl von Innen und Außen verändert werden. 
\item[komplex] Die einzelnen Bestandteile eines Ökosystems wirken in einem komplexen Geflecht dauerhaft unterschiedlich aufeinander ein.
\item[offen] Es gibt keine klar definierten Grenzen zwischen mehreren Ökosystemen, sie gehen ineinander über. Lebewesen (und auch Energie) können sich frei zwischen mehrere Ökosystemen bewegen.
\end{description} 
 
\noindent Ein Ökosystem setzt sich aus der \textbf{Biozönose}, allen dort lebenden Lebewesen, und dem \textbf{Biotop}, dem nicht lebendigem Lebensraum der Lebewesen. Bäume o.ä. sind auch Lebewesen. \newline
Umweltfaktoren durch das Biotop, wie das Grundwasser, der Boden oder der Niederschalg, sind \textbf{abiotisch}. Umweltfaktoren der Biozönose, wie Zellatmung und Fotosynthese oder Konkurrenz zwischen Organismen, sind \textbf{biotisch}. Abiotische und biotische Umweltfaktoren können auch auf den jeweils anderen Teil des Ökosystems einwirken: abiotische Faktoren wirken sich viel auf die Biozönose aus und biotische Faktoren können sich auch auf das Biotop auswirken.
 
\section{Potenz} 
Die Potenz gibt an, in welchem Bereich eines Umweltfaktores ein Lebewesen überleben und sich fortpflanzen kann. Dabei wird unterschieden in \textbf{physiologische} und \textbf{ökologische} Potenzen, jeweils in experimentell hergestellten Optimalbedingungen und unter realitätsnahen Einflüssen von durch andere Umweltfaktoren.
 
\subsection{Toleranzkurve}
Die Toleranzkurve ist eine kurve, der Normalverteilung ähnlich, welche die Potenz eines Lebewesens basierend auf einen Umweltfaktor darstellt. Die gesamte Kurve zeigt den \textbf{Toleranzbereich}, den Bereich, in dem Lebewesen leben können. An den enden ist das \textbf{Minimum und Maximum}. In der Mitte ist das \textbf{Optimum}, der Bereich um das Optimum ist der \textbf{Präferenzbereich}. Kurz vor den enden bis zu den enden, die zwei Bereiche mit der geringsten Potenz, können Lebewesen gradeso überleben aber sich nicht fortpflanzen. Das ist das \textbf{Pessimum}.
 
\subsection{Ökogramm}
Ein Ökogramm ist basically eine zweidimensionale Toleranzkurve, zeigt die Potenz nach zwei Umweltfaktoren.
 
\end{document}
 
