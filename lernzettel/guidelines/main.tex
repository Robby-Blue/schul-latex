\documentclass{article} 
\usepackage{microtype}
\usepackage[a4paper]{geometry}
 
\title{
 \LARGE{\textls[200]{Guidelines}} \\
 \normalsize{\textls[250]{für die}} \\ [-0.7em]
 \LARGE{Formattierung} \\
 \normalsize{\textls[250]{der}} \\ [-0.7em]
 \LARGE{\textls[200]{Lernzettel}} 
} 
 
\date{24. März, 2025}
\begin{document}
 
\maketitle
 
\noindent Damit alle Lernzettel zusammen passen und miteinander konsistent sind braucht es (mehr oder weniger) strikte Guidelines zur Formattierung dieser.
 
\section{Formeln}
Das Differenzial eines Integrals wird als \verb|\,\mathrm{d}x| augeschrieben. Der Ableitungsoperator wird als \verb|\frac{\mathrm{d}}{\mathrm{d}x}| aufgeschrieben. \newline
Damit die Textgröße konsistent bleibt wird \verb|\dfrac| oder ähnliches nicht verwendet. \newline
Die Eulersche Zahl nutzt \verb|\mathrm{e}|. Bekannte, definierte, Punkte nutzen auch \verb|\mathrm{...}|
 
\section{Graphen}
Der Graph ansich ist ohne weiteres \verb|->, blue, thick|. Besondere Punkte werden generell in schwarz mit einem normal mal zeichen (\verb|\times|) markiert. Werden mehrere Farben benötigt, stehen \verb|blue|, \verb|red| und \verb|green!50!black| zur auswahl. \newline
Wird \verb|minipage| genutzt, wird der Graph als $\approx 1$ cm größer angegeben, als er inhaltlich ist. 
 
\section{Sections}
Jedes Dokument sollte das Thema des Dokuments als haupt \verb|\section| verwenden. Bezieht sich das Dokument direkt auf ein Fach, so können alle sections bis \verb|\subsubsection| verwendet werden. Bezieht sich das Dokument auf ein über Thema, kann es nur bis \verb|\subsection| gehen.
 
\end{document}
 
 
