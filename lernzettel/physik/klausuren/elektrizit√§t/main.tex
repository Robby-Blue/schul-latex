\documentclass{article}
\usepackage{darkmode} 
\usepackage{tikz}
\usepackage{amsmath}
\enabledarkmode
\usepackage{svg}
\usepackage[a4paper]{geometry}
\usepackage{fancyhdr}
\pagestyle{fancy}
\lhead{Physik, erste Klausur Lernzettel}
\rhead{September 2024}
\begin{document}
\section{Mathe Basics}

\begin{align*}
    E &= \frac{U}{d} \\
    E &= \frac{F_{el}}{Q} \\
    W &= F \cdot s \\
    F &= m \cdot g \\
    U &= \frac{W}{Q} \\
    I &= \frac{Q}{t} \\
    \sigma &= \frac{Q}{A} \\
    \sigma &= \frac{\varepsilon_0}{E} \\
    F &= m \cdot a \\
    v_x &= \frac{l}{t}
\end{align*}

\section{Experimente}
\subsection{Millikan}
Unglaublich kleine Öltropfen rein sprühen, die dann positiv Geladen sind.
Am Plattenkondensator Spannung erhöhen bis eins Schwebt, also 
\[F_g=F_{el}\]
Beim Einfügen von $F_g$ und $F_{el}$ kommt man auf
\begin{align}
\setcounter{equation}{0}
    m \cdot g &= E \cdot q \\
    m \cdot g &= \frac{U}{d} \cdot q \\
    q &= m \cdot g \cdot \frac{d}{U}
\end{align}
Bei wiederholung fällt auf, dass alle Öltröpfchen - bis auf Messfehler - den gemeinsamen gleichen kleinsten Teiler
\[q=e=1,602 \cdot 10^{-19} C\]

\subsection{Braunsche Röhre}
Aufbau:
\begin{enumerate}
\item Glühwendel
\item Wehneltzylinder
\item Ringanode
\item Ablenkkondensator
\item Leuchtschirm
\end{enumerate}

Zudem muss man wissen:
\[e_{kin} = \frac{1}{2}m{v_x}^2\]

\subsection{Kraft der Ablenkung}
\[tan(\gamma)=\frac{F_{el}}{F_G}\]
von
\[tan(\gamma)=\frac{\Delta x}{\Delta y}\]

\subsection{Ladungsmessung mit Amperemeter}
In einem t-I-Diagramm entsteht ein exponentiel Abnehmende Graph, dessen Fläche unter der Kurve Q darstellt. Kann mit Trapezen angenähert werden, wobei \(\frac{a+b}{2} \cdot h\)

\subsection{Glimmlampe}
\begin{enumerate}
\item Neon under geringem Druck
\item freie Elektronen und positive Ne-Ionen, weil radioaktive Strahulung
\item elektronen angezogen zur Anode, schlagen auf, aber unsichtbar
\item Ne-Ionen angezogen zur Kathode, schlagen auf, schlagen Elektronen raus
\item Elektronen treffen auf Ne-Ionen, neutralisieren zu Ne-Atomen. Dies gibt Energie in form von Licht ab
\item Elektronen ionisieren wieder, es fängt von vorne kontinuierlich an
\end{enumerate}

\section{Herleitungen}
\subsection{$E=\frac{U}{d}$}
Für Arbeit eine negative Ladung entgegen den Spannungsgradienten gilt \(W = F \cdot s\).
Das führt du einer Spannungsänderung, wobei \(U = \frac{W}{Q}\).
Nun muss man es nach \(E\) Auflösen.

\begin{align*}
    W &= U \cdot Q \\
    F \cdot s &= U \cdot Q
\intertext{Zusammen mit \(E = \frac{F}{Q}\) gilt \(F = E \cdot Q\)}
    E \cdot Q \cdot s &= U \cdot Q \\
    E \cdot s &= U \\
    E &= \frac{U}{s}
\end{align*}
Mit der gegebenen Strecke $s$ = $d$ gilt dann
\[E = \frac{U}{d}\]

\subsection{Radialfeld}
\begin{align*}
    E &= \frac{\sigma}{\varepsilon_0} \\
    \sigma &= \frac{Q}{A} \\
    \sigma &= \frac{Q}{4 \pi r^2}
\intertext{Beide Seiten durch $\sigma$ teilen}
    E &= \frac{Q}{4 \pi r^2} \cdot \frac{1}{\varepsilon_0}
\intertext{Mit $F = Q * E$ und umstellen}
    F &= \frac{1}{4 \pi \varepsilon_0} \cdot \frac{Q_1 \cdot Q_2}{r^2}
\end{align*}

\section{Anderes}
Linien gehen von positiv zu negativ

\subsection{Influenz}
Freie Elektronen bewegen sich zur positiven Umgebungsladung
\subsection{Polarisation}
Elektronen innerhalb der Hülle bewegen sich zur positiben Umgebungsladung, die Hülle ist so nicht ganz rund.

\end{document}
