\documentclass{article} 
\usepackage{amsmath}
\usepackage{amssymb} 
\usepackage[a4paper]{geometry}
\usepackage{fancyhdr}
\pagestyle{fancy} 
\lhead{Coulombsche Gesetz}
\rhead{August 2025}
\begin{document}
\section{Coulombsche Gesetz} 
Das coulombsche Gesetz beschreibt die wirkende Kräfte zwischen zwei Punktladungen.
Es gilt
\[
 \vert F_1 \vert = \vert F_2 \vert = \frac{1}{4 \pi \varepsilon_0} \cdot \frac{Q_1 \cdot Q_2}{r^2} 
\]
Mit $r$ als Distanz zwischen den beiden Punktladungen. An $\vert F_1 \vert = \vert F_2 \vert$ ist zu erkennen, dass beide Punktladungen eine gleich große Kraft aufeinander auswirken, wobei aber, weil sie sich gegenseitig entweder anziehen oder abstoßen, die beiden Kräfte in entgegengesetzte Richtung zeigen.
\subsection{Herleitung}
Die Formel kann dadurch begründet werden, dass $F_1 = Q_1 \cdot E_2=Q_1 \cdot \dfrac{\sigma}{\varepsilon_0}$, wobei $\sigma = \dfrac{Q_2}{A}$, ausbreitend als Kugel, mit $A_{\text{kugel}} = 4 \pi r^2$, ist. Wird gleiches für $F_2$ aufgestellt, folgt die gleiche Formel, nur mit $Q_1$ und $Q_2$, beide im Zähler, getauscht. Aufgrund des Kommutativgesetzes ist dies irrelevant.
 
\end{document}