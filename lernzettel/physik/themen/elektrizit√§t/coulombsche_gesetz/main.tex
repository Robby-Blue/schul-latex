\documentclass{article} 
\usepackage{amsmath}
\usepackage{amssymb} 
\usepackage[a4paper]{geometry}
\usepackage{fancyhdr}
\pagestyle{fancy} 
\lhead{Coulombsche Gesetz}
\rhead{August 2025}
\begin{document}
\section{Coulombsche Gesetz} 
Das coulombsche Gesetz beschreibt die wirkende Kräfte zwischen zwei Punktladungen.
Es gilt
\[
 F_1 = F_2 = \frac{1}{4 \pi \epsilon_0} \cdot \frac{Q_1 \cdot Q_2}{r^2}
\]
Dabei ist $r$ die Distanz zwischen den beiden Punktladungen.
\subsection{Herleitung}
% TODO: evtl umschreiben, mit ganzer herleitung oder garnicht, weil es mit abi relevant ist 
Die Formel kann dadurch begründet werden, dass die Flächenladungsdichte (proportional zur elektrischen Feldstärke, proportional zur Kraft) mit $1 / A$ wächst, wobei sich die Kräfte als Kugel ausbreite, wobei $A_{\text{kugel}} = 4 \pi r^2$ ist.
 
\end{document}