\documentclass{article}
\usepackage{amsmath}
\usepackage[a4paper]{geometry}
\usepackage{fancyhdr}
\pagestyle{fancy}
\lhead{Elektrische Feldstärke in einem Plattenkondensator}
\rhead{August 2025}
\begin{document}
\section{Elektrische Feldstärke in einem Plattenkondensator} 
Zwischen einem Plattenkondensatoren im Abstand $d$ gilt
\[
 E=\frac{U}{d} 
\]
 
\subsection{Herleitung}
Um eine Ladung zu bewegen muss eine Arbeit $W$ verrichtet werden, wodurch die Spannung verändert wird
\[
 W=F \cdot s
 \quad \text{und} \quad 
 U = \frac{W}{Q} \implies W = U \cdot Q 
\]
Aus diesel beside Gleichungen nach $W$ gleichgesetzt und der Umformung nach $E = \dfrac{F}{Q}$ folgt in einer Form
\[
  E = \frac{F}{Q} = \frac{U \cdot Q}{s \cdot Q}
\]
Durch das Wegfallen der $Q$s und der Strecke $s=d$ folgt $E = \dfrac{U}{d}$
\end{document}