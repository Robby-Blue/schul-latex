\documentclass{article}
\usepackage{csquotes}
\usepackage[a4paper]{geometry}
\usepackage{fancyhdr}
\pagestyle{fancy}
\lhead{Das Mach-Zehnder-Interferometer}
\rhead{November 2025}
\begin{document}
\section{Das Mach-Zehnder-Interferometer}
Ein \emph{Mach-Zehnder-Interferometer} ist ein Interferometer, bestehend aus einem Laser, zwei Strahlteilern, zwei Spiegeln und zwei Detektoren.
 
\subsection{Beobachtung} 
Bei Detektor $2$ kommen zwei Lichtsprünge mit zwei Phasensprüngen an, so dass es zu einem Interferenzmuster bei maximaler konstruktiver Interferenz konnt. Bei Detektor $1$ kommen Lichtstrahlen mit jeweils $1$ und $2$ Phasensprüngen an, so dass es mit $\Delta = \lambda / 2$ zu maximaler destruktiver Interferenz kommt.
 
Die Photonen verhalten sich delokalisiert, das Interferenzmuster deutet darauf hin, dass jedes Photon mit sich selbst interferiert. Wird ein Lichtweg detektor ein um den Lichtweg zu \textquote{messen}, so verschwindet das Interferenzmuster. 
 
\subsection{Komplimentaritätsprinzip}
Es ist nicht möglich in einem Experiment sowohl die Welleneigenschaften, z.\,B. ein Interferenzmuster, und die Teilcheneigenschaften, z.\,B. \textquote{Weg}, von Photonen zu messen.
\end{document}