\documentclass{article}
\usepackage{amsmath}
\usepackage[a4paper]{geometry}
\usepackage{fancyhdr}
\pagestyle{fancy}
\lhead{Das Linienspektrum von Wasserstoffatomen}
\rhead{Dezember 2025}
\begin{document}
\section{Das Linienspektrum von Wasserstoffatomen}
Geht ein Elektron vom einen in das andere Energieniveau innerhalb eines Wasserstoffatoms über, so gibt es Licht in mit bestimmten Frequenzen ab. Diese folgen dem Muster
\[
 f = f_{Ryd} \cdot \left(\frac{1}{n^2} - \frac{1}{m^2}\right)
 \qquad \text{mit $f_{Ryd}$ als Rydbergkonstante} 
\]
Dabei sind $m$ und $n$ die Indexe der Elektronenbahnen. $m$ ist die Startbahn, $n$ die Endbahn.
 
\end{document}