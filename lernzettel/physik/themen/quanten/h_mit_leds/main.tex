\documentclass{article}
\usepackage{csquotes}
\usepackage[a4paper]{geometry}
\usepackage{fancyhdr}
\pagestyle{fancy}
\lhead{$h$ durch LEDs}
\rhead{September 2025}
\begin{document}
\section{$h$ durch LEDs} 
Das planksche Wirkungsquantum $h$ kann mithilfe von LEDs bestimmt werden.
 
\subsection{Hintergrundwissen} 
Eine LED (leuchtdiode) produziert Licht, indem bei anliegender Spannung freie Valenzelektronen der n-dotierung in die \textquote{Löcher} der p-dotierung springen. Dabei wird Energie frei, welche als Licht an die Umwelt abgegeben wird. 
 
\subsection{Aufbau}
%TODO: tikz
Es wird eine LED einer bestimmten, bekannten, Wellenlänge $\lambda$, aus welcher $f$ folgt, in einen Schaltkreis an eine Gleichstromquelle mit variabler Spannung angeschlossen. Die Spannung wird erhöht, bis die LED geradeso anfängt Licht zu produzieren. Somit ist nun die Energie des Photons, $E_{p} = h \cdot f$, die, wenn es keinen Energieverlust gäbe, gesamte hinzugefügte elektrische Energie $E_{el} = q \cdot U$. Es gilt also
\[
 h \cdot f = q \cdot U
\]
Wird dieser Versuch mit mehreren LEDs mit unterschiedlichen Wellenlängen wiederholt, können die Messungen in einem $f$-$E$-Diagramm dargestellt werden, wobei, wie immer, $h$ die Steigung ist. 
\end{document}