\documentclass{article}
\usepackage{amsmath}
\usepackage{hyperref}
\usepackage[a4paper]{geometry}
\usepackage{fancyhdr}
\pagestyle{fancy}
\lhead{Fotoeffekt}
\rhead{September 2025}
\begin{document}
\section{Fotoeffekt}
Wird das Elektroskop vom \hyperref[Der Hallswachs-Versuch]{Hallwachs-Versuch} durch eine Ringanode und eine Kathide, verbunden durch ein hochohmiges Voltmeter, und die Glasscheibe durch Farbfilter, welche jeweils Licht von nur einer Wellenlänge durchlassen, ersetzt, so kann die entstehende Spannung in abhängigkeit von der Wellenlänge des Lichtes gemessen werden. Das Voltmeter muss hochohmig sein, damit die Spannung sich nicht von selbst durch Stromflüsse auflöst.
 
Folgend ist $U = \dfrac{W}{Q}$, mit $Q=e$ und einem $W$, so groß, wie die kinetische Energie, vom letzten Elektron, welches entgegen dem elektrischen Feld auf die Kathode übertragen wurde. Daraus folgt
\[
 E_{kin} = e \cdot U
\] 
 
Die Energie des Photons ist somit all die Energie, welche es in Form von $E_{kin}$ an das Elektron übergibt, plus all die Energie, welche es genutzt hat, um das Elektron überhaupt zu lösen, auch die Ablösearbeit $W_A$ oder Ionisierungsenergie genannt.
\[
 E_{Photon} = W_A + E_{kin} 
\] 
 
\subsection{$f$-$E$-Graphen} 
In einem $f$-$E$-Graphen fallen drei besondere Punkte auf
\begin{enumerate}
 \item Es gibt eine bestimmte Frequenz $f_G$, welche die Grenzfrequenz darstellt, ab welcher ein Photon genug Energie hat um die Elektronen aus der Kathode zu lösen. Diese ist bei $E(f_G)=0$ auffindbar
 \item Die Ablösearbeit $W_A$ ist der y-Achsenabschnitt. Dies kann so begründet werden, indem auffällt, dass $E(0) - E(f_G)$ einerseits beschreibt, wie viel weitere Energie hinzugefügt werden muss, bevor die Elektronen abgelöst werden, andererseits aber auch gleich $E(0)$ ist.
 \item Die Energie der Photon ist linear proportional zur Frequenz, mit einem Faktor $h$, dem \emph{Plankschen Wirkungsquantum}. Somit gilt
\[
 E_{Photon} = h \cdot f 
\] 
\end{enumerate} 
 
\subsection{Elektronenvolt} 
Ein Elektronenvolt, $1 \, eV$, beschreibt die Energie, welche ein Elektron besitzt, wenn es durch eine Spannung von einem Volt beschleunigt wurde
\[
 1 \, eV = 1.602 \cdot 10^{-19} \, J 
\] 
\end{document}