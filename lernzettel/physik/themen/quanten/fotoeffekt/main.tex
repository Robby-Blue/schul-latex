\documentclass{article}
\usepackage{amsmath}
\usepackage{hyperref}
\usepackage[a4paper]{geometry}
\usepackage{fancyhdr}
\pagestyle{fancy}
\lhead{Fotoeffekt}
\rhead{September 2025}
\begin{document}
\section{Fotoeffekt}
Wird das Elektroskop vom \hyperref[Der Hallswachs-Versuch]{Hallwachs-Versuch} durch eine Ringanode und eine Kathide, verbunden durch ein hochohmiges Voltmeter, und die Glasscheibe durch Farbfilter, welche jeweils Licht von nur einer Wellenlänge durchlassen, ersetzt, so kann die entstehende Spannung in abhängigkeit von der Wellenlänge des Lichtes gemessen werden. Das Voltmeter muss hochohmig sein, damit die Spannung sich nicht von selbst durch Stromflüsse auflöst.
 
Folgend ist $U = \dfrac{W}{Q}$, mit $Q=e$ und einem $W$, so groß, wie die kinetische Energie, vom letzten Elektron, welches entgegen dem elektrischen Feld auf die Kathode übertragen wurde. Daraus folgt
\[
 E_{kin} = e \cdot U
\] 
 
Die Energie des Photons ist somit all die Energie, welche es in Form von $E_{kin}$ an das Elektron übergibt, plus all die Energie, welche es genutzt hat, um das Elektron überhaupt zu lösen, auch die Ablösearbeit $W_A$ oder Ionisierungsenergie genannt.
\[
 E_{Photon} = W_A + E_{kin} 
\] 
 
\subsection{f-E-Graphen} 
% probe neues doc, mut _h dann 
\end{document}