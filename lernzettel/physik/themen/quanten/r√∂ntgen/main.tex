\documentclass{article}
\usepackage{amsmath}
\usepackage[a4paper]{geometry}
\usepackage{fancyhdr}
\pagestyle{fancy}
\lhead{Röntgenstrahlung}
\rhead{September 2025}
\begin{document}
\section{Röntgenstrahlung}
Bei der Entstehung der Röntgenstrahlung wird eine Glühkathode mit einer $U_A$ von $100\, \text{V}$ - $200\, \text{kV}$ betreieben. Die austretenden e$^-$ treffen daraufhin auf eine Anode, bestehend aus einem Material mit einer hohen Ordnungszahl. Beim Auftreffen werden die Elektronen rasant abgebremst, weshalb Röntgenstrahlung entsteht. Die Anode ist generell als Trapez geformt, so dass die Röntgenstrahlung in die freie Seite scheinen kann.
 
Dabei besteht die Röntgenstrahlung eigentlich aus zwei Strahlungen: der \emph{Bremsstrahlung} und der \emph{charakteristische Strahlung}
\begin{description} 
 \item[Bremsstrahlung] Es gibt ein kontinuierliches Spektrum für die Energie jedes $f < f_{max}$. Diese geben dementsprechend weniger Energie ab.
 \item[Charakteristische Strahlung] Die Lage der Peaks ist für das Anodenmaterial charakteristisch. 
\end{description} 
 
\subsection{Kurzwellige Grenze $\lambda_{\text{min}}$} 
Es gilt 
\[
 e \cdot U = h \cdot f_{max} 
\]
Es folgt, beim darüber nachdenken,
\[
 \lambda_{min} = \frac{h \cdot c}{e \cdot U} 
\] 
 
\end{document}