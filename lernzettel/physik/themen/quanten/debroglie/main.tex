\documentclass{article}
\usepackage{amsmath}
\usepackage[a4paper]{geometry}
\usepackage{fancyhdr}
\pagestyle{fancy}
\lhead{Die de-Broglie-Wellenlänge}
\rhead{Oktober 2025}
\begin{document}
\section{Die de-Broglie-Wellenlänge}
Louis de Brogie fand heraus, dass auch Quantenobjekte, wie Elektronen, sogesehen eine Wellenlänge haben können, ohne tatsächlich Wellen zu sein.
Dabei gilt
\[
 \lambda = \frac{h}{p} 
 \qquad
 \text{mit dem Impuls}
 \qquad
 p = m \cdot v 
\]
% TODO: massiv erweitern 
\end{document}