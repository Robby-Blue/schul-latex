\documentclass{article}
\usepackage[a4paper]{geometry}
\usepackage{fancyhdr}
\pagestyle{fancy}
\lhead{Heisenbergsche Unschärferelation}
\rhead{Oktober 2025}
\begin{document}
\section{Heisenbergsche Unschärferelation}
Die \emph{Heisenbergsche Unschärferelation} stellte die Ortsschärfe $\Delta x$ und die Impulsschärfe $\Delta P$ in relation zueinander
\[
 \Delta x \cdot \Delta p \approx \frac{h}{2} 
\]
Weil das Produkt der beiden eine Konstante darstellt folgt eine Antiproportionalität. Je größer, also schärfer, die eine ist, desto unschärfer muss die andere sein. 
% TODO evtl erweitern 
\end{document}