\documentclass{article}
\usepackage{amsmath}
\usepackage{hyperref}
\usepackage[a4paper]{geometry}
\usepackage{fancyhdr}
\pagestyle{fancy}
\lhead{Hallwachs}
\rhead{September 2025}
\begin{document}
\section{Der Hallswachs-Versuch} 
Wird eine Lampe, welche auch UV-Strahlen abgibt, wie eine Quecksilber Dampflampe, auf ein negativ geladenes Elektroskop mit einem ausgeschlagenem Zeiger geschienen, so geht der Zeiger des Elektroskopes runter. Wird aber gleiches Wiederholt, mit einer Glasscheibe zwischen der Lampe und dem Elektroskop, so passiert, unabhängig von der Intensität des Lichtes oder der Distanz der Lampe, erstmal nichts, zeigte der Hallwachs-Versuch.
 
Dies ist dadurch zu begründen, dass das Licht genug Energie hat um frei bewegliche Elektronen aus dem geladenen Metall des Elektroskops zu lösen. Dass das Licht all diese Energie augenscheinlich verlor als die Glasscheibe hinzugefügt wurde, ist darin zu begründen, dass das Glass kein UV durchlässt, welches aber die meiste Energie beinhaltet.
 
Dass die Energie des Lichtes von der Wellenlänge (UV oder nicht) und nicht von der Intensität des Lichtes abhängt lässt sich nicht durch das Wellenmodell erklären, sondern zeigt, dass das Licht aus Teilchen, den \emph{Photonen}, besteht.
\end{document}