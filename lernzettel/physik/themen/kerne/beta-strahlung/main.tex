\documentclass{article}
\usepackage{amsmath} 
\usepackage[a4paper]{geometry}
\usepackage{fancyhdr}
\pagestyle{fancy}
\lhead{Beta-Strahlung}
\rhead{Januar 2026}
\begin{document}
\section{Beta-Zerfall}
Beim Beta-Zerfall bekommt das aus dem Kern emittierte Elektron seine kinetische Energie aus der Massendifferenz der beiden Elemente. Zerfällt A zu B, so ist
\[ 
 E = (m_A - m_B) \cdot c^2
\]
Diese Energie ist aber nicht die gesamte $E_{kin}$ des Elektrons; diese Energie wird statistisch auf den Rückstoß des Atomkerns, die Energie das Antineutrinos und das $E_{kin}$ des Elektrons aufgeteilt.
 
In der Nuklidkarte ist der Mittelwert des $E_{kin}$ des Elektrons angegeben.
 
Diese liegt immer kontinuierlich zwischen $0$ und $2,1\,\text{MeV}$. Der Mittelwert liegt bei ungefähr $1/2$ bis $1/3$ des $E_{max}$.
\end{document}