\documentclass{article}
\usepackage{amsmath}
\usepackage[a4paper]{geometry}
\usepackage{fancyhdr}
\pagestyle{fancy}
\lhead{Atomekerne} 
\rhead{Januar 2026}
\newcommand{\safecenter}[1]{\makebox[3cm][c]{#1}} 
\newcommand{\s}[3]{\safecenter{$\displaystyle ^{#1}_{#2}#3$}} 
\begin{document} 
\section{Atomekerne} 
Atomkerne, \emph{Nuklide}, bestehen aus positiven Protonen und neutralen Neutronen. Beides sind Nukleonen. Somit kann jedes Nuklid durch den Namen des Elementes und drei Zahlen beschrieben werden:
\begin{enumerate}
 \item Die Anzahl der Protonen $Z$
 \item Die Anzahl der Neutronen $N$
 \item Die Anzahl der Nukleonen $A$ 
\end{enumerate} 
Es gilt hier $A = N + Z$. Isotope sind Atomkerne des gleichen Elementes, mit einer unterschiedlichen Nukleonenzahl.
 
\subsection{Schreibweise}
Nuklide können auf drei Art und Weisen aufgeschrieben werden: 
\[
 ^A_Z\text{Element}
 \qquad \text{und} \qquad
 \text{Element-}A
 \qquad \text{und} \qquad
 \text{Element}_A
\] 
So können auch Protonen, Neutronen, Elektronen und Positronen angegeben werden.
\begin{center}
 \begin{tabular}{p{3cm} p{3cm} p{3cm} p{3cm}}
  \safecenter{Protonen} & \safecenter{Neutronen} & \safecenter{Elektronen} & \safecenter{Positronen} \\ 
  \s{1}{1}{p} & \s{1}{0}{n} & \s{0}{-1}{e} & \s{0}{1}{e} 
 \end{tabular} 
\end{center} 
  
\subsection{Masse}
Die Masse wird in der atomaten Masseneinheit $u$ angegeben,
\[
 1\,u = \frac{1}{12} m_{\text{c-}12} 
\] 
 
\subsection{Mol} 
Ein Mol ist eine Mengeneinheit an Teilchen. Ein Mol sind immer, unabhängig des Elements, $6,022 \cdot 10^{23}$ Elemente. Dies ist die \emph{Avogadro}-Konstante 
 
\end{document}