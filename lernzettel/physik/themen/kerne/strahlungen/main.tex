\documentclass{article}
\usepackage{amsmath}
\usepackage[a4paper]{geometry}
\usepackage{fancyhdr}
\pagestyle{fancy}
\lhead{Strahlung}
\rhead{Januar 2026}
\newcommand{\e}{\ksymb{0}{-1}{e}}
\newcommand{\p}{\ksymb{1}{1}{p}}
\newcommand{\n}{\ksymb{1}{0}{n}}  
\newcommand{\ksymb}[3]{\substack{#1 \\ #2}#3} 
\begin{document}
\section{Strahlung} 
Strahlung kann in drei Arten aufgeteilt werden, $\alpha$-, $\beta^-$-, $\beta^+$- und $\gamma$-Strahlung. 
\begin{description}
 \item[$\alpha$-Strahlung] Bei der $\alpha$-Strahlung werden zwei Positronen und zwei Neutronen emittiert, genau der Aufbau eines Heliumkerns. Somit kann die $\alpha$-Strahlung als $\ksymb{4}{2}{He}$ oder als $\ksymb{4}{2}{\alpha}$ aufgeschrieben werden. Dies ist zweifach positiv und sehr einfach Abschirmbar. 
 \item[$\beta^-$-Strahlung] Bei der $\beta^-$-Strahlung wird ein einfaches Elektron $\e$, einfach negativ, emittiert. Dieses kann durch ungefähr 100 Blätter oder $5\,mm$ Aluminiumblech abgeschirmt werden.
 
Im Kern selbst wird dabei ein Neutron in ein Proton, ein Elektron und ein Antielektronneutrino $\overline{\nu}_e$ geteilt.
\[
 \n \rightarrow \p + \e + \overline{\nu}_e
\] 
 \item[$\beta^+$-Strahlung] Die $\beta^+$-Strahlung ist der $\beta^-$-Strahlung sehr ähnlich, mit dem Unterschied dass anstelle eines Elektrons ein einfach positives Prositron $\ksymb{0}{1}{e}$ emittiert wird.
 
Hier wird im Kern ein Proton in ein Positron, ein Elektron und ein Elektronenneutrino geteilt.
\[
 \p \rightarrow \ksymb{0}{1}{e} + \n + \nu_e
\]
 \item[$\gamma$-Strahlung] Die $\gamma$-Strahlung emittiert e-m-Wellen, also ein Photon beziehungsweise ein $\gamma$-Quant $\ksymb{0}{0}{\gamma}$. Dieses ist neutral. Es kann von dickem Blei und Beton absorbiert werden. Der Kern wird abgeregt.
\end{description} 
\end{document}