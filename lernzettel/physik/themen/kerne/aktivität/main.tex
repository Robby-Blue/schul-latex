\documentclass{article}
\usepackage{amsmath}
\usepackage[a4paper]{geometry}
\usepackage{fancyhdr}
\pagestyle{fancy}
\lhead{Aktivität}
\rhead{Januar 2026}
\begin{document}
\newcommand{\dd}{\text{d}} 
\section{Aktivität}
Die \emph{Aktivität} einer radioaktiven Substanz gibt die Anzahl der Kernumwandlungen pro Sekunde an. Gibt es $N$ Kerne, so ist
\[
 A = -\frac{\Delta N}{\Delta t}
 \qquad \text{in Bq (Becquerel)} 
\] 
Die Aktivität ist vom Element selbst als auch von der Menge der insgesamt vorliegenden Kernen abhängig. Für diese Abhängigkeit wird die Zerfallskonstante $\lambda$ genutzt.
\[
 A = \lambda \cdot N 
\]
Somit gibt die Zerfallskonstante den prozentualen Anteil an Kernen, welche innerhalb einer Sekunde zerfallen an.
 
\subsection{Das Zerfallsgesetz}
Werden die beiden obigen Definition gleichgesetzt, so folgt 
\[
 \frac{\dd N}{\dd t} = -\lambda \cdot N 
\]
Wird auf beiden Seiten durch $N$ dividiert und ein Integral genommen, so folgt
\begin{align*} 
 \int \frac{\dd N}{N} &= \int -\lambda \cdot dt \\ 
 \ln{N} &= -\lambda \cdot t + c \\
 N &= e^{-\lambda \cdot t} \cdot e^c
\end{align*}
Wird dies nun bei $t=0$ betrachtet, so ist $e^{-\lambda \cdot 0} = 1$, so dass $e^c=N_0$, die Anzahl der unzerfallenen Kernen am Anfang, sein muss.
 
Daraus folgt die Formel für $N(t)$, welches die Anzahl der noch vorhandenen, unzerfallenen, Kerne zum Zeitpunkt $t$ angibt.
\[
 N(t) = N_0 \cdot e^{-\lambda \cdot t} 
\]
Wird beides durch $t$ dividiert folgt das gleiche 
\[
 A(t) = A_0 \cdot e^{-\lambda \cdot t} 
\]
\end{document}