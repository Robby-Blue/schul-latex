\documentclass{article}
\usepackage[a4paper]{geometry}
\usepackage{fancyhdr}
\pagestyle{fancy}
\lhead{Geiger-Müller-Zähler}
\rhead{Januar 2026}
\begin{document}
\section{Geiger-Müller-Zähler}
Ein \emph{Geiger-Müller-Zähler} wird genutzt um Strahlung zu detektieren.
 
\subsection{Aufbau}
Der Geiger-Müller-Zähler ist vereinfacht ein Stromkreis zwischen einem Metallmantel und einem Metalldraht als Anode, welche zusammen, getrennt durch ein Widerstand, das Geiger-Müller-Zählrohr darstellen. Dieses ist mit Gas gefüllt.
 
Betritt Strahlung das Zählrohr, so Ionisiert es Gas und löst dessen Elektronen. Dem elektrischen Feld nach bewegen diese sich zum Metallrohr, auf dessen Weg sie der Stoßionisation nach $10^9$ mal so viele weitere Lösen. Der Stromstoß kann gemessen, gezählt und als Ton abgespielt werden.
 
\subsection{Totzeit} 
Nach einer Zählung liegt einer \emph{Totzeit} vor. Es kann für ungefähr $10^{-3}$ Sekunden keine weitere Strahlung gemessen werden.
 
Dies liegt daran, dass aufgrund des vielen Ladungsträgerpaare es zu einem kleineren Widerstand kommt. Mit $U = R \cdot I$ sinkt die Spannung, Elektronen können im Feld nicht mehr ausreichend beschleunigt werden. 
\end{document}