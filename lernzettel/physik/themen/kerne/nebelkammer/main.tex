\documentclass{article}
\usepackage{amsmath}
\usepackage[a4paper]{geometry}
\usepackage{fancyhdr}
\pagestyle{fancy}
\lhead{Die Nebelkammer}
\rhead{Januar 2026}
\begin{document}
\section{Die Nebelkammer}
Eine \emph{Nebelkammer} oder \emph{Wilson-Expansionsnebelkammer} kann genutzt werden um $\alpha$-Strahlung detektieren und deren Energie einzuordnen.
 
\subsection{Aufbau} 
Eine Nebelkammer ist eine Kammer mit Glasfenstern und einem verschiebbaren Kolben. Sie enthält zudem das Präparat. In der Kammer liegt zudem ein Luft-Wasser-Alkohol Gasgemisch vor, bei ungefähr $20^\circ\,\text{C}$.
 
\subsection{Funktionsweise}
Wird der Kolben verschoben, so dass das Gas mehr Platz hat, so nutzt es diesen. Zum expandieren nutzt es die eigene Energie; es kühlt sich auf ungefähr $10^\circ$ ab.
 
Bewegt sich nun $\alpha$-Strahlung durch die Kammer, so erzeugt es der Flugbahn entlang tausende an freie Elektronen und positiven Ionen.
 
Weil Wassermoleküle Dipole sind kondensieren sie bevorzugt an diesen besagten positiven Ionen. Dies kann durch das Glas beobachtet werden. Ist die Kondensationsspur länger, so wurden mehr paare an Elektronen und Ionen erzeugt, weil eine $\alpha$-Teilchen mehr Energie hatte. 
 
\end{document}