\documentclass{article}
\usepackage[a4paper]{geometry}
\usepackage{fancyhdr}
\pagestyle{fancy}
\lhead{Halbleiterdetektor}
\rhead{Februar 2026}
\begin{document}
\section{Halbleiterdetektor}
Trifft Strahlung innerhalb der Grenzschicht auf einen Halbleiter, so kann das getroffene Elektron gelöst werden uns ein Elektronen-Loch-Paar entsteht.
 
Sind nun beide Seiten des Halbleiters über einen Widerstand in einem Stromnetz miteinander verbunden, so fließt das nun freie Elektronen dem Stromnetz entlang. Fließt es weiter durch den Halbleiter, so kann es weitere Elektronen lösen.
 
Die Elektronen, die durch das Netz fließen, können nun durch die Stromstärke gemessen werden.
 
Eine stärkere Strahlung führt zu mehr Löchern, zu einer größeren Stromstärke. 
\end{document}