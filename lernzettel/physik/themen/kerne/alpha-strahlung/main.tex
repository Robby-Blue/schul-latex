\documentclass{article}
\usepackage[a4paper]{geometry}
\usepackage{fancyhdr}
\pagestyle{fancy}
\lhead{Alpha-Strahlung}
\rhead{Januar 2026}
\begin{document}
\section{Alpha-Strahlung}
Wird sich die Gleichung eines $\alpha$-Zerfalls angeguckt und die Massen der Nukleonen aller Elemente berechnet und die darausfolgenden Energien, so fällt auf, dass diese nicht dem Energieerhaltungssatz folgen. Damit dieser eingehalten wird, muss eine weitere Energie als Edukt hinzugefügt werden.
 
Diese Energie ist aber in keiner Tabelle der Energiespektren der $\alpha$-Teilchen aufzufinden; dies liegt daran, dass die Energie auf das $\alpha$-Teilchen, den Rückstoß des Kerns und $\gamma$-Strahlung aufgeteilt werden muss.
 
Energiespektren sind diskret. 
\end{document}