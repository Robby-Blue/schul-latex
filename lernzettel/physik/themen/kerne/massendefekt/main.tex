\documentclass{article}
\usepackage{amsmath}
\usepackage[a4paper]{geometry}
\usepackage{fancyhdr}
\pagestyle{fancy}
\lhead{Der Massendefekt}
\rhead{Januar 2026}
\begin{document}
\section{Der Massendefekt}
Wird die Masse der einzelnen Teilchen eines Atomkerns zusammenaddiert, also die Anzahl der Protonen des Atoms mal die Masse eines Protonen plus die Anzahl der Neutronen des Atoms mal die Masse eines Neutronen so fällt auf, dass alle diese zusammen mehr wiegen als der Atomkern (Atommasse minus Masse der Elektronen) selbst. Dies ist der \emph{Massendefekt}.
 
\subsection{Berechnung} 
Somit ist die tabellarische Kernmasse $(m_{\text{atom}} - m_e)$ und die Masse aller Teilchen des Kerns $(m_p + m_n)$. Somit folgt die Differenz
\[ 
 \Delta m = (m_{\text{atom}} - n_e \cdot m_e) - (n_p \cdot m_p + n_n \cdot m_n)
\]
Müssen beide einzelnen Werte nicht angegeben werden kann einfach
\[ 
 \Delta m = m_{\text{atom}} - n_e \cdot m_e - n_p \cdot m_p - n_n \cdot m_n
\]
Hier sind $n_e$, $n_p$ und $n_n$ jeweils die Anzahlen der Elektronen, Protonen und Neutronen und $m_e$, $n_p$ und $m_n$ sind die Massen der drei Teilchenarten. Diese sind auch im Taschenrechner als \texttt{\_Me}, \texttt{\_Mp} und \texttt{\_Mn} vorgespeichert. 
 
\subsection{Kernbindungsenergien}  
Die $\Delta m$ kann mit $E=mc^2$ in eine Energie umgewandelt werden. Geteilt durch die Nukleonenzahl ($n_p + n_n$) folgt die \emph{Kernbindungsenergie}, die auf jedes Nukleon wirkt.
\end{document}