\documentclass{article}
\usepackage[a4paper]{geometry}
\usepackage{fancyhdr}
\pagestyle{fancy}
\lhead{Der Massendefekt}
\rhead{Januar 2026}
\begin{document}
\section{Der Massendefekt}
Wird die Masse der einzelnen Teilchen eines Atoms zusammenaddiert, also
 
die Anzahl der Elektronen des Atoms mal die Masse eines Elektrons plus
 
die Anzahl der Protonen des Atoms mal die Masse eines Protonen plus
  
die Anzahl der Neutronen des Atoms mal die Masse eines Neutronen \newline
so fällt auf, dass alle diese zusammen mehr wiegen als das Atom selbst. Dies ist der \emph{Massendefekt}.
 
\subsection{Kernbindungsenerien}  
Die Differenz zwischen der selbst ausgerechneten Masse der Teilchen und der Atommasse $\Delta m$ kann $E=mc^2$ nach in eine Energie umgewandelt werden. Geteilt durch die Anzahl an Teilchen folgt die \emph{Kernbindungsenergie}, die auf jedes Nukleon wirkt.
\end{document}