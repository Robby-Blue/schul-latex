\documentclass{article} 
\usepackage{amsmath} 
\usepackage[a4paper]{geometry}
\usepackage{fancyhdr}
\pagestyle{fancy} 
\lhead{Stromstärke}
\rhead{August 2025}
\begin{document}
 
\section{Stromstärke} 
Die elektrische Stromstärke $I$ beschreibt, wie viel Ladung pro Zeiteinehit fließt. Der Definiton nach gilt
\[
 I = \frac{Q}{t}
 \quad \text{und} \quad
 Q = I \cdot t 
\]
Demnach kann für ein $Q(t)$, welches die Ladung $Q$ bei Zeitpunkt $t$ angibt, ein $I(t)=Q'(t)$ gefunden werden. Offensichtlich ist andersherum $\displaystyle Q(t) = \int I(t) \,\mathrm{d}x$
 
\subsection{Ampere} 
Die Stromstärke wird in Ampere gemessen, wobei
\[
 1 \, \text{Ampere} = \frac{1 \, \text{Coloumb}}{1 \, \text{Sekunde}}
\] 
 
\end{document}