\documentclass{article}
\usepackage{amsmath}
\usepackage[a4paper]{geometry}
\usepackage{fancyhdr}
\pagestyle{fancy}
\lhead{Energiebilanz eines Plattenkondensatoren}
\rhead{August 2025}
\begin{document}  
 
\section{Energiebilanz eines Plattenkondensatoren}
Für einen Körper, wie ein Elektron, zwischen einem Plattenkondensatoren, gilt die Energiebilanz
\[
 \Delta E_{pot} + \Delta E_{kin} = 0
 \quad \text{mit} \quad
 E_{pot} = q \cdot U
 \quad \text{und, wie immer} \quad
 E_{kin} = \frac{1}{2} mv^2
\] 
$E_{pot}$ kann hergeleitet werden, durch 
\[
 E_{kin} = W
 \quad \text{und} \quad
 U = \frac{W}{Q} 
\]
 
\section{Elektronengeschwindigkeit im Plattenkondensator} 
Wird $E_{kin}$ nach $v$ aufgelöst und der Energiebilanz nach $E_{kin}$ durch $E_{pot}$ ersetzt, folgt mit $q=e$ die Geschwindigkeit eines Elektrons nach Verlassens der Feldes
\[
 v = \sqrt{\frac{2 eU}{m}}
\] 
 
\end{document}