\documentclass{article}
\usepackage{amsmath}
\usepackage[a4paper]{geometry}
\usepackage{fancyhdr}
\pagestyle{fancy}
\lhead{Kapazität eines Kondensators bestimmen}
\rhead{September 2025}
\begin{document}
\section{Kapazität eines Kondensators bestimmen}
Die Kapazität eines Kondensators kann experimentell bestimmt werden, indem bei einer bekannten $U$ die dazugehörige $Q$ gefunden wird, beides eingesetzt in
\[
 C = \frac{Q}{U} 
\] 
Die $Q$ kann bestimmt werden, indem ein Plattenkondensator aufgeladen wird und eine Funktion $I(t)$ bestimmt wird, wobei bekannterweise
\[
 Q = \int_0^\infty I(t) \, \text{d}x
\] 
\end{document}