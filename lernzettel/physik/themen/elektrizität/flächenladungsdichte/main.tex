\documentclass{article} 
\usepackage{amsmath}
\usepackage[a4paper]{geometry}
\usepackage{fancyhdr}
\pagestyle{fancy} 
\lhead{Flächenladungsdichte}
\rhead{August 2025}
\begin{document} 
 
\section{Flächenladungsdichte} 
Die Flächenladungsdichte $\sigma$, Sigma, ist die Dichte der Ladung auf einer Fläche. Ein Kondensator mit der Ladung $Q$ und dem Flächeninhalt $A$ hat die Flächenladungsdichte
\[
 \sigma = \frac{Q}{A}  
 \quad \text{in} \quad
 \frac{C}{m^2} 
\]
 
\subsection{Die elektrische Feldkonstante $\varepsilon_0$} 
Die Flächenladugsdichte ist zur elektrischen Feldstärke $E$ proportional, mit dem Faktor der Feldkonstante $\varepsilon_0$. Somit ist
\[
 \sigma = \varepsilon_0 \cdot E 
\]
 
\end{document}