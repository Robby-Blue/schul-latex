\documentclass{article}
\usepackage{amsmath}
\usepackage[a4paper]{geometry}
\usepackage{fancyhdr}
\pagestyle{fancy}
\lhead{Elektrische Größen}
\rhead{August 2025}
\begin{document}
 
\section{Elektrische Größen} 
 
\subsection{Spannung}
Die elektrische Spannung $U$, gemessen in Volt $V$, beschreibt die pro Ladung verrichtete Arbeit, also ist
\[
 U = \frac{W}{Q} 
\]
Zudem beschreibt die Spannung die Differenz zwischen dem Potenzial an zwei Punkten. Mit dem Potenzial $\varphi(P)$ an Punkt $P$ ist die Spannung zwischen $P_1$ und $P_2$ somit
\[
 U = \varphi(P_2) - \varphi(P_1) 
\]
 
\subsection{Stromstärke} 
Die elektrische Stromstärke $I$ beschreibt, wie viel Ladung pro Zeiteinehit fließt. Der Definiton nach gilt
\[
 I = \frac{Q}{t}
 \quad \text{und} \quad
 Q = I \cdot t 
\]
Demnach kann für ein $Q(t)$, welches die Ladung $Q$ bei Zeitpunkt $t$ angibt, ein $I(t)=Q'(t)$ gefunden werden. Offensichtlich ist andersherum $\displaystyle Q(t) = \int I(t) \,\mathrm{d}x$ 
Die Stromstärke wird in Ampere gemessen, wobei
\[
 1 \, \text{Ampere} = \frac{1 \, \text{Coloumb}}{1 \, \text{Sekunde}}
\]  
 
\subsection{Widerstand}
Der elektrische Widerstand $R$, gemessen in Ohm, notiert duch das griechische Omega, $\Omega$, folgt
\[
 R = \frac{U}{I}
 \quad \text{so dass} \quad
 1\,\text{Ohm} = \frac{1\,\text{Volt}}{1\,\text{Ampere}} 
\] 
Der elektrische Widerstand leistet dem fließendem Strom, wie der Name bereits sagt, Widerstand. Dies wird auch durch $I = U / R$ offensichtlich, welches zeigt, dass bei einem höheren Widerstand antiproportional viel weniger Strom fließt. 
 
\subsection{Flächenladungsdichte} 
Die Flächenladungsdichte $\sigma$, Sigma, ist die Dichte der Ladung auf einer Fläche. Ein Kondensator mit der Ladung $Q$ und dem Flächeninhalt $A$ hat die Flächenladungsdichte
\[
 \sigma = \frac{Q}{A}  
 \quad \text{in} \quad
 \frac{C}{m^2} 
\] 
Zudem ist die Flächenladugsdichte zur elektrischen Feldstärke $E$ proportional, mit dem Faktor der Feldkonstante $\varepsilon_0$. Somit ist
\[
 \sigma = \varepsilon_0 \cdot E 
\]
 
\subsection{Kapazität eines Kondensators}
Ein Kondensator hat die maximale Kapazität $C$. Die Kapazität beträgt
\[
 C = \frac{Q}{U} \text{,}
 \quad \text{angegeben in} \quad
 1 \,\text{Farad} = \frac{1 \,\text{Coulomb}}{1 \,\text{Volt}} 
\]
Darüberhinaus gilt in einem Plattenkondensator, dessen beiden Platten jeweils, einzeln, den Flächeninhalt $A$ haben, und eine Distanz $d$ voneinander entfernt sind.
\[
 C = \varepsilon_0 \cdot \varepsilon_r \cdot \frac{A}{d} 
\]
Somit ist die Kapazität auch zur \emph{relativen Permittivität}, auch die \emph{relative Dielektrizitätskonstante} genannt, $\varepsilon_r$, proportional. Weil die relative Permittivität von Luft bei $\varepsilon_r \approx 1$ liegt, wird diese oft weggelassen.
\end{document}