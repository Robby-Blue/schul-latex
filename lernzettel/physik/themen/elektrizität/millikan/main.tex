\documentclass{article}
\usepackage{amsmath}
\usepackage[a4paper]{geometry}
\usepackage{fancyhdr}
\pagestyle{fancy}
\lhead{Der Millikan-Versuch}
\rhead{September 2025}
\begin{document}
\section{Der Millikan-Versuch}
Mit dem \emph{Millikan-Versuch} wurde erstmals die Elektronenladung $e$ bestimmt. Es wird geladenes Öl in ein homogenes elektrisches Feld zwischen zwei Platten, an welchen eine Spannung $U$ liegt, gesprüht. Daraufhin wird auf jeweils ein Öltröpfchen fokussiert und eine Spannung angelegt, welche so groß ist, dass dieses auf der Stelle schwebt. Damit dies geschieht, muss es auf das Elektron eine Kraft, die elektrische Kraft, wirken, welche sie genau so stark nach oben zieht, wie das Tröpchen von der Erdanziehungskraft nach unten gezogen wird, so dass ein Kraftgleichgewicht vorliegt.
 
Qualitativ gilt also
\begin{align*}
 F_e &= F_g \\
 q \cdot E = q \cdot \frac{U}{d} &= m \cdot g
\intertext{Nach $q$ umgeformt also}
 q = m \cdot g \cdot \frac{d}{U} 
\end{align*}
Wird dieser Veruch mit mehreren, unterschiedlich großen, Öltröpfchen wiederholt, so fällt auf, dass alle Ölladungen, egal wie klein sie auch sind, immer ein vielfaches von $\approx 1,602 \cdot 10^{-19} \,C$ ist, der Elementarladung $e$. 
\end{document}