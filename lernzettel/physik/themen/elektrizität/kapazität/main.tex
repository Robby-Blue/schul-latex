\documentclass{article}
\usepackage{amsmath}
\usepackage[a4paper]{geometry}
\usepackage{fancyhdr}
\pagestyle{fancy}
\lhead{Kapazität eines Kondensators}
\rhead{August 2025}
\begin{document}
\section{Kapazität eines Kondensators}
Ein Kondensator hat die maximale Kapazität $C$. Die Kapazität beträgt
\[
 C = \frac{Q}{U} \text{,}
 \quad \text{angegeben in} \quad
 1 \,\text{Farad} = \frac{1 \,\text{Coulomb}}{1 \,\text{Volt}} 
\]
Darüberhinaus gilt in einem Plattenkondensator, dessen beiden Platten jeweils, einzeln, den Flächeninhalt $A$ haben, und eine Distanz $d$ voneinander entfernt sind.
\[
 C = \varepsilon_0 \cdot \varepsilon_r \cdot \frac{A}{d} 
\]
Somit ist die Kapazität auch zur \emph{relativen Permittivität}, auch die \emph{relative Dielektrizitätskonstante} genannt, $\varepsilon_r$, proportional. Weil die relative Permittivität von Luft bei $\varepsilon_r \approx 1$ liegt, wird diese oft weggelassen.
\end{document}