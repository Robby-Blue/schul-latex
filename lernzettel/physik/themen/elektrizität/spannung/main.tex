\documentclass{article} 
\usepackage{amsmath}
\usepackage[a4paper]{geometry}
\usepackage{fancyhdr}
\pagestyle{fancy} 
\lhead{Spannung}
\rhead{August 2025}
\begin{document}
 
\section{Spannung}
Die elektrische Spannung $U$, gemessen in Volt $V$, beschreibt die pro Ladung verrichtete Arbeit, also ist
\[
 U = \frac{W}{Q} 
\]
Zudem beschreibt die Spannung die Differenz zwischen dem Potenzial an zwei Punkten. Mit dem Potenzial $\varphi(P)$ an Punkt $P$ ist die Spannung zwischen $P_1$ und $P_2$ somit
\[
 U = \varphi(P_2) - \varphi(P_1) 
\] 
 
\end{document}