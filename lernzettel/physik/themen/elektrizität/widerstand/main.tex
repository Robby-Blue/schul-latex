\documentclass{article}
\usepackage{amsmath}
\usepackage[a4paper]{geometry}
\usepackage{fancyhdr}
\pagestyle{fancy}
\lhead{Widerstand}
\rhead{August 2025}
\begin{document}
\section{Widerstand}
Der elektrische Widerstand $R$, gemessen in Ohm, notiert duch das griechische Omega, $\Omega$, folgt
\[
 R = \frac{U}{I}
 \quad \text{so dass} \quad
 1\,\text{Ohm} = \frac{1\,\text{Volt}}{1\,\text{Ampere}} 
\] 
Der elektrische Widerstand leistet dem fließendem Strom, wie der Name bereits sagt, Widerstand. Dies wird auch durch $I = U / R$ offensichtlich, welches zeigt, dass bei einem höheren Widerstand antiproportional viel weniger Strom fließt.
\end{document}