\documentclass{article}
\usepackage[a4paper]{geometry}
\usepackage{fancyhdr}
\pagestyle{fancy}
\lhead{Polarisation}
\rhead{September 2025}
\begin{document}
\section{Polarisation}
Die Polarisation einer Transversalwelle beschreibt die Schwingungsrichtung des schwingungsfähigen System.
 
Die Polarisation kann mit einem \emph{Polarisator} überprüft werden: einem horizontalem oder vertikalem Gitter, welches nur Wellen durchlässt, welche eine bestimmte Polarisation haben.
 
Es ist sich eine Welle vorzustellen, welche sich an einem Seil entlang bewegt. Schwingt dieses vertikal, während der Polarisator horizontale Gitter hat, so blockieren die Gitter das Seil und es kann nicht weiter schwingen. Schwingt das Seil in richtung des Gitters, so kann es problemlos schwingen. Bei elektromagnetischen Wellen ist es sich gegenteilig vorzustellen: schwingt eine diese entlang des Gitters, so bilden sich weitere Wellen, welche mit den originalen hinter dem Polarisator destruktiv interferieren und die Welle komplett auslöschen.  
 
Werden zwei Polarisatoren unterschiedlicher Richtung hintereinander aufgestellt, so wird der zweite ein \emph{Analysator} genannt und sie lassen beide zusammen nur Wellen durch, welche in keine Richtung polarisiert ist, heißt, Longitudinalwellen.
\end{document}