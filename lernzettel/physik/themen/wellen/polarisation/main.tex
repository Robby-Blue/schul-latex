\documentclass{article}
\usepackage[a4paper]{geometry}
\usepackage{fancyhdr}
\pagestyle{fancy}
\lhead{Polarisation}
\rhead{September 2025}
\begin{document}
\section{Polarisation}
Die Polarisation einer Transversalwelle beschreibt die Schwingungsrichtung des schwingungsfähigen System.
 
Die Polarisation kann mit einem \emph{Polarisator} überprüft werden: einem horizontalem oder vertikalem Gitter, welches nur Wellen durchlässt, welche der Höhe des Gitters entlang schwingen.
 
Werden zwei Polarisatoren unterschiedlicher Richtung hintereinander aufgestellt, so wird der zweite ein \emph{Analysator} genannt und sie lassen beide zusammen nur Wellen durch, welche in keine Richtung polarisiert ist, heißt, Longitudinalwellen.
\end{document}