\documentclass{article}
\usepackage{csquotes}
\usepackage[a4paper]{geometry}
\usepackage{fancyhdr}
\pagestyle{fancy}
\lhead{Stehende Wellen}
\rhead{Septemeber 2025}
\begin{document}
\section{Stehende Wellen} 
Wird konsant eine Welle bei passender Frequenz erzeugt, in einem System, in welchem dieses am Ende des Mediums reflektiert, so kann durch die Interferenz eine \emph{stehende Welle} enstehen. Eine stehende Welle nimmt die Form einer typischen Welle, eine Sinusfunktion, an, ohne sich aber zu bewegen.
Sie besteht aus Wellenbäuchen, den Bergen und Tälern, und den Wellenknoten, die Punkte, welche zwischen zwei Bäuchen sind. Weil eine Wellenbauch einer stehenden Welle jeweils so lang wie ein Wellenberg ist, ist $\lambda$ das doppelte der Länge eines Wellenbauches.  
% TODO tikz, eine Sinusfunktion, mit Wellenbauch und Knoten 
 
\subsection{Experimente}
Stehende Wellen können bei unterschiedlichen Versuchen beobachtet werden
\begin{description}
 \item[Sandmuster] Wird mit einer bestimmen Frequenz eine runde Platte, auf welcher sich Sand befindet, vibriert, so sorgen stehende Wellen dafür, dass die Amplitude der Vibration an manchen Stellen konstant größer ist als an anderen. \textquote{Springen} nun die Sandkörnchen aufgrund der Vibrationen weg, dorthin, wo nur weniger Vibrationen sind, bildet sich ein Muster an periodisch wiederholten Kreisen.
 \item[Mikrowellen] Wird eine Tafel Schokolade in eine Mikrowelle gestellt, dessen Drehteller entfernt wurde, so schmelzen nur kleine Teile der Schokolade, so dass die Welleneigenschaften bestimmt werden können. Die Ultraschall wellen der Mikrowelle, welche auch diejenigen Wellen sind, welche das Essen erhitzen sollen, werden zu einer stehende Welle, so dass an bestimmten Orten der Mikrowelle (dort, wo die Wellenbäuche sind) geballtere Energie ist, als an anderen. An diesen Orten schmilzt dann auch die Schokolade mehr. \newline 
Werden so zwei kleine geschmolzene Punkte in der Schokolade, in einem Abstand $d$ zueinander, so ist dies auch die Länge des Wellenbauches und es folgt $\lambda = 2 \cdot d$ 
\end{description} 
\end{document}