\documentclass{article}
\usepackage[a4paper]{geometry}
\usepackage{fancyhdr}
\pagestyle{fancy}
\lhead{Ausbreitung}
\rhead{September 2025}
\begin{document}
\section{Ausbreitung} 
Wellen breiten sich dem \emph{huygensschen Prinzip} nach aus, welches besagt, dass jeder Punkt einer Wellenfront als Ausgangspunkt einer neuen Elementarwelle angesehen werden kann. Die neue Wellenfront setzt sich somit aus der Überlagerung aller neuen Elementarwellen zusammen. \newline
Durch dieses Prinzip kann die Ausbreitung, die Reflexion, die Brechung und die Beugung von Wellen erklärt werden.  
\begin{description} 
 \item[Reflexion] Trifft eine Welle auf eine Wand, so wird sie Reflektiert. Der Einfallswinkel und Reflexionswinkel, im Vergleich zum Lot, also der Normale der Wand.
 \item[Brechung] Wechselt eine Welle das Medium, so bricht der Strahl und ändert seine Ausbreitungsrichtung basierend auf den Ausbreitungsgeschwindigkeiten in den beiden Medien. Es gilt 
\[
 \frac{\sin \alpha}{\sin \beta} = \frac{c_1}{c_2} 
\] 
Dabei hat das erste Medium, dasjenige Medium, welches verlassen wird, eine Ausbreitungsgeschwindigkeit von $c_1$ und den Winkel zwischen der Ausbreitungsrichtung im Medium 1 und der Normale der Mediumsgrenze $\alpha$. Gleiches gilt für Medium 2 und $c_2$ und $\beta$.
  \item[Beugung] Betritt eine Welle den Bereich hinter einem Hinderniss, so kann diese trotzdem durch die Beugung den verhinderten Schatten betreten. Dies liegt daran dem huygenschen Prinzip nach die Wellenfront Elementarwellen bildet, welche seitlich, in Richtung des Schattens, sich ausbreiten können.
\end{description}  
 
\subsection{Reflexion mit Phasensprung}
Kommt eine Welle am Ende des Mediums, in welchem sie sich befindet, an, so wird die Reflektiert.
 
Dies kann entweder ohne einem Phasensprung passieren, so dass die Welle quasi die Richtung wechselt, ein Wellenberg aber ein Wellenberg bleibt, oder mit einen Phasensprung passieren, so dass bei der Reflexion die Phase der Welle verändert wird (ein Teil über\emph{sprungen} wird). In der Regel liegt ein Phasensprung, wenn überhaupt, in der Größe von $\lambda / 2$ vor, so dass ein Wellenberg zu einem Wellental wird.
 
Ob es zu einem Phasensprung kommt oder nicht hängt davon ab, es sich um eine Transversalwelle oder Longitudinalwelle handelt und ob das Ende \emph{fest} oder \emph{lose} ist, heißt, ob es mitschwingen kann.
 
\begin{center}
\begin{tabular}{ |c|c|c| }
\hline
   & \textbf{Transversalwelle} & \textbf{Longitudinalwelle} \\
\hline
 \textbf{festes Ende} & Phasensprung $\lambda / 2$ & kein Phasensprung \\
\hline
 \textbf{loses Ende} & kein Phasensprung & Phasensprung $\lambda / 2$ \\
\hline
\end{tabular}
\end{center}
 
\end{document}