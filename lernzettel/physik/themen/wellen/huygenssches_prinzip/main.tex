\documentclass{article}
\usepackage[a4paper]{geometry}
\usepackage{fancyhdr}
\pagestyle{fancy}
\lhead{Huygenssches Prinzip}
\rhead{September 2025}
\begin{document}
\section{Huygenssches Prinzip} 
Das huygensche Prinzip besagt, dass jeder Punkt einer Wellenfront als Ausgangspunkt einer neuen Elementarwelle angesehen werden kann. Die neue Wellenfront setzt sich somit aus der Überlagerung aller neuen Elementarwellen zusammen. \newline
Durch dieses Prinzip kann die Ausbreitung, die Beugung (dem Eintretendes geometrischen Schattens hinter einem Hindernisses), die Brechung und die Reflexion von Wellen erklärt werden. 
\end{document}