\documentclass{article}
\usepackage{hyperref}
\usepackage[a4paper]{geometry}
\usepackage{fancyhdr}
\pagestyle{fancy}
\lhead{Mechanische Wellen}
\rhead{September 2025}
\begin{document}
\section{Mechanische Wellen} 
% TODO: tikz für sinusfunktion, für diese andere darstellungsweise 
Eine mechanische Welle ist eine örtliche und zeitliche periodische Veränderung einer physikalischen Größe. Dem \emph{Resonanzprinzip} nach wird die Energie eines \emph{Erregers} über die Kopplung mit einem benachbarten schwingungsfähigen System auf dieses übertragen.
 
Eine mechanische Welle kann sich sozusagen als eine \hyperref[Mechanische Schwingungen]{mechanische Schwingung} vorgestellt werden, welche sich aber darüberhinaus örtlich weiterverbreitet.
 
\subsection{Größen} 
Eine Welle hat eine \emph{Wellenlänge} $\lambda$, welche beschreibt, wie weit eine Welle sich örtlich während einer Periodendauer. Wird die Welle als Sinusfunktion dargestellt, ist $\lambda$ eine Periode. Daraus folgt auch, dass Wellen eine \emph{Ausbreitungsgeschwindigkeit} $c$ haben
\[
 c = \frac{\lambda}{T} = \lambda \cdot f 
\]
Die Ausbreitungsgeschwindigkeit $c$ ist manchmal, aber nicht immer, auch die Lichtgeschwindigkeit $c$.
 
\subsection{Schwingungsrichtungen} 
Es wird bei mechanischen Wellen in zwei Schwingungsrichtungen unterschieden
\begin{description}
 \item[Transversalwellen] Teilchen schwingen senkrecht zur Ausbreitungsrichtung
 \item[Longitudinalwellen] Teilchen schwingen in Ausbreitungsrichtung, z.\,B. Schallwellen
\end{description} 
\end{document}