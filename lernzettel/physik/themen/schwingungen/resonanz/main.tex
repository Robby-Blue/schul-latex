\documentclass{article}
\usepackage[a4paper]{geometry}
\usepackage{fancyhdr}
\pagestyle{fancy}
\lhead{Resonanz}
\rhead{Oktober 2025}
\begin{document}
\section{Resonanz}
\emph{Resonanz} tritt auf, wenn die Schwingung des einen schwingungenden Systems die Schwingung eines anderen Verursacht. Dabei wird Energie vom \emph{Erreger} zum \emph{Resonator} übertragen. Wie viel Energie übertragen wird hängt von der Schwingungsfrequenz des Erregers und der Eigenfrequenz des Resonators ab, je ähnlicher diese sind, desto mehr Energie wird übertragen. Sind diese gleich, kommt es zu einer \emph{Resonanzkatastrophe}.
 
\subsection{Beispiele}
Zu den Beispielen gehören
\begin{enumerate}
 \item Brücken, auf welchen Marschiert wird 
 \item Federungen
 \item Gläser 
\end{enumerate} 
\end{document}