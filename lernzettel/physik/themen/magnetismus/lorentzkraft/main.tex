\documentclass{article}
\usepackage{xcolor}
\usepackage[a4paper]{geometry}
\usepackage{fancyhdr}
\pagestyle{fancy}
\lhead{Lorentzkraft}
\rhead{September 2025}
\begin{document}
 
\section{Lorentzkraft}
Die Lorentzkraft, $F_L$, ist die Kraft, welche auf elektrisch geladene Teilchen in einem Magnetfeld wirkt. 
\subsection{\emph{UVW}-Regel} 
Die Richtung der Auswirkung eines Magnetfeldes kann mithilfe \emph{UVW}-Regel bestimmt werden. Dabei wird der Daumen und der Zeige- und Mittelfinger in jeweils $90^\circ$ Winkeln ausgestreckt. Dabei stellt jeder Finger immer die Richtung einer Kraft dar.
\begin{center}
\begin{tabular}{ |l|c|c| }
\hline
 \colorbox{red!30}{U}rsache & Stromfluss & Daumen \\
\hline
 \colorbox{red!30}{V}ermittlung & Magnetfeld & Zeigefinger \\
\hline
 \colorbox{red!30}{W}irkung & Kraft & Mittelfinger \\
\hline
\end{tabular}
\end{center}
Sind zwei Richtungen bekannt, kann diese Regele angewand werden, um die dritte zu finden.
 
\subsection{Quantitative Bestimmung}
Die Lorentzkraft ist das Produkt von $B$, $I$ und der Länge der Leiters $l$ 
\[
 F_L = B \cdot I \cdot L 
\] 
 
\end{document}