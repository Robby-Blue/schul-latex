\documentclass{article}
\usepackage{amsmath}
\usepackage{hyperref}
\usepackage[a4paper]{geometry}
\usepackage{fancyhdr}
\pagestyle{fancy}
\lhead{Hallsonde}
\rhead{September 2025}
\begin{document}
\section{Hallsonde} 
Eine Hallsonde ist ein Sensor, welcher in einem Magnetfeld die wirkende \hyperref[Hall Effekt]{Hallspannung} misst. Es ist davon auszugehen, dass diese zu jedem Zeitpunkt, auch wenn sie sich nicht im für die Messung relevanten Magnetfeld befindet, misst. Diese \emph{Offsetspannung} muss später bei den relevanten Messwerten abgezogen werden. \newline
Wird mit einer Hallsonde die $U_H$ einer langen Spule gemessen, so gilt mit dem gemessen $U_H$ und einer Proportionalitätskonstante $k$ in $\dfrac{\text{T}}{\text{V}}$
\[
 B = k \cdot U_H 
\] % TODO: oder so  
\end{document}