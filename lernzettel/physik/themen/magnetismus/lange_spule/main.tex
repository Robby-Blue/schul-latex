\documentclass{article}
\usepackage{amsmath}
\usepackage[a4paper]{geometry}
\usepackage{fancyhdr}
\pagestyle{fancy}
\lhead{Lange (Schlanke) Spulen}
\rhead{September 2025}
\begin{document}
\section{Lange (Schlanke) Spulen} 
Eine \emph{lange Spule}, auch \emph{schlanke} Spule genannt, ist eine Spule, für welche $l > 3 \cdot d$ gilt. In einer langen Spule mit der Windungszahl $N$ gilt das homogene Magnetfeld
\[
 B = \mu_0 \cdot \frac{N \cdot I}{l}
 \qquad \text{mit $\mu_0$ als \emph{magnetische Feldkonstante}}
\] 
Somit ist das Magnetfeld von der Querschnittsfläche unabhängig. 
 
\subsection{Magnetische Feldkonstante $\mu_0$} 
Für die magnetische Feldkonstante gilt dabei
\[
 \mu_0 = 4 \pi \cdot 10^{-7} \frac{\text{T} \cdot \text{m}}{\text{A}} 
\] 
\end{document}