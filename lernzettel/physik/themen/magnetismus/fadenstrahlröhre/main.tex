\documentclass{article}
\usepackage{hyperref}
\usepackage{amsmath}
\usepackage[a4paper]{geometry}
\usepackage{fancyhdr}
\pagestyle{fancy}
\lhead{Die Fadenstrahlröhre}
\rhead{September 2025}
\begin{document}
\section{Die Fadenstrahlröhre} 
Die Fadenstrahlröhre ist ein Messgerät, welches genutzt werden kann, um die \emph{spezifische Ladung} von Elektronen (das Verhältniss zwischen der Ladung und der Masse, also $\frac{e}{m}$) zu bestimmen und daraus dann, mithilfe der durch Millikan bereits bekannten Elektronenladung $e$, die Masse der Elektronen zu berechnen.
 
Mit einer typischen Elektronenkanone werden Elektronen in einem gasgefüllten Glaskolben, welcher in einem Magnetfeld, welches orthogonal zu den Elektronen, liegt, gefeuert. Aufgrund der \hyperref[Lorentzkraft]{Lorentzkraft}, welches auch mit der UVW-Regel bestimmt werden kann, wirkt auf die Elektronen zu jedem Zeitpunkt eine Kraft in einem $90^\circ$ Winkel zu ihrer Flugbahn, sodass sie letztendlich in einem Kreis fliegen. Diese kreisförmige Flugbahn wird durch das Gas im Kolben sichtbar.
 
Beim Aufbau ist in der Regel die Elektronenkanone nach oben gerichtet und das homogene Magnetfeld in die ebene hinein gerichtet.
 
\subsection{Herleitung}%PROC Abiturrelevant
Weil die Lorentzkraft dafür sorgt, dass die Elektronen in einem Kreis fliegen kann sie auch als eine Zenipetalkraft angesehen werden, also
\[
 F_L = e \cdot v \cdot B = F_Z = m \cdot \dfrac{v^2}{r}
\]
Wird dies nach der spezifischen Ladung umgeformt, gilt
\[
 \frac{e}{m} = \frac{v}{r \cdot B} 
\]
Die Elektronengeschwindigkeit ist dabei
\[
 v = \sqrt{\frac{2eU}{m}}
 \quad \text{folgend aus} \quad
 e \cdot U = \frac{1}{2} m \cdot v^2 
\]
Ineinander eingesetzt gilt nun
\[
 \frac{e}{m} = \frac{\sqrt{\frac{2eU}{m}}}{r \cdot B} = \frac{2 \cdot U}{r^2 \cdot B^2}
\]
Also
\[
 m = \frac{r^2 \cdot B^2}{2 \cdot U} \cdot e
\] 
\end{document}