\documentclass{article}
\usepackage{amsmath}
\usepackage[a4paper]{geometry}
\usepackage{fancyhdr}
\pagestyle{fancy}
\lhead{Magnetische Flussdichten}
\rhead{September 2025}
\newcommand{\ounit}[1]{
 1\,\text{#1} 
} 
 
\begin{document}
\section{Magnetische Flussdichten}
Die \emph{magnetische Flussdichte} $B$ beschreibt die Stärke eines Magnetfeldes. Für ein magnetisches Feld, in welchem ein elektrischer Leiter ist, gilt
\[
 B = \frac{F_L}{I \cdot l} 
 \quad \text{mit $l$ als Länge des Leiters} 
\]
Somit ist die magnetische Flussdichte $B$ sehr ähnlich zur elektrischen Feldstärke $E$, so dass beides die wirkende Kraft pro einer elektrischen Größe darstellt. \newline
$B$ wird dabei in der Einheit Tesla $T$ angegeben, wobei, aus der obigen Formel offensichtlich,
\[ 
 \ounit{Tesla} = \frac{\ounit{Newton}}{\ounit{Ampere} \cdot \ounit{Meter}}
\] 
Die magnetische Flussdichte an einem Punkt welcher eine Distanz $d$ vom Leiter entfernt ist beträgt
\[
 B = \mu_0 \cdot \frac{I}{2 \pi d} 
\] 
\end{document}