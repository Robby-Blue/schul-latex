\documentclass{article}
\usepackage[a4paper]{geometry}
\usepackage{fancyhdr}
\pagestyle{fancy}
\lhead{Magnetfelder}
\rhead{September 2025}
\begin{document}
\section{Magnetfelder}
Ein jeder Magnet hat zwei Pole; einen \emph{Nord-} und einen \emph{Südpol}. Ähnlich wie bei der Ladung, ist es auch hier so, dass sich gleinamige Pole abstoßen, ungleichnamige Pole ziehen sich an. \newline  
Um einen jeden Magneten existiert ein Magnetfeld, dargestellt durch geschlossene Feldlinien, welche von Nord nach Süd gehen. \newline 
Darüberhinaus hat auch jeder stromdurchflossene Leiter ein eigenes Magnetfeld, wobei so aufgestellt ist, dass es mithilfe der linken Hand bestimmt werden kann. Zeigt der Daumen von der linken Hand in die Richtung, in welche sich die Elektronen bewegen, so zeigen die anderen Finger, wenn sie zusammengezogen werden, die Richtung der Feldlinien an.
\end{document}