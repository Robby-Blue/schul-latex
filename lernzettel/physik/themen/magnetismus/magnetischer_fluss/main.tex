\documentclass{article}
\usepackage[a4paper]{geometry}
\usepackage{fancyhdr}
\pagestyle{fancy}
\lhead{Magnetischer Fluss}
\rhead{September 2025}
\begin{document}
\section{Magnetischer Fluss}  
Ist $B$ die magnetische Flussdichte, so muss es natürlich auch die physikalische Größe des magnetischen Flusses geben, dessen Dichte die magnetische Flussdichte sein kann. Der magnetische Fluss $\Phi$, aufgeschrieben als ein großes Phi, ist
\[
 \Phi = A \cdot B 
\]  
 
\subsection{Die Veränderte \emph{UVW}-Regel} 
Bewegt sich eine Leiterschleife durch ein Magnetfeld, gilt auch hier die \emph{UVW}-Regel, nur dass der Daumen, die Ursache, hier natürlich die rollende Schleife darstellt. Der Zeigefinger, als Vermittlung, ist das Magnetfeld. Der Mittelfinger bleibt als Richtung des Stromflusses übrig. 
\end{document}