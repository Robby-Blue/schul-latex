\documentclass{article}
\usepackage{amsmath}
\usepackage{hyperref}
\usepackage[a4paper]{geometry}
\usepackage{fancyhdr}
\pagestyle{fancy}
\lhead{Transformatoren}
\rhead{September 2025}
\begin{document}
\section{Transformatoren} 
Ein \emph{Transformator} ist ein technische Anwendung der \hyperref[Induktion]{Induktion}. Um einen Eisenkern werden zwei Spulen gespannt, eine \emph{Primärspule} mit der Windungszahl $N_1$ und der Spannung $U_1$ und eine \emph{Sekundärspule}, mit $N_2$ und $U_2$. Dabei ist
\[
 \frac{N_1}{N_2} = \frac{U_1}{U_2} 
 \qquad \text{und} \qquad
 \frac{N_1}{N_2} = \frac{I_2}{I_1} 
\]
Weil die Induktion ein sich änderndes Magnetfeld benötigt, muss hier Wechselstrom anliegen. Bei Gleichstrom ist das Magnetfeld konstant, es kommt zu keiner Induktion und der Transformator kann nicht funktionieren. 
\end{document}