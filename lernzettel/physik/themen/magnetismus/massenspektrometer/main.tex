\documentclass{article}
\usepackage{amsmath}
\usepackage{hyperref}
\usepackage[a4paper]{geometry}
\usepackage{fancyhdr}
\pagestyle{fancy}
\lhead{Massenspektrometer}
\rhead{September 2025}
\begin{document}
\section{Massenspektrometer} 
Wird hinter ein \hyperref[Wien Filter]{Wien Filter} ein homogenes Magnetfeld, senkrecht zu den ankommenden Ionen, platziert, an wessen Anfang auch eine Detektorplatte platziert wird, werden die durchgelassenen Ionen konstant durch die Lorentzkraft, wirkend wie die Zentripetalkraft, abgelenkt, so dass es einen Halbkreis mit Radius $r$ dreht. Somit ist der Auftreffpunkt auf der Detektorplatte vom Eingangspunkt $2r$ entfernt.
 
Aus $F_L = F_Z$ folgt nun
\[
 r = \frac{m \cdot v}{q \cdot B_a}
 \qquad \text{mit} \quad
 v = \frac{E}{B_w} 
\]
Dies kann auch zu $m$, $q$ oder $\dfrac{q}{m}$ umgestellt werden. 
\end{document}