\documentclass{article}
\usepackage{amsmath}
\usepackage[a4paper]{geometry}
\usepackage{fancyhdr}
\pagestyle{fancy}
\lhead{Wien Filter}
\rhead{September 2025}
\begin{document}
\section{Wien Filter}
Ein \emph{Wien Filter} besteht aus einem homogenen Magnetfeld und einem Plattenkondensator, welcher ein homogenes elektrisches Feld erzeugt. Diese Felder sind dabei aber so angeordnet, dass das elektrische Feld ein von links ankommendes Elektron mit $F_{el}$ nach oben ablenkt, während das Magnetfeld das gleiche Elektron mit $F_{l}$ nach unten ablenkt.
 
Das Elektron kann also den Filter passieren ohne abgelenkt zu werden, wenn die beiden wirkenden Kräfte gleich groß sind, also wenn
\begin{align*}
 F_{el} &= F_l \\
 e \cdot E &= e \cdot v \cdot B \\
 v &= \frac{E}{B} 
\end{align*} 
Ist $v \ne \dfrac{E}{B}$, so ist eine Kraft stärker als die andere und das Elektron wird dementsprechend abgelenkt. 
\end{document}