\documentclass{article}
\usepackage{hyperref}
\usepackage[a4paper]{geometry}
\usepackage{fancyhdr}
\pagestyle{fancy}
\lhead{Induktion}
\rhead{September 2025}
\begin{document}
 
\newcommand{\derive}[2]{\frac{\mathrm{d}#1}{\mathrm{d}#2}} 
 
\section{Induktion}  
Solange sich ein Magnetfeld um einem Leiter verändert, wird in diesem Leiter eine Spannung, die \emph{Induktionsspannung}, \emph{induziert}. \newline
Um diese Potenzialdifferenz auszugleichen fließt darauffolgend Strom, der \emph{Induktionsstrom}. Der Induktionsstrom ist immer so gerichtet, dass das  durch den Stromfluss entstehende Magnetfeld entgegen der Ursache der Induktion wirkt, besagt das Lenzsche Gesetz. 
\[
 U_{Ind} = -N \cdot \derive{\Phi}{t}
\] 
Mit der Definition vom \hyperref[Magnetischer Fluss]{Magnetischem Fluss} und der \hyperref[Ableitungsregeln]{Produktregel} lautet
\[
 \derive{\Phi}{t} = A \cdot \derive{B}{t} + B \cdot \derive{A}{t} 
\] 
Normalerweise wird aber zu jedem Zeitpunkt nur eine der beiden Größen verändert, sodass nur einer der beiden Summanden beachtet werden muss; der jeweils andere ist dann immer $0$.
Den weiteren Ableitungsregeln nach, hat ein sich konstant veränderndes Magnetfeld eine konstante $U_{Ind}$ und ein Magnetfeld, dessen Größe einer Sinusfunktion folgt, eine Induktionsspannung, welche der gleichen Sinusfunktion, um eine halbe Periode versetzt, folgt. 
\end{document}