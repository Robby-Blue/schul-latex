\documentclass{article}
\usepackage{hyperref}
\usepackage[a4paper]{geometry}
\usepackage{fancyhdr}
\pagestyle{fancy}
\lhead{Induktion}
\rhead{September 2025}
\begin{document}
 
\newcommand{\derive}[2]{\frac{\mathrm{d}#1}{\mathrm{d}#2}} 
 
\section{Induktion}  
Solange sich ein Magnetfeld um einem Leiter verändert, wird in diesem Leiter eine Spannung, die \emph{Induktionsspannung}, \emph{induziert}. \newline
Um diese Potenzialdifferenz auszugleichen fließt darauffolgend Strom, der \emph{Induktionsstrom}. Der Induktionsstrom ist immer so gerichtet, dass das  durch den Stromfluss entstehende Magnetfeld entgegen der Ursache der Induktion wirkt, besagt das Lenzsche Gesetz. 
\[
 U_{Ind} = -N \cdot \derive{\Phi}{t}
\] 
Mit der Definition vom \hyperref[Magnetischer Fluss]{Magnetischem Fluss} und der \hyperref[Ableitungsregeln]{Produktregel} lautet
\[
 \derive{\Phi}{t} = A \cdot \derive{B}{t} + B \cdot \derive{A}{t} 
\] 
Normalerweise wird aber zu jedem Zeitpunkt nur eine der beiden Größen verändert, sodass nur einer der beiden Summanden beachtet werden muss; der jeweils andere ist dann immer $0$.
Den weiteren Ableitungsregeln nach, hat ein sich konstant veränderndes Magnetfeld eine konstante $U_{Ind}$ und ein Magnetfeld, dessen Größe einer Sinusfunktion folgt, eine Induktionsspannung, welche der gleichen Sinusfunktion, um eine halbe Periode versetzt, folgt. 
 
\subsection{Qualitativer Versuch}
Die Induktion kann experimentell qualitativ untersucht werden, indem ein Magnet in der Nähe einer Spule, welche an einem Voltmeter angeschlossen ist, oder die Spule in der Nähe des Magneten bewegt wird. Dabei kann eine Spannung gemessen werden, welche von Geschwindigkeit der Veränderung des relativen Magnetfeldes, also der Geschwindigkeit und Position des Magneten relativ zur Spule, abhängig ist.
 
Gleiches gilt natürlich auch wenn das Magnetfeld des Magnetes durch ein Magnetfeld einer stromdurchflossenen Spule ersetzt wird.
 
Darüberhinaus wird die Induktion beobachtet, wenn sich das Magnetfeld aufgrund einer Veränderung der Stromsärke verändert.  
 
\subsection{Stromerzeugung in Generatoren}
Wird ein einem konstanten Magnetfeld eine Induktionsspule mit konstanter Geschwindigkeit gedreht, so entsteht Wechselspannung. Es herrscht ein konstante $B$, wobei aber die Fläche, auf welche die Feldlinien tatsächlich wirken, sich konstant verändert. Diese Fläche $A$ ist $A \sim \sin{\alpha}$, wobei $\alpha$ der konstant wachsende Winkel der sich drehende Spule ist, bei einer Rotationsgeschwindigkeit von $\omega$ gilt also $\alpha = \omega \cdot t$. Sodass gilt auch $\Phi \sim \sin{\omega \cdot t}$. Die Spannung ist also proportional zu $\displaystyle \derive{\Phi}{t}$ mit $\Phi \sim \sin{(\omega \cdot t)}$, sodass
\[
 U_{ind} \sim \omega \cdot \cos{(\omega \cdot t)}
\] 
 
\end{document}