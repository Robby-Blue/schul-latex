\documentclass{article}
\usepackage[a4paper]{geometry}
\usepackage{fancyhdr}
\pagestyle{fancy}
\lhead{Floureszenz}
\rhead{Februar 2026}
\begin{document}
\section{Floureszenz}
Licht kann, z.\,B. beim strahlen durch Olivenöl, innerhalb dessen die Farbe ändern.
 
Dies ist darin zu begründen, dass die Photonen der Wellenlänge absorbiert werden und die Elektronen der Ölatome auf ein höheres Niveau gebracht werden, aber nur schrittweise in den Grundzustand zurückspringen. Weil die einzelnen, mehreren Sprünge jeweils kleiner sind, haben die dabei emittierten Photonen andere Wellenlängen.   
\end{document}