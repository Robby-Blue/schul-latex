\documentclass{article}
\usepackage{amssymb}
\usepackage{amsmath}
\usepackage{hyperref}
\usepackage[a4paper]{geometry}
\usepackage{fancyhdr}
\pagestyle{fancy}
\lhead{Das Bohrsche Atommodell}
\rhead{Januar 2026}
\begin{document}
\section{Das Bohrsche Atommodell}
Basierend auf dem \hyperref[Der Frank-Hertz-Versuch]{Frak-Hertz-Versuch} und der \hyperref[Das Linienspektrum von Wasserstoffatomen]{Serienformel des H-Atoms} wurde das Bohrsche Atommodell erfunden.
 
\subsection{Annahmen}
Das Modell teilt zwei Annahmen 
\begin{enumerate}
 \item Elektronen bewegen sich auf diskreten, strahlungsfreien Bahnen. Dabei ist der Bahndrehimpuls $L=rmv$ hier
\[
 L = n \cdot \frac{h}{2 \pi}
 \qquad \text{mit} \quad
 n \in \mathbb{N},\,n \ge 1 
\] 
 \item Geht ein Elektron aus einer energiereicheren Bahn in eine energieärmere Bahn über, wird die Energiedifferenz als Photom emittiert. Dabei ist
\[
 h \cdot f = E_n - E_m 
\]  
\end{enumerate} 
 
\subsection{Bahnradius im H-Atom}
Um den Radius des Bahn des Elektrons um den Proton eines H-Atoms zu bestimmen, wird angemerkt dass $n=1$, so dass
\[
 \frac{h}{2\pi} = r \cdot m \cdot v
 \Longrightarrow r = \frac{h}{2\pi \cdot m \cdot r} 
\]
Dieses $r$ wird in die Gleichung, welche die Coloumbkraft und die Zentripetalkraft gleichsetzt eingesetzt, so folgt
\[
 r = \frac{h^2 \cdot \epsilon_0}{e^2 \cdot \pi \cdot m} = 5,3 \cdot 10^{-11}\,\text{m} 
\] 
\subsection{Lichtfrequenzen bei Bahnübergängen}
Die Gesamtenergie eines Elektrons auf der $n$ten Bahn ist natürlich
\[
 E_n = E_{kin} + E_{pot} 
\] 
Dabei ist $E_{pot}$ ein $E=W=f \cdot s$, wobei dieses $f=F_{\text{coloumb}}(r)$. Somit
\[
 E_{pot} = \int F_{\text{coloumb}}\,\text{d}r 
\]
In die nun gefundene Formel für $E_n$ können $r_n$ und $v_n$, gefunden beim berechnen des Bahnradius, eingesetzt werden. Nach dem kürzen folgt
\[
 E_n = -\frac{1}{8} \cdot \frac{m \cdot e^4}{\epsilon_0^2 \cdot h^2} \cdot \frac{1}{n^2} 
\] 
\subsection{Leistungen und Grenzen}
Das Bohrsche Atommodell konnte
\begin{itemize}
 \item genaue quantitative Ergebnisse für Elektronengrößen liefern
 \item die Vorgänge des H-Atoms sehr gut beschreiben
 \item die Wasserstoff Linienspektren erklären
 \item das Streuexperiment erklären.
\end{itemize} 
 
Dabei stand des aber im Wiederspruch: die Kreisbewegung der Elektronen ist eine beschleunigte Bewegung, welche e-m-Strahlung verursacht; es kann eine strahlungsfreie Bahn geben.
 
Darüberhinaus ist es nocht ohne weitere Probleme auf andere Atome übertragbar, hat probleme mit der Quantenphysik und geht davon aus, dass Atome flach seien. 
\end{document}