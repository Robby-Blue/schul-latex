\documentclass{article}
\usepackage{hyperref}
\usepackage[a4paper]{geometry}
\usepackage{fancyhdr}
\pagestyle{fancy}
\lhead{Die Gasentladungsröhre}
\rhead{Februar 2026}
\begin{document}
\section{Die Gasentladungsröhre}
Eine \emph{Gasentladungsröhre} ist eine bei geringem Druck mit Gas, z.\,B. He, gefüllte Röhre an dessen Enden jeweils eine Anode und eine Kathode anliegen, mit einer Spannung von einigen hundert Volt. Das Gas fängt dabei an zu leuchten. 
 
\subsection{Funktionsweise}
Angesichts der Elektronen, welche thermisch aus der Kathode gelöst und im elektrischen Feld stark beschleunigt werden, gibt es zwei Prozesse um Licht zu erzeugen.
\begin{enumerate}
 \item Ähnlich der \hyperref[Röntgenstrahlung]{charakteristischen Röntgenstrahlung} können hochenergetische Elektronen die He-Atome ionisieren. Daraufhin springen die Elektronen der He-Atome aus den höheren Energieniveaus in die entstandenen Lücken und geben dabei ihre Energie als ein Photon ab.
 \item Elektronen mit passender $E_{kin}$ können dem \hyperref[Der Frank-Hertz-Versuch]{Frank-Hertz-Versuch} nach die He-Atome anregen. Mit dieser Energie geht ein Elektron aus einer unteren Schale auf eine höhere Schale über. Beim Rückgang wird die gleiche Energie als Lichtquant emittiert.
\end{enumerate} 
\end{document}