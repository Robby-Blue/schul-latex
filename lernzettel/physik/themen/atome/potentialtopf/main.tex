\documentclass{article}
\usepackage{hyperref}
\usepackage{amsmath}
\usepackage{csquotes}
\usepackage[a4paper]{geometry}
\usepackage{fancyhdr}
\pagestyle{fancy}
\lhead{Der Potentialtopf}
\rhead{Januar 2026}
  
\newcommand{\ekin}[0]{E_{\text{kin}}} 
\newcommand{\epot}[0]{E_{\text{pot}}} 
 
\begin{document} 
\section{Der Potentialtopf}
Der \emph{Potentialtopf}, oder vereinfacht der \emph{lineare Potentialtopf}, veranschaulicht die Verteilung der Elektronen als Quantenobjekte in einem Atom.
 
\subsection{Aufbau} 
Elektronen haben dem Atomkern gegenüber eine $\epot$. Diese sinkt mit sinkender Distanz zum Atomkern.
 
So bildet sie einer Achse entlang einen \textquote{Topf} an $\epot$, aus welcher ein Elektron nicht rauskommen kann. Dieser als Graph aufgezeichnet werden, auf der $x$-Achse die Position von $0$ bis zu Länge des Topfes $L$, weshalb der Atomkern in der Mitte, bei $P_0$ liegt, und der $\epot$ auf der $y$-Achse. Weiter vereinfacht, mit unendlich hohen Wänden, gilt
\[ 
 \epot =  
 \begin{cases} 
  0, & 0 \le x \le L \\
  \infty, & \text{sonst}
 \end{cases}
\] 
 
\subsection{Als Welle}
Die Elektronen, welche auch als Welle angesehen werden können, liegen im Potentialtopf als stehende Welle vor. Dabei liegt der Grundzustand bei $n=1$. Somit folgt
\[
 L = n \cdot \frac{\lambda}{2} 
\]
 
Dem Pauli-Prinzip nach kann jedes $n$ von maximal zwei Elektronen besetzt werden.
 
\subsection{Wellenfunktion} 
Es folgt die Wellenfunktion $\psi(x,t)$. $\vert \psi \vert^2$ gibt die Auftreffwahrscheinlichkeit eines Elektrons an.
 
\subsection{Energie} 
Wird $\ekin$ genommen und in mit der Definition des Impulses in Abhängigkeit von $p$ gesetzt, so folgt
\[
 \ekin = \frac{1}{2m} p^2
\]
Nun kann das $p$ der \hyperref[Die de-Broglie-Wellenlänge]{de-Broglie-Wellenlänge} nach mit $lambda = h/p$ ersetzt werden, so dass nur noch ein $\lambda$ als Unbekannte vorliegt. Auch diese kann ersetzt werden, mit
\[
 \lambda = 2 \cdot \frac{L}{n} 
\]
Somit folgt
\[
 E = \frac{h^2}{8mL^2} \cdot n^2
\] 
\end{document}