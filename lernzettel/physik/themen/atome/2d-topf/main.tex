\documentclass{article}
\usepackage[a4paper]{geometry}
\usepackage{fancyhdr}
\pagestyle{fancy}
\lhead{Der mehrdimensionale Potentialtopf}
\rhead{Januar 2026}
\begin{document}
\section{Der mehrdimensionale Potentialtopf}
In einem mehrdimensionalen Potentialtopf können die unterschiedlichen Achsen unterschiedliche Werte für $n$ haben. Somit entstehen eine Vielzahl an möglichen Verteilungen. Deshalb wird ein $n$ aufgeteilt, in $n_x$, $n_y$ und $n_z$.
 
Die bevorzugten Aufenthaltsbereiche der Elektronen in diesem Topf können visualisiert werden.
 
\subsection{Der 2D-Topf}
In einem zweidimensionalen Potentialtopf liefert der erste angeregte Zustand, $n=2$ zwei mögliche Verteilungen. Jeweils eine Achse hat $n=2$, für die andere gilt $n=1$.
 
Die Aufenthaltsbereiche bestehen aus zwei Rechtecken, entlang der Seite dessen $n=1$ und geteilt in zwei auf der Seite dessen $n=2$.
 
\subsection{Der 3D-Topf}
Jeweils eine der drei Achsen hat ein $n=2$, beide anderen nutzen $n=1$. Daraus folgen drei möglichkeiten, eine pro Achse.
 
Die Aufenthaltsbereiche werden auf die gleiche Art und Weise des 2D-Topfes gezeichnet. 
 
\subsection{Coulombpotential}
Das Coulombpotential ist aber nicht würfelförmig. Deshalb sind die Aufenthaltsbereiche Pillenförmig, entlang jeweils einer Achse auf beiden Seiten des Ursprungs.
 
Diese Pillenförmigen Aufenthaltsbereiche sind \emph{Orbitale}, spezifisch die drei $p$-Orbitale $p_x$, $p_y$ und $p_z$. In jedem Orbital sind zwei Elektronen, mit jeweils unterschiedlichem Spin.
 
\end{document}