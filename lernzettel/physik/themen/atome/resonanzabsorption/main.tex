\documentclass{article}
\usepackage[a4paper]{geometry}
\usepackage{fancyhdr}
\pagestyle{fancy}
\lhead{Resonanzabsorption}
\rhead{Februar 2026}
\begin{document}
\section{Resonanzabsorption}
Strahlt eine NA-Dampf-Lampe durch Na-Dampf auf ein Schirm, so entsteht aufgrund des Na-Dampfes ein Schatten auf dem Schirm.
 
Dies liegt daran, dass die Na-Atome der Wolke die Photonen der passenden Wellenlänge absorbieren, wodurch diese in einen angeregten Zustand gebracht werden. Sobald das Atom den angeregten Zustand wieder verlässt, emittiert dieses die Atome in \textit{alle} richtungen. Somit landen weniger Photonen im Bereich hinter der Wolke. 
\end{document}