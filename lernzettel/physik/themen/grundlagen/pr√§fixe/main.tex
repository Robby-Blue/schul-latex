\documentclass{article}
\usepackage{amsmath}
\usepackage[a4paper]{geometry}
\usepackage{fancyhdr}
\pagestyle{fancy}
\lhead{Einheitenpräfixe}
\rhead{August 2025}
\newcommand{\fact}[1]{$10^{#1}$ \rule{0pt}{11pt}}
\begin{document} 
 
\section{Einheitenpräfixe}  
\begin{minipage}[t]{\dimexpr\linewidth-7cm}
 \vspace{0pt} 
 Hier sind die relevantesten Prefixe aller Einheiten. Um von der einen Einheit in die andere Umzuwandeln muss Wert mit der Differenz zwischen Exponent der Faktors der ersten Größe und der zweiten Größe multipliziert werden. \newline
Wird beispielsweise $5\,\text{km}$ in $\text{cm}$ umgewandelt, so folgt ${5 \cdot 10^{3-(-2)} = 500\,000 \,\text{cm}}$. Werden $5\,\text{cm}$ in $\text{km}$ umgewandelt, so folgt ${5 \cdot 10^{(-2)-3} = 0.00005 \,\text{km}}$.
\end{minipage}
\hfill
\begin{minipage}[t]{7cm}
 \vspace{0pt} 
 \begin{center}
 \begin{tabular}{ |c|c|c| }
 \hline
  \textbf{Name} & \textbf{Faktor} & \textbf{Abkürzung} \\
 \hline
  Giga & \fact{9} & G \\
 \hline
  Mega & \fact{6} & M \\
 \hline
  Kilo & \fact{3} & k \\
 \hline
  Dezi & \fact{-1} & d \\
 \hline
  Zenti & \fact{-2} & c \\
 \hline
  Milli & \fact{-3} & m \\
 \hline
  Mikro & \fact{-6} & $\mu$ \\
 \hline
  Nano & \fact{-9} & n \\
 \hline
 \end{tabular}
 \end{center}
\end{minipage}
 
\section{Zehnerpotenzen} 
Die relevantesten Zehnerpotenzen und deren Namen sind  
\begin{center}
\begin{tabular}{ |c|c|c| }
\hline
 \textbf{Name} & \textbf{Faktor} \\
\hline
 Tausend & \fact{3} \\
\hline
 Million & \fact{6} \\
\hline 
 Milliarden & \fact{9} \\
\hline 
 Billion & \fact{12} \\
\hline 
 Billiarden & \fact{15} \\
\hline
\end{tabular}
\end{center}
 
\end{document}