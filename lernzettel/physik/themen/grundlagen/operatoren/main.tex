\documentclass{article}
\usepackage[a4paper]{geometry}
\usepackage{fancyhdr}
\pagestyle{fancy}
\lhead{Operatoren}
\rhead{September - Oktober 2025}
\begin{document}
\section{Operatoren} 
\subsection{Bestimmen und Berechnen}
Muss etwas \emph{bestimmt} werden, so muss nur der Ansatz und das Ergebnis aufgeschrieben werden. Bei einer \emph{berechnung} darf eine eigene Rechnung, inklusive Umformungen, nicht fehlen.
 
\subsection{Beurteilen} 
Der Operator \emph{beurteilen} kann bearbeitet werden indem
\begin{enumerate}
 \item Alle relevanten Kriterien aufgeschrieben werden
 \item Die Kriterien gewichtet werden, wobei die Summe natürlich $100\%$ bleiben muss
 \item Mithilfe der Gewichtung, in Prozent, jedem Kriterium eine maximale Punktzahl zugeordnet wird, so dass es sowohl einfach zu rechnen als auch prozentual gesehen korrekt gewichtet ist 
 \item Alle Möglichkeiten mithilfe der Kriterien bewertet werden, die Möglichkeit mit der insgesamt höchsten Punktzahl bestimmen
 \item Ein Fazit gezogen wird, in einem kurzen Satz mit einer kleinen Begründung 
\end{enumerate} 
\end{document}