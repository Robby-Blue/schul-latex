\documentclass{article}
\usepackage{multicol}
\usepackage{amssymb}
\usepackage[a4paper]{geometry}
\usepackage{amsmath}
\usepackage{fancyhdr}
\pagestyle{fancy}
\lhead{Umgang mit dem Taschenrechner in Physik}
\rhead{August 2025}
\begin{document}
\section{Umgang mit dem Taschenrechner in Physik} 
\subsection{Einheiten und Konstanten}
Alle relevanten Einheiten und Konstanten sind im Taschenrechner vordefiniert. Sie können durch ihren Namen, welcher immer mit einem Unterstrich beginnt, aufgerufen werden, ähnlich wie eine Variable. Der Unterstrich ist im Menu der \texttt{?!$\blacktriangleright$}-Taste aufzufinden, die gesamten Einheiten und Konstanten im Menu der Buchtaste. Der Taschenrechner kann alle Einheiten zusammenfassen und umwandeln. Manuell in eine bestimmte Einheit umgerechnet werden kann mit dem Umrechnungsassistenten oder als Rechnung mit dem Umrechungssymbol, beide im Menu der \texttt{?!$\blacktriangleright$}-Taste. \newline
Der Name der meisten Konstanten ist offenslichtlich, manche ausnahmen gibt es jedoch.
\begin{multicols}{3}
 \begin{description}
  \item[Lichtgeschwindigkeit] \texttt{\_c} 
  \item[Elektronenmasse] \texttt{\_Me}
  \item[Elektronenladung] \texttt{\_q} 
 \end{description} 
\end{multicols}  
\subsection{Regressionen}
Eine gute Regression wird im \texttt{Data \& Statistics} Menu angefangen. In der obersten Zeile müssen den einzelnen Spalten Namen zugeordnet werden. In der Zeile darunter kann eine Funktion bestimmt werden, mit welcher die Spalte automatisch befüllt wird, dabei kann einfach der Name einer anderen Spalte wie ein Variablenname verwendet werden. In den folgenden Zeilen können alle tatsächlichen Messwerte angegeben werden.
 
Nach dem Eintragen kann mit \texttt{ctrl} + \texttt{doc}, also \texttt{+ page}, eine neue \texttt{5 Data \& Statistics} hinzugefügt werden. In diesem Diagramm können die Spalten den Achsen zugeordnet werden. Darauffolgend kann mit \texttt{menu}, \texttt{4}, \texttt{6} das Menu zur Auswahl der Regressionsart aufgerufen werden.
 
\end{document}