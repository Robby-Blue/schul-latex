\documentclass{article}
\usepackage{hyperref}
\usepackage[a4paper]{geometry}
\usepackage{fancyhdr}
\pagestyle{fancy}
\lhead{Der Kohärenzbegriff}
\rhead{September 2025}
\begin{document}
\section{Der Kohärenzbegriff} 
Zwei Lichtquellen werden \emph{Kohärent} genannt, wenn ihre Phasendifferenz zeitlich konstant bleibt. Ist dies nicht der Fall, sei es weil die Lichter unterschiedliche Frequenzen hat oder von unterschiedliche, ungenauen, Lampen kommen, so sind diese weniger Kohärent.
 
Trotzdem können zwei zueinander kohärente Lichtstrahlen erzeugt werden, indem das Licht von einer Lichtquelle (welche mit sich selbst Kohärent ist) in zwei geteilt wird. Es ist sich der \hyperref[Der Doppelspalt]{Doppelspalt} vorzustellen, wobei beide Spalten mit dem Licht nur einer Quelle scheinen.
\end{document}