\documentclass{article}
\usepackage{amsmath}
\usepackage[a4paper]{geometry}
\usepackage{fancyhdr}
\pagestyle{fancy}
\lhead{Messung der Lichtgeschwindigkeit}
\rhead{September 2025}
\begin{document}
\section{Messung der Lichtgeschwindigkeit}
Die Lichtgeschwindigkeit wurde historisch von den verschiedensten Personen auf unterschiedliche Art und Weise gemessen.
 
\subsection{Fizeau} 
Eine dieser Personen war \emph{Hippolyte Fizeau}. Dieser baute ein Versuch auf, bei welchem er Licht durch die Lücken eines $720$-Zähnigem Zahnrad scheinen ließ, welches nach $8.6\,\text{km}$ wieder reflektiert wurde und wieder durch die gleiche Zahnradlücke zurücknahm und messbar wurde. Nun fing er an, das Zahnrad mit einer konstanten Geschwindigkeit zu drehen.
 
Hörte das Licht auf, sichtbar zu sein, so heißt dies, dass das allererste ausgesandte Licht, welches die Lücke durchquerte, sobald die da war, erst dann zurücknahm, als die Lücke schon vorbei gedreht wurde. Heißt, das Licht hat. Heißt, dass das Licht die $8.6\,\text{km}$ insgesamt $2$-mal zurückgelegt hat, in der Zeit, welche das Rad brauchte, um sich um eine Zahnrad zu drehen. Gibt es jeweils 720 Räder und Lücken, sind das bei einer Rotationsdauer $T$ dann $t=T/1440$.
 
Ist bekannt, wie lange das Licht für $17.2\,\text{km}$ brauchte, folgt die Lichtgeschwindigkeit. 
 
\subsection{Lichtgeschwindigkeit in Medien}
Die Lichtgeschwindigkeit in einem Medium $c_m$, dessen Brechungsindex $n$ ist, liegt bei
\[
 c_m = \frac{c_0}{n}
 \qquad \text{mit} \qquad
 n = \sqrt{\varepsilon_r \cdot \mu_r}  
\]
\end{document}