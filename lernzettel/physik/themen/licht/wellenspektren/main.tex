\documentclass{article}
\usepackage{csquotes}
\usepackage[a4paper]{geometry}
\usepackage{fancyhdr}
\pagestyle{fancy}
\lhead{Wellenspektren}
\rhead{Januar 2026}
\begin{document}
\section{Wellenspektren} 
Ein Wellenspektrum beschreibt die zuweisung der Lichtintensität aller Wellenlängen eines Lichtes.
 
\subsection{Bestimmung}
Ein Wellenspektrum kann bestimmt werden, indem das zu messene Licht mit einem Prisma oder einem Gitter in ein Interferenzmuster aufgeteilt. Somit liegt nun auf dem Schirm das Licht alle einzelnen Wellenlängen nebeneinander vor. Dieses kann von einem lichtempfindlichen Widerstand, einem LDR, \textquote{abgefahren} werden und die Leitfähigkeit des Widerstands gemessen werden.
 
Der LDR ist ein Halbleiterelement aus z.\,B. Silizium. Jedes auftreffende Photon löst eine Valenzelektron, so dass die Leitfähigkeit zur Anzahl der auftreffenden Photonen proportional ist.
 
Somit folgt aus einer höheren Lichtintensität eine höhere Leitfähigkeit und das Gegenteil kann geschlussfolgert werden. 
\end{document}