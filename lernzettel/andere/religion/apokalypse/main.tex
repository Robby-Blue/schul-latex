\documentclass{article}
\usepackage{csquotes}
\usepackage[a4paper]{geometry}
\usepackage{fancyhdr}
\pagestyle{fancy}
\lhead{Apokalypse}
\rhead{November 2025}
\begin{document}
\section{Apokalypse}
Es gibt $11$ hauptsächliche Merkmale von apokalyptischen Geschichten.
\begin{description}
 \item[Apokalypsis] Die Enthüllung einer verborgenen Wirklichkeit, hier insbesondere in Bezug auf die Offenbarung von einer göttlichen Wahrheit
 \item[Bildsprache \& Chiffren] Bilder, Symbole und Chiffren mit tieferer Bedeutung, damit nur Eingeweihte den Text verstehen und außenseiter abgegrenzt werden können.
 \item[Dualismus] Es wird ein Dualismus zwischen dem unrechten Jetzt und der gerechten Zukunft, nach der Offenbarung, aufgebaut. 
 \item[Periodisierung] Einteilen der Geschichte in Abschnitte (Epochen), welche sich zum Ende der Welt hin entwickeln. Die Epoche zur Zeit des Autors ist die schlimmste. 
 \item[Gericht und Endgültigkeit] Das endgültige Urteil Gottes richtet über jeden Einzelnen. 
 \item[Pseudonymität, Rückdatierung] Texte werden unter falschen Namen, unter verfälschten Ver-öffentlichungsjahren präsentiert. Damit es so scheine, als ob alles schon feststehe, schon lange vorhergesagt sei. 
 \item[Gegenwartsrealismus, eschatologischer Optimismus] Die Wahrnehmung gegenüber dem Dualismus. Ein \textquote{Realismus} (Pessimismus) gegenüber der Gegenwart, ein Optimismus gegenüber der Zukunft.
 \item[Kosmische Symbolik] Bilder, Zeichen aus der Natur für übernatürliche Zusammenhänge, gött-liches Handeln
 \item[Zahlensprache, Begrenzungsformeln] Zahlensprache (feste Ereignisse/Anzahl) und zeitliche Begrenzungen zur förderung der Glaubwürdigkeit und dass es nah ist
 \item[Kollektive Perspektive] Das Volk Gottes als Gemeinschaft, welche am ende belohnt werden, nachdem sie gelitten hatten.
 \item[Zentralfigur (Menschensohn)] Himmlisches Wesen, verkörpert Gottes Herrschaft, erscheint zu Ende als Richter, Retter. Steht für Hoffnung auf neue, bessere, Welt. 
\end{description}  
\end{document}