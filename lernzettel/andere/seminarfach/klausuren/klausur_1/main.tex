\documentclass{article}
\usepackage{darkmode}
\enabledarkmode
\usepackage{svg}
\usepackage[a4paper]{geometry}
\usepackage{fancyhdr}
\pagestyle{fancy}
\lhead{Seminarfach}
\rhead{Dezember 2024}
\begin{document}
 
\section{Theorien}
\subsection{Gender Studies}
Anblick auf das Gender und dessen darstellung. Ob es und die damit verbundenen Stereotype selbst nutzt oder dekonstruiert. Ob es binarismen beinhaltet. Ob es den Male Gaze o.ä. gibt.
\subsubsection{Intersektionalität}
Analyse der überschneidung verschiedener Diskriminierungskategorien
 
\subsection{Postkolonialismus}
Analyse nach den Nachwirkungen des Kolonialismus.
\subsubsection{Othering}
Das Aufteilen von Gruppen in "uns", als politische Mehrheit, und "die Anderen", oftmals als "komisch" oder "exotisch" o.ä. angesehen, als \textbf{binäre Gruppe}. Um Machtstrukturen zu rechtfertigen, Stereotypisierung, Entmenschlichung und Unsichtbarmachung.
 
\subsubsection{Binarismen}
Einteilung in Binarismen \newline
Beispiele 
\begin{center}
\begin{tabular}{ |c|c| }
\hline
 normal & komische \\
\hline
 zivilisiert & primitiv \\
\hline 
 modern & traditionell \\
\hline
 rational & emotional \\
\hline
\end{tabular}
\end{center}
  
\subsection{White Saviour Trope}
Wenn der weiße Held alle anderen rettet, weil sie es selbst nicht können. Zählt eig auch zu binarismen, der weiße kann alles und rettet, die anderen sind zu "primitiv" um sich selbst zu retten und brauchen hilfe.
 
\begin{description}
\item[Zentrierung der weißen Figur]
\item[Entmündigung der marginalisierten Gruppe]  
\item[Moralische Erlösung des "Retters"]
Fokusiert sich auf den Story Arc und die Veränderung der weißen Figur, nicht auf die Realität der geretteten Figur.  
\item[Ignoranz gegenüber strukturellen Problemen]
\end{description}
 
\newpage
 
\section{Kamera} 
Einstellungsgrößen
\begin{center}
\begin{tabular}{ |c|c| } 
\hline
 Weit & Umfeld, Überblick \\
\hline
 Totale & große Gruppe an Personen, ein Raum \\
\hline
 Halbtotale & paar Personen \\
\hline
 Amerikanisch & Amerikanisch \\
\hline
 Halbnah & Hüfte aufwärts \\
\hline
 Nah & mitte Oberkörper \\
\hline
 Gross & konzentriert auf Kopf \\
\hline
 Ganz Gross / Detail & Teil des Gesichts \\
\hline
\end{tabular}
\end{center} 
und Perspektiven
 
\begin{center}
\begin{tabular}{ |c|c| }
\hline
 Normalsicht & Augenhöhe \\
\hline
 Aufsicht/Obersicht/Vogelperspektive & von Oben \\
\hline
 Untersicht/Froschperspektive & von Unten \\
\hline
\end{tabular}
\end{center}
Kameras können sich bewegen
 
\section{Dramaturgie}
In zwei Formen 
\begin{center}
\begin{tabular}{ |c|c|c| }
\hline
 & \textbf{geschlossene Form} & \textbf{offene Form} \\
\hline
 Struktur & linear, kausal & episodisch \\
\hline 
 Ende & klar & unklar, offen \\
\hline 
 Erzählwise & klar & interpretativ \\
\hline 
\end{tabular}
\end{center} 
  
\subsection{Anfang, Beginn und Exposition}  
\begin{description}
\item[Anfang] Ersten Bilder
\item[Beginn] Nach Vorspann beginnt Geschichte
\item[Exposition] zeigt Figuren und Handlungssituation 
\end{description} 
 
\subsection{Wendepunkt, Konflikte, Höhepunkt}  
\begin{description}
\item[Wendepunkt] steuert auf Konflikt zu, oft erstmals unerkennbar
\item[Konflikte] Konflikt
\item[Höhepunkt] Intressantester Punkt, führt zu Lösung
\end{description} 
 
\subsection{Schluss}  
oftmals Happy End mit Stabilität und Sicherheit, manchmal tragischer Schluss
 
\section{Erzählstrategien} 
\subsection{POV}
\begin{description}
\item[Auktoriale Erzähler] aka allwissender Erzähler weiß alles. Dazu Zählt sowohl die \textbf{Außenperspektive}, was passiert, als auch die \textbf{Innenperspektive}, die Gefühle und Gedanken spezifische Figuren
\item[Ich-Erzähler] Aus der Perspektive einer Figur, weiß nur, was die Figure wissen kann
\end{description}
 
\subsection{Zeit}
Erzählzeit und erzählte Zeit, Zeitraffung und Zeitdehnung und Vorgreifen und Rückwenden/blenden, selbsterklärend
 
\section{Figuren}
\subsection{Figurenkonstruktion}
Beschreibt eine Figure bzw. die Figuren an sich, deren Geschichte, Eigenschaften, Ziele, Entwicklungen, etc 
 
\subsection{Figurentypologie} 
Der Typus einer Figur, z.B. als Held*In, Protagonist*In, Antagonist*In, Sidekick, etc 
 
\subsection{Figurenkonstellation}
Die Beziehungen der einzelnen Figuren untereinander, welche Freundschaften, Feindschaften, Konflikte, Hierachien, etc es gibt. \newline
Äußerer Konflikt mit jemand anderem, innerer Konflikt einer Figur mit sich selbst
 
\section{Schnitt}
Damit das geschehen Ununterbrochen wirkt, z.B. Dialoge als Schuss-Gegenschuss-Verfahren
 
\end{document}
 
 
 
