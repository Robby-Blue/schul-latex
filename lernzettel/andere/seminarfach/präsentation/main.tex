\documentclass{article} 
\usepackage{csquotes}
\usepackage[utf8]{inputenc}
\usepackage{xcolor}
\usepackage{soulutf8}
\usepackage[ngerman]{babel}
\usepackage[a6paper, top=9mm, headsep=3mm, left=5mm, right=5mm, bottom=15mm]{geometry}
\usepackage{fancyhdr}
\pagestyle{fancy}
\lhead{Seminarfach}
\rhead{1492} 
 
\colorlet{hlred}{red!30}
\colorlet{hlyellow}{yellow!70}
 
\newcommand{\note}[1]{\sethlcolor{hlred}\hl{\texttt{[#1]}}}  
\newcommand{\imp}[1]{\sethlcolor{hlyellow}\hl{#1}}
  
\newcommand{\nl}{\newline} 
  
\newcommand{\stitle}[1]{\noindent\textbf{#1}\nl} 
 
\begin{document}
\stitle{konstellation} 
\note{kolumbus} \imp{Christoph Kolumbus} als \imp{Protagonist} im Mittelpunkt. \nl
Er stellt sein Plan einer \imp{Kommission der Universität Salamanca} vor \note{salamanca}, diese \imp{lehnt ab}, weil sie denkt, dass der Weg weiter sei als Kolumbus von ausgeht. \nl
Erster \imp{Antagonist}, welcher \imp{aktiven Widerstand} leistet. Über einen Umweg kommt er dann in Kontakt mit der \imp{Königin} \note{isabella}, welche ihm seinen Plan erlaubt.
 
Es kommt zu einem \imp{inneren Konflikt} mit sich selbst, er hat über die \imp{Distanz gelogen}, damit seine Mannschaft mitkommt \note{kolonisten}. Er \imp{gesteht} seine \imp{Lügen} einem \note{beichtvater}. Das zeigt, dass nicht perfekt ist, sich seiner \imp{moralischen Inkorrektheit} bewusst ist. Dass er besonders \imp{Ambitioniert} und \imp{Ehrgeizig} ist, dass seine Pläne ihm wichtiger sind als die Wahrheit.
 
Bei der 2. Ankunft sind alle 39 \imp{Europäer tot}. \note{Gespräch} Kolumbus und \note{kazike} mit \note{utapán} als Dolmetscher. Ureinwohner meinen es \imp{nicht gewesen} zu sein. Kolumbus glaubt ihnen, wollen zusammen arbeiten.
 
Neuer Charakter, Antagonist, \imp{Adrián de Moxica}. Widerspricht \note{moxica} Kolumbus, \note{beschuldigt} die Ureinwohner, welche er \textquote{\imp{Affen}} nennt, des Mordes und fordert die \imp{Todesstrafe}. \imp{Kontrast} zwischen Kolumbus und de Moxica, zwischen Humanismus und Rassismus. De Moxica als besonders unsympathisch und rassistisch damit Kolumbus im \imp{Kontrast} als Protagonist sympathischer wirkt. \nl 
Höhepunkt als Ureinwohner meint \imp{Zwangsabgaben} nicht zahlen zu können. Die Ureinwohner müssen \imp{Gold sammeln} und zahlen. De Moxica hackt ihm die Hand ab \note{unbenannt}. Kolumbus ist dagegen, will ihn rechtsstaatlich \note{bestrafen}. Zeigt weiter den Kontrast. Kolumbus setzt sich \imp{persönlich} ein um \imp{Ungerechtgkeiten} beim \imp{Einsammeln der Zwangsabgabe} zu beseitigen, zeigt der Film.
aber. Wäre nicht besser, \imp{strukturelle Probleme} die beim Einsammeln der Zwangsabgaben zu Ungerechtigkeit führen? oder. Warum gibt es überhaupt Zwangsabgaben? Ureinwohner sammeln, Europäer \imp{sammeln ein} oder Soldaten daneben. Ungerecht, rassistisch, kolonial. Warum gibt es das? --- Eine Frage, keine Antwort im Film. Film ignoriert \imp{systematische Diskriminierung}. \imp{unbenannt}. Keine Figur unternimmt dagegen, obwohl können. Kolumbus als \imp{Gouverneur}. Keiner benennt das \imp{Problem}. \imp{White Saviour}
 
Nachdem Kolumbus bestrafen will zetteilt eine Moxica eine \imp{Rebellion} an \note{rebellion}. Kolumbus diese nieder \note{niederschlag}. Bestraft mit \imp{Todesstrafe}. Innerer Konflikt: am Anfang vom Film \note{inquisition} töten \note{kezter}, Kolumbus und Sohn gucken zu. Kolumbus gegen Todesstrafe, nun fühlt sich gezwungen, \imp{selber nutzen}. 
 
Aufgrund von Aussage von \note{mönch} gegenüber Königin musser zurück nach Spanien, \imp{verliert Amt und Ehre}. Sohn \note{fernando} schreibt auf um Ehre zurückzubekommen.
 
\stitle{konstruktion} 
Ausführliche Konstellation. Weiter mit den Figuren selbst.
 
\imp{Genauer} \note{kolumbus}. Wie bereits beschrieben \imp{nicht perfekt, positiv, sympathisch, nicht rassistisch}. Im Film Arbeit und Ambition im \imp{Vordergrund}.
Weiter den \imp{Bau der Kirche} Kolonie \imp{La Isabela} angucken. Besonder für Adel: \note{arbeitet mit}, erwartet Gleiches von anderen. Verausgabt sich, auch körperlich. \imp{Arbeit wichtiger} als \imp{Aussehen, Ruhm und Ehre}. Lieber Arbeit mit \imp{ungewaschenen zerzausten Haaren} und \note{~} \imp{einfachen und nicht-mehr-weißen, eher dreckigen und zerissenen} Hemd. Normal. Volksnah. Bodenständig. Sympathisch konstruiert. \nl
Zusammen mit \imp{zwei inneren Konflikten} dass \imp{Lügt für Ziel} oder Thema der \imp{Todesstrafe}, in sich komplexe Konstruktion. 
 
\imp{Gegenteil}: \note{moxica}. Arbeitet nicht. Kurz vor \imp{Glockenbau} kommt mit \imp{Pferd} vorbei, arbeitet aber nicht. Auch Pferd soll \imp{nicht arbeiten}. So abgehoben, will \imp{nicht mal Pferd} arbeiten lassen, mind neben den \imp{Normalen}. Nach Gruppenzwang überreden ihn Pferd arbeiten zu lassen, er verschwindet aber. Währenddessen Trinkt in \imp{schicker Kleidung Wein}, von \imp{Dienerin gebracht}. Ruhm über Arbeit. Mit bereits beschriebenem Rassimus sehr unsympathisch konstruiert.
 
Genug Europäer: \note{kazike}, wer ist das? Kazike und für \imp{Plot relevant}, aber Kazike ist kein Name, lernt man in E-Phase; ein Titel. Antwort: Hat \imp{keinen Namen}, nicht im Film genannt. Was hat er denn? Eine Meinung; will mit \imp{Kolumbus arbeiten}. Nich wie, nur dass. Sehr oberflächlich. Äußert nur, als drauf angesprochen. Passiv. \imp{Fast keine Konstruktion}, nur eine oberflächliche Meinung und aussehen, mit Schmuck. \nl 
\note{abgabenzahler}, wer ist das? \imp{Niemand weiß}. Auch er, wie Kazike, \imp{relevant für Plot} aber keine Konstruktion. Kein Name, keine \imp{Meinung}, wenig \imp{eigenes Handeln}. Europäern wollen aktiv Gold, Europäer bestrafen ihn aktiv, er fand nur nichts, nicht wirklich aktiv. \nl 
\note{utapan}, wer ist das? Utapán, dolmetscher von Kolumbus. \imp{Einziger} Ureinwohner mit einem Namen. Läuft weg, Meinung: \imp{beschwert} sich über Kolumbus, Sprache lernen. Geradeso aktiv. Eigene Konstruktion mit \imp{eigener Meinung} und \imp{aktiver Handlung}. \imp{Kein anderer} Ureinwohner so weit konstruiert.
 
\imp{Keine Meinung untereinander}, \imp{kein Gespräch} miteinander ohne Europäer daneben.  
 
Waren alle 3 eigenen, relevanten Ureinwohner. Ureinwohner als \note{gruppe}. Im Stamm \imp{gleiches Aussehen}. Gleiche \imp{Kleidung, Frisur, Körperbau}, gleiches \imp{Alles}. Keine Meinung, \imp{Keine} eigene Konstruktion. \nl
Intressantes \imp{Handeln}: Erstes \note{treffen} mit Europäern, beide gleiche Situation, beide haben die anderen noch nie gesehen. \imp{Anderes Verhalten}. Ureinwohner \imp{unsicherer, verwirrt}, wenn nicht sogar \imp{dümmer}, dargestellt. \imp{Tasten} sich \imp{langsam} ran, \imp{glauben} ihren \imp{Augen nicht}, müssen \note{anfassen}. Europäer stehen da. Nicht verwirrt. \imp{Selbstsicher}. \imp{Binarismen}: schwach, \imp{primitiv}, \imp{dumm} gegen stark.
 
\stitle{zusammenfassung}
Zusammengefasst sieht hier erste Stelle viele Europäer. Ureinwohner in den Hintergrund. Die, die \imp{relevant} sind, für Plot, sind \imp{doch irrelevant}. \imp{Passiv}, \imp{wenig Konstruktion}, \imp{oberflächliche Meinung}, \imp{wenig Interaktion untereinander}. Vergleich zu Europäern mit komplexen Konstellation, Konstruktionen. \imp{Reden} miteinander, haben komplexe \imp{Meinungen}, \imp{innere Konflikte}, wie die 2 von Kolumbus. Innerer Konflikt von Ureinwohner im Film unvorstellbar. \nl
Stellt Kolumbus durchaus positiv, \imp{antirassistisch} dar, \imp{setzt sich} für Ureinwohner \imp{ein}, während \imp{systematische Probleme} ignoriert werden, reproduziert \imp{White Saviour Trope} und \imp{Binarismen}.
\end{document}