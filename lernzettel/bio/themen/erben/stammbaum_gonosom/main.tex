\documentclass{article}
\usepackage[a4paper]{geometry}
\usepackage{fancyhdr}
\pagestyle{fancy}
\lhead{Gonosomale Analyse von Stammbäumen}
\rhead{Dezember 2025}
\begin{document}
\section{Gonosomale Analyse von Stammbäumen} 
Bei gonosomaler Analyse wird davon ausgegangen, dass die Allelen auf einem Gonosom liegen, entweder auf einem \texttt{X} oder dem \texttt{Y}.
 
Hier ist es wichtig zu beachten, dass rezessive Allelen sich ausbilden können, auch wenn sie nur ein mal vorhanden sind, wenn es keine andere, dominante, Allele gibt.
 
\subsection{Das \texttt{Y}-Chromosom}
Weil die Nachkommen immer das Y-Chromosom des Vaters und kein anderes Chromosom haben werden, werden sie immer die gleiche Ausprägung haben wie der Vater. Somit ist eine gonosomale Vererbung über das Y-Chromosom trivial zu erkennen.
 
\subsection{Das \texttt{X}-Chromosom}
Die Analyse ist durchaus komplexer, wenn das Allel auf dem X-Chromosom liegt, weil dieses mehrfach vorhanden ist.
 
Haben zwei Elternteile mit einer Ausprägung ein Kind ohne dieser, so kann die Vererbung werder gonosomal-rezessiv noch gonosomal-dominant stattgefunden haben, sie muss autosomal gewesen sein.
 
Gegenteilig, zwei Elternteile ohne Ausprägung haben ein Kind mit einer Ausprägung, so findet die Vererbung rezessiv statt.
 
Dass ein Vater ohne Ausprägung und eine Mutter mit Ausprägung einen Sohn mit Ausprägung haben kann sowohl rezessiv als auch dominant passieren.
 
Alle anderen Fälle passieren dominant. 
 
\end{document}