\documentclass{article}
\usepackage[a4paper]{geometry}
\usepackage{fancyhdr}
\pagestyle{fancy}
\lhead{Stammbäume}
\rhead{December 2025}
\begin{document}
\section{Stammbäume}
Stammbäume zeigen den Verlauf einer biologischen Familie.
 
\subsection{Darstellung}
Lebewesen werden durch Symbole dargestellt. Dabei werden weibliche Lebewesen durch einen Kreis, männliche Lebewesen durch ein Viereck dargestellt. Horizontale Linien zwischen zwei Lebewesen sind sexuelle Beziehungen. Diese enden nach einer vertikalen Verbindung mit weiteren Lebewesen weiter unten, den Kindern der ersten beiden.
 
Markierte, z.\,B. ausgemalte, Lebewesen besitzen eine ausgeprägtes Merkmal in dessen Phänotyp.
 
Beim Darstellen des Genotyps werden dominante Allelen durch ein Großbuchstaben, rezessive Allelen durch einen Kleinbuchstaben dargestellt. In der Regel ist dies ein \texttt{A} beziehungsweise ein \texttt{a}.
 
\subsection{Nutzen}
Mithilfe eines ausreichend komplexen Stammbaumes einer Familie kann gefolgert werden, ob ein bestimmtes Merkmal rezessiv oder dominant oder autosomal (über ein Autosom vererbt) oder gonosomal (über ein Gonosom vererbt) ist. Dafür werden sich zwei Elternteile und eines von dessen Nachkommen angeguckt.
\end{document}