\documentclass{article}
\usepackage[a4paper]{geometry}
\usepackage{fancyhdr}
\pagestyle{fancy}
\lhead{Autosomale Analyse von Stammbäumen}
\rhead{Dezember 2025}
\begin{document}
\section{Autosomale Analyse von Stammbäumen}
Wird davon ausgegangen, dass die Vererbung eines Merkmals autosomal geschieht, kann geschlussfolgtert werden, ob diese dominant oder rezessiv passieren muss.
 
\subsection{Vorgang}
Für alle drei beteiligten Lebewesen werden die möglichen Genotype aufgeschrieben, angenommen die Vererbung sei rezessiv beziehungsweise dominant. Dabei wird darauf geachtet, ob eine der beiden Annahmen einen widerspruch erzeugt, indem diese Beispielsweise voraussetzen würde, dass bestimmte allele Chromosomen an das Kind vererbt werden müssten, welche aber nicht in den Eltern vorhanden sind. 
 
\subsection{Beobachtung} 
Wird der obige Schritt für alle Möglichkeiten durchgegangen, so fällt auf, dass wenn nur eines der beiden Elternteile das Merkmal hat oder alle drei haben das Merkmal haben oder es nicht haben, nichts geschlussfolgert werden kann.
 
Haben beide Eltern das Merkmal, das Kind aber nicht, so ist es dominant. Haben beide Eltern das Merkmal nicht, das Kind aber, so muss es rezessiv sein.  
\end{document}