\documentclass{article}
\usepackage{tikz}
\usepackage{mhchem}
\usepackage{amsmath}
\usepackage[a4paper]{geometry}
\usepackage{fancyhdr}
\pagestyle{fancy}
\lhead{Das Aktionspotential}
\rhead{Dezember 2025}
\begin{document}
\section{Das Aktionspotential}
Erreicht ein Reiz die Dendriten, so führt dieser zu einer elektrischen Erregung, dem \emph{Rezeptorpotential}. Die Zelle wird leicht depolarisiert. Dieses nimmt mit der Distanz leicht ab, bis es am Axonhügel ankommt, wo ein \emph{Aktionspotential} entstehen kann.
 
\subsection{Ablauf}
Unter dem Strich werden einfach nur, wenn der Schwellenwert erreicht wird, \ce{Na+}-Ionenkanäle geöffnet und geschlossen, wodurch die Zelle depolarisiert wird bevor \ce{K+}-Ionenkanäle geöffnet und geschlossen, wodurch die Zelle re- und hyperpolarisiert wird.
 
\begin{description}
 \item[Vordepolarisation] Aufgrund des Rezeptorpotentials kommt es zur leichten Depolarisation am Axonhügel, heißt die Polarisation sinkt leicht.
 \item[Depolarisation] Wird ein Schwellenwert von $-50\,\text{mV}$ erreicht, so öffnen sich spannungsgesteuerte \ce{Na+}-Ionenkanäle, das \ce{Na+} strömt in den Zellinnenraum. Die Spannung steigt auf $+35\,\text{mV}$.
 \item[Repolarisation] Die \ce{Na+}-Ionenkanäle schließen sich wieder, aufgrund der hohen Spannung öffnen sich \ce{K+}-Ionenkanäle, das \ce{K+} strömt aus der Zelle raus. Die Spannung wird erneut negativ.
 \item[Hyperpolarisation] Das Potential erreicht einen Wert negativer als das Ruhepotential.
 \item[Ruhepotential] Alle spannungsgesteuerten Ionenkanäle schließen sich wieder, das Ruhepotential kann sich erneut bilden. 
\end{description} 
\begin{center}
\begin{tikzpicture}
 \newcommand{\y}[1]{
  ({#1}+100)/(150/8)
 }
 \newcommand{\marker}[1]{
  \draw[thick] (-0.1, {\y{#1}}) -- (0.1, {\y{#1}});
  \node at (-0.3, {\y{#1}}) {#1};
 }  
 
 \newcommand{\culine}[7]{
  \pgfmathsetmacro{\xzero}{#1}
  \pgfmathsetmacro{\yzero}{\y{#2}}
  \pgfmathsetmacro{\xone}{#3}
  \pgfmathsetmacro{\yone}{\y{#4}}
  \pgfmathsetmacro{\dyzero}{#5}
  \pgfmathsetmacro{\dyone}{#6}
  \pgfmathsetmacro{\dx}{\xone-\xzero}
  \draw[draw=#7,domain=\xzero:\xone,smooth,samples=100,thick] plot (\x,{
    (2*((\x-\xzero)/\dx)^3 - 3*((\x-\xzero)/\dx)^2 + 1)*\yzero
    + (((\x-\xzero)/\dx)^3 - 2*((\x-\xzero)/\dx)^2 + (\x-\xzero)/\dx)*\dx*\dyzero
    + (-2*((\x-\xzero)/\dx)^3 + 3*((\x-\xzero)/\dx)^2)*\yone
    + (((\x-\xzero)/\dx)^3 - ((\x-\xzero)/\dx)^2)*\dx*\dyone
  });
} 
 
 \draw[->] (0, 0) -- (12, 0) node[above]{$t$ in ms};
 \draw[->] (0, 0) -- (0, 8) node[above]{$U$ in mV};
 
 \marker{-70} 
 \marker{-50}
 \marker{0}
 \marker{40}
 
 \draw[green!50!black] (0, {\y{-70}}) -- (2.5, {\y{-70}});
 \culine{2.5}{-70}{4}{-50}{0}{2}{orange}; 
 \culine{4}{-50}{6}{20}{2}{0}{red};
 \culine{6}{20}{8}{-70}{0}{-4}{cyan};
 \culine{8}{-70}{8.75}{-95}{-4}{0}{blue};
 \culine{8.75}{-95}{9.5}{-70}{0}{0}{blue};  
 \draw[green!50!black] (9.5, {\y{-70}}) -- (12, {\y{-70}});
 
\end{tikzpicture} 
\end{center} 
 
\subsection{Alles-oder-nichts-Prinzip} 
Wird der Schwellenwert von $-50\,\text{mV}$ nicht erreicht, so entsteht kein Aktionspotential. Es folgt das \emph{Alles-oder-nichts-Prinzip}.
 
\end{document}