\documentclass{article}
\usepackage{mhchem}
\usepackage{amsmath}
\usepackage[a4paper]{geometry}
\usepackage{fancyhdr}
\pagestyle{fancy}
\lhead{Das Aktionspotential}
\rhead{Dezember 2025}
\begin{document}
\section{Das Aktionspotential}
Erreicht ein Reiz die Dendriten, so führt dieser zu einer elektrischen Erregung, dem \emph{Rezeptorpotential}. Die Zelle wird leicht depolarisiert. Dieses nimmt mit der Distanz leicht ab, bis es am Axonhügel ankommt, wo ein \emph{Aktionspotential} entstehen kann.
 
\subsection{Ablauf}
Unter dem Strich werden einfach nur, wenn der Schwellenwert erreicht wird, \ce{Na+}-Ionenkanäle geöffnet und geschlossen, wodurch die Zelle depolarisiert wird bevor \ce{K+}-Ionenkanäle geöffnet und geschlossen, wodurch die Zelle re- und hyperpolarisiert wird.
 
\begin{description}
 \item[Vordepolarisation] Aufgrund des Rezeptorpotentials kommt es zur leichten Depolarisation am Axonhügel, heißt die Polarisation sinkt leicht.
 \item[Depolarisation] Wird ein Schwellenwert von $-50\,\text{mV}$ erreicht, so öffnen sich spannungsgesteuerte \ce{Na+}-Ionenkanäle, das \ce{Na+} strömt in den Zellinnenraum. Die Spannung steigt auf $+35\,\text{mV}$.
 \item[Repolarisation] Die \ce{Na+}-Ionenkanäle schließen sich wieder, aufgrund der hohen Spannung öffnen sich \ce{K+}-Ionenkanäle, das \ce{K+} strömt aus der Zelle raus. Die Spannung wird erneut negativ.
 \item[Hyperpolarisation] Das Potential erreicht einen Wert negativer als das Ruhepotential.
 \item[Ruhepotential] Alle spannungsgesteuerten Ionenkanäle schließen sich wieder, das Ruhepotential kann sich erneut bilden. 
\end{description} 
% TODO color coded graph 
 
\subsection{Alles-oder-nichts-Prinzip} 
Wird der Schwellenwert von $-50\,\text{mV}$ nicht erreicht, so entsteht kein Aktionspotential. Es folgt das \emph{Alles-oder-nichts-Prinzip}.
 
\end{document}