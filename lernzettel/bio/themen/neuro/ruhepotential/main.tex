\documentclass{article}
\usepackage{hyperref}
\usepackage{amsmath}
\usepackage{csquotes}
\usepackage{mhchem} 
\usepackage[a4paper]{geometry}
\usepackage{fancyhdr}
\pagestyle{fancy}
\lhead{Das Ruhepotential}
\rhead{Dezember 2025}
\begin{document}
\section{Das Ruhepotential}
\textquote{Ruht} eine Zelle, leitet also momentan keinen Reiz weiter, so besteht ein \emph{Ruhepotential} von ungefähr $-70\,\text{mV}$. Der Innenraum der Zelle ist leicht negativer geladen als die Umgebung,
 
\subsection{Die Entstehung} 
Wird sich eine Nervenzelle angeguckt, dessen Ladung innerhalb der Zelle der Ladung außerhalb der Zelle identisch ist, so hat diese kein elektrisches Potential.
 
Weil diese Nervenzelle eine semipermeable Membran hat, eine Membran, die nur manche Teilchen durchlässt, in diesem Fall an erster Stelle Kalium \ce{K+} und teilweise Chlorid \ce{Cl-}, können diese die Zelle nun jeweils verlassen und betreten. Darüberhinaus werden die dazu gedrängt dies zu tun um vorherige Konzentrationsunterschiede auszugleichen.
 
Diffundiert \ce{K+} aus dem Innenraum der Zelle nach außen, so wird der Innenraum weniger positiv, also negativ und der Außenraum positiver. Betritt das \ce{Cl-} die Zelle passiert gleiches.
 
Proportional zum entstandenen Potential werden die relevanten Teilchen zur anderen Seite gezogen um dieses auszugleichen. Die beiden Kräfte gleichen sich aus, so dass es zu einem Gleichgewicht bei einem Ruhepotential von $-70\,\text{mV}$ kommt. 
 
\subsection{Die Natrium-Kalium-Pumpe} 
Natrium \ce{Na+} ist leicht permeable. Das \ce{Na+} diffundiert sehr langsam in den Innenraum der Zelle, so dass dieser weniger negativ wird und das Ruhepotential verschwindet.
 
Dagegen wirkt die \emph{Natrium-Kalium-Pumpe}. Mit \hyperref[APT]{ATP} als Energie kann das \ce{Na+} aus der Zelle raus, dafür \ce{K+} in die Zelle rein gepumpt werden. Während das \ce{Na+} aus der Zelle rausgehalten wird, kann das \ce{K+} sowieso wieder durch die Membran diffundieren. Das Ruhepotential bleibt erhalten.
 
\end{document}