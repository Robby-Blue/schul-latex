\documentclass{article}
\usepackage[a4paper]{geometry}
\usepackage{fancyhdr}
\pagestyle{fancy}
\lhead{Nervensystem}
\rhead{Dezember 2025}
\begin{document}
\section{Das Nervensystem}
Das Nervensystem kann wird auf zwei haupsächlichen Art- und Weisen unterteilt.
 
\begin{description}
 \item[zentral] Das zentrale Nervensystem (ZNS) besteht aus dem Gehirn und Rückenmark. Hier werden alle Informationen verarbeitet.
 \item[periphär] Daws periphere Nervensystem (PNS) stellt den Rest des Nervensystems dar. Es wird genutzt, um Informationen an das ZNS weiterzuleiten.
\end{description} 
 
Und funktional:
\begin{description}
 \item[somatisch] Das somatische (willkürliche) Nervensystem steuert bewusst die Muskelbewegungen im Körper.
 \item[vegetativ] Das vegetative Nervensystem steuert unterbewusste Prozesse, wie die automatische Atmung.
\end{description} 
 
\subsection{Reflexe} 
Ein Reiz wird durch einen \emph{Rezeptor} (\emph{Sensor}) aufgenommen. Durch \emph{afferente} (\emph{sensorische}) Nervenzellen wird dieser zur Verarbeitung in das Rückenmark weitergeleitet. Nachdem die dortigen \hyperref[Synapsen]{Synapsen} die Information verarbeitet haben, wird das Ergebnis durch \emph{efferene} (\emph{motorische}) Nervenzellen zurückgeleitet, zu einem \emph{Effektor}. Hier kommt es zu einer \emph{Reaktion}.
\end{document}