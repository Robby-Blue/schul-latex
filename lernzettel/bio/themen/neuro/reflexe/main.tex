\documentclass{article}
\usepackage{hyperref}
\usepackage[a4paper]{geometry}
\usepackage{fancyhdr}
\pagestyle{fancy}
\lhead{Reflexe}
\rhead{Dezember 2025}
\begin{document}
\section{Reflexe}
Reflexe sind automatische Reaktionen auf bestimmte Reize.
 
\subsection{Funktionsweise}
Kommt es an einem \hyperref[Nervenzellen]{Rezeptor} zu einem Reiz, so kommt es zu einem \hyperref[Das Rezeptorpotential]{Rezeptorpotential}. Ist dieses groß genug, kommt es am Axonhügel zum \hyperref[Das Aktionspotential]{Aktionspotential}. Dieses wird mithilfe von der \hyperref[Die Erregungsweiterleitung]{Erregungsweiterleitung} bis zur \hyperref[Synapsen]{Synapse} weitergeleitet, wo es in der folgenden Nervenzelle ein EPSP erzeugt. Dieses wird genauso weitergeleitet.
 
So wird die Information des Reizes von afferenten Nervenzeillen in das Rückenmark geleitet und von efferenten Nervenzellen zurück zum Effektor, z.\,B. einem Muskel, geleitet, damit es dort zu einer Reaktion kommen kann.
 
Manche Reflexe, wie der Kniesehnenreflex, benötigen gerademal eine Synapse, im Rückenmark. 
 
\end{document}