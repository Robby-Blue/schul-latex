\documentclass{article}
\usepackage{csquotes}
\usepackage{mhchem}
\usepackage[a4paper]{geometry}
\usepackage{fancyhdr}
\pagestyle{fancy}
\lhead{Synapsen}
\rhead{Dezember 2025}
\begin{document}
\section{Synapsen}
Das Ende einer jeden Nervenzelle, genannt die \emph{präsynaptische Endigung} oder \emph{Endknöpfchen}, sind verdickt. Darauf folgt der \emph{synaptische Spalt}, danach die Folgezelle. Dies bildet eine Synapse.
 
\subsection{Aufbau}
Das Endknöpfchen beinhaltet viele \emph{Vesikel}, eine Art Bläschen voller \emph{Neurotransmitter}, also Botenstoffen. Die Art der Neurotransmitter ist je nach Synapse spezifisch. In der Membram des Endknöpfchens, der \emph{präsynaptischen} Membran, sind spannungsgesteuerte \ce{Ca2+}-Ionenkanäle aufzufinden, in der \emph{postsynaptischen} Membran dafür ligandengesteuerte Ionenkanäle und Enzyme, welche die Neurotransmitter spalten.
 
\subsection{Funktionsweise} 
Erreicht das Aktionspotential die Synapse, so öffnen sich die \ce{Ca2+}-Ionenkanäle und \ce{Ca2+}-Ionen gelangen in die präsynaptische Membran. Diese binden sich an die Vesikel, welche sich daraufhin zur Membran bewegen.
 
Der Vesikel \textquote{verschmelzen} mit der Membran, deren Inhalt, ACh wird in den synaptischen Spalt freigelassen. Diese binden an Rezeptoren der postsynaptischen Membran, so dass die postsynaptischen Ionenkanäle geöffnet werden.
 
Kationen strömen in die postsynapse, sie wird depolarisiert. Das Aktionspotential wurde übertragen; als \emph{EPSP} (\emph{exitatorisches postsynaptisches Potential}).
 
Cholinesterase spaltet das ACh zu A und Ch, welche einzeln wieder in die präsynaptische Membran gelangen, dort ACh bilden und die Vesikel erneut bilden. 
\end{document}