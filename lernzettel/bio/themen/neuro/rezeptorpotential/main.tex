\documentclass{article}
\usepackage{hyperref}
\usepackage{mhchem}
\usepackage[a4paper]{geometry}
\usepackage{fancyhdr}
\pagestyle{fancy}
\lhead{Das Rezeptorpotential}
\rhead{Dezember 2025}
\begin{document}
\section{Das Rezeptorpotential}
Misst eine Sinneszelle einen Reiz, so kommt es zum \emph{Rezeptorpotential}. Ist dieser stark genug kann ein \hyperref[Das Aktionspotential]{Aktionspotential} folgen.
 
\subsection{Funktionsweise} 
An den Dendriten des Sinneszellen sind \ce{Na+}-Ionenkanäle aufzufinden, welche basieren auf dem zu messenden Reiz, dem \emph{adäquaten} Reiz, gesteuert sind. Soll eine Sinneszelle beispielsweise berührungen Messen, so ist dies ihr adäquater Reiz. Die zuständige Sinneszelle könnte mechanisch gesteuerte Ionenkanäle haben, welche sich öffnen wenn die Sinneszelle durch eine Berührung gedehnt wurde.
 
Wurden die Ionenkanäle geöffnet, so strömt \ce{Na+} in den Zellinnenraum und depolarisiert diesen. Diese Depolarisation ist das Rezeptorpotential. Diese Prozess selbst, oder andere Prozesse, bei welchen ein adäquater Reiz durch Veränderung des Membranpotentials in ein elektrisches Potential umgewandeld wird, werden die \emph{Signaltransduktion} genannt.
\end{document}