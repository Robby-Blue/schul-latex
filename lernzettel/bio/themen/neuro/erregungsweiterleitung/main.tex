\documentclass{article}
\usepackage{csquotes}
\usepackage{amsmath}
\usepackage{mhchem}
\usepackage[a4paper]{geometry}
\usepackage{fancyhdr}
\pagestyle{fancy}
\lhead{Erregungsweiterleitung}
\rhead{Dezember 2025}
\begin{document}
 
\section{Erregungsweiterleitung}  
Nachdem das Aktionspotential entstanden ist, muss dieses über das Axon bis zu den Synapsen weitergeleitet werden.
 
\subsection{Ablauf} 
Wird der Schwellenwert von $-50\,\text{mV}$ an einem bestimmten Punkt im Axon erreicht so öffnen sich dort spannungsgesteuerte \ce{Na+}-Ionenkanäle, \ce{Na+}-Ionen strömen in das Axon. Weiter entlang des Axons liegen negative Ladungen, welche diese Ionen anziehen, so dass sie sich bis zu den nächsten Ionenkanälen bewegen. Dort verursachen die Ionen eine weitere Depolarisation, aufgrund welcher sich die Ionenkanäle öffnen und der Zyklus sich wiederholen kann.
 
Die Erregung kann aber nicht wieder rückwärts geleitet werden, weil die \ce{Na+}-Ionenkanäle, sich nach dem schließen für eine kurze Zeit nicht wieder öffnen können. Diese Zeit ist die \emph{Refraktärzeit}.
 
\subsection{Kontinuierliche und Saltatorische Erregungsleitung}
Es gibt zwei Arten der Erregungsleitung; die \emph{kontinuierliche} und die \emph{saltatorische} Erregungsleitung. Beide funktionieren basierend auf der gleichen, oben beschriebenen Art und Weise.
 
Der einzige Unterschied liegt in der Verteilung der Ionenkanälen und der Isolation des Axons.
\begin{description}
 \item[kontinuierliche Erregungsleitung] liegt vor, wenn kontinuierlich, ununterbrochen, Ionenkanäle vorliegen und bei allen ein neues Aktionspotential entstehen muss.
 \item[saltatorische Erregungsleitung] liegt vor, wenn am Axon zwischen den Stellen mit Ionenkanälen, den Ranvierschen Schürringen, auch Myelinscheiden vorliegen. An diesen liegen keine Ionenkanäle und minimierte Spannungen zwischen Ladungen innerhalb und außerhalb des Axons vor. Somit muss, beziehungsweise kann, an diesen Stellen kein Aktionspotential ausgelöst werden, und die elektrische Erregung \textquote{überspringt} diese Stellen sozusagen.
\end{description} 
Weil das Aufbauen des Aktionspotentials jedes mal wieder Zeit braucht und die benötigen Aktionspotentiale pro Distanz bei der saltatorischen Erregungsleitung geringer ist, ist diese um einiges schneller.
 
Darüberhinaus ist die Erregungsleitung schneller, je dicker das Axon ist, weil sich die Ionen bei einem größeren Innenraum schneller ausbreiten können. 
  
\end{document}