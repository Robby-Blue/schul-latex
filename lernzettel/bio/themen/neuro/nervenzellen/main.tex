\documentclass{article}
\usepackage{hyperref}
\usepackage[a4paper]{geometry}
\usepackage{fancyhdr}
\pagestyle{fancy}
\lhead{Nervenzellen}
\rhead{Dezemeber 2025}
\begin{document}
\section{Nervenzellen}
Nervenzellen leiten die eingehende Information an die nächste, folgende, Nervenzelle weiter. Sie bestehen aus drei hauptsächlichen Teilen:
\begin{description}
 \item[Soma] Das Soma beinhaltet den Zellkern u.\,ä.
 \item[Dendriten] Kurze, verzweigte, Verbindungen, welche aus dem Soma herausgehen. Sie nehmen die Erregungen der Rezeptoren oder anderer Nervenzellen auf.
 \item[Axon] Aus dem \emph{Axonhügel} des Somas geht das Axon, eine lange Verbindung, hervor. Über dieses werden elektrische Erregungen weitergeleitet. Am Ende des Axons sitzt die \emph{präsynaptische Endigung}.
 
Axone von Wirbeltieren sind von der \emph{Myelinscheide} beziehungsweise \emph{Markscheide}, bestehend aus \emph{Gliazellen}, umwickelt. Diese isolieren das Axon, welches für die \hyperref[Die Weiterleitung]{weiterleitung} relevant ist.
\end{description}
Zwischen der präsynaptischen Endigung der einen Zelle und des Dendriten der folgenden Zelle sitzt die \hyperref[Synapsen]{Synapse}. 
 
 
\end{document}