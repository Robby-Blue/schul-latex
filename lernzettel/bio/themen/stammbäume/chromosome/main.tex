\documentclass{article}
\usepackage[a4paper]{geometry}
\usepackage{fancyhdr}
\pagestyle{fancy}
\lhead{Chromosomensatz}
\rhead{November 2025}
\begin{document}
\section{Chromosomensatz} 
\subsection{Chromosome} 
Die Erbinformation liegt im Zellkern in langen Strängen vor, welche einfach abgelesen werden können. Dies ist das \emph{Chromatin}.
 
Zur Zellteilung verdichten diese sich zu \emph{Chromosomen}. Chromosome bestehen aus zwei gleichen \emph{Chromatiden}, zusammengehalten durch das \emph{Centromer}. Es wird zwischen \emph{Zwei-Chromatid-} und \emph{Ein-Chromatid-}Chromosomen unterschieden.
 
\subsection{Aufbau}
Menschen haben 46 Chromosome, bestehend aus $22$ \emph{Autosomen}, bestehend aus zwei \emph{homologen Chromosomen}. Dass diese \emph{homolog} sind heißt, dass ihre Allelen an den gleichen Positionen vorkommen. Diese können aber unterschiedliche Gen encodieren. Dazu kommen die \emph{Gonosomen}, XX oder XY.
 
Zellen, welche alle 46 Chromosome beinhalten sind \emph{diploid} oder $2n$. Ei- und Spermazellen haben nur das Chromosom von einem Elternteil; sie haben einen einfachen Chromosomensatz und sind somit \emph{haploid} oder $1n$. 
 
\subsection{Allelen}
In jedem Chromosom liegen Allelen vor. Diese codieren für as gleiche \emph{Genprodukt}, müssen dabei aber nicht identisch sein. Sind sie es doch, so sind die \emph{homozygoten}, ansonsten sind es \emph{heterozygoten}. Jede Allele liegt eigentlich zweifach vor, wird aber nur ein mal gezählt weil beide Chromatide identisch sind.
 
Es wird zwischen \emph{dominanten} und \emph{rezessiven} Merkmalen unterschieden; liegen zwei heterozygoten vor, dann ist die Allele, welche sich durchsetzt dominant. Die andere ist rezessiv.
 
\subsection{Phäno- und Genotypen}
Ein \emph{Phänotyp} beschreibt das ersichtliche Erscheinungsbild eines Organismus. Dieses hängt natürlich von der Erbinformation desselben ab, des \emph{Genotypen}.
\end{document}