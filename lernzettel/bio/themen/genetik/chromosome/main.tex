\documentclass{article}
\usepackage[a4paper]{geometry}
\usepackage{fancyhdr}
\pagestyle{fancy}
\lhead{Chromosomensatz}
\rhead{November 2025}
\begin{document}
\section{Chromosomensatz} 
\subsection{Chromosome} 
Die Erbinformation liegt im Zellkern in langen Strängen vor, welche einfach abgelesen werden können. Dies ist das \emph{Chromatin}.
 
Zur Zellteilung verdichten diese sich zu \emph{Chromosomen}. Chromosome bestehen aus zwei \emph{Chromatiden}, zusammengehalten durch das \emph{Centromer}. Es wird zwischen \emph{Zwei-Chromatid-} und \emph{Ein-Chromatid-}Chromosomen unterschieden.
 
\subsection{Aufbau}
Menschen haben 46 Chromosome, bestehend aus $22$ \emph{Autosomen}, bestehend aus zwei \emph{homologen Chromosomen}. Dazu kommen die \emph{Gonosomen}, XX oder XY.
 
Zellen, welche alle 46 Chromosome beinhalten sind \emph{diploid} oder $2n$. Ei- und Spermazellen haben nur das Chromosom von einem Elternteil; sie haben einen einfachen Chromosomensatz und sind somit \emph{haploid} oder $1n$. 
\end{document}