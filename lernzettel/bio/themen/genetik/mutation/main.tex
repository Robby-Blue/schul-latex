\documentclass{article}
\usepackage[a4paper]{geometry}
\usepackage{fancyhdr}
\pagestyle{fancy}
\lhead{Mutation}
\rhead{November 2025}
\begin{document}
\section{Mutationen} 
Bei einer Mutation kommt es zu einer kleinen Veränderung der DNA, welche mehr oder weniger relevante Auswirkungen haben kann.
Diese werden in ihren Ursachen und Folgen unterschieden.
\begin{description}
 \item[Substitution] Ein Nukleotid der Sequenz wird substituiert, heißt ausgetauscht. Wie groß dessen folgen sind, kommt darauf an, welche Sequenz wie ausgetauscht ist.
  \begin{description}
   \item[Missense] Eine Aminosäure verändert sich nach der Translation. 
   \item[Nonsense] Die Aminosäure wird zu einem Terminator, die Aminosäurensequenz endet frühzeitig. 
   \item[Still] Bei einer stillen Mutation kommt es zu einer Mutation ohne Auswirkungen. Eine Base wurde durch eine andere Base ersetzt, welche aber zur gleichen Aminosäure codiert. Nach der Translation ist das Protein unverändert.
  \end{description} 
 \item[Frameshift] Die Nukleotide werden verschoben (geshifted), weil eines hinzugekommen oder verschwunden ist. Alle darauffolgenden Basen verändern sich.
\end{description} 
\end{document}