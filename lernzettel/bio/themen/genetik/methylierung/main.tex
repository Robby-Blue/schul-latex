\documentclass{article}
\usepackage[a4paper]{geometry}
\usepackage{fancyhdr}
\pagestyle{fancy}
\lhead{Methylierung}
\rhead{November 2025}
\begin{document}
\section{Methylierung}
Durch die Methylierung der DNA können auch erworbene Eigenschaften, welche nicht in der DNA selbst gespeichert sind vererbt werden.
 
\subsection{Funktionsweise}
Hat ein Stück der DNA einen höheren Methylierungsgrad, so ist dieser mehr kondensiert. Dadurch kann die RNA-Polymerase schlechter binden und der Transkriptionsgrad sinkt. Der Methylierungsgrad kann, und somit natürlich auch der Transkriptionsgrad und die Ausprägung der dazugehörigen Proteine, vererbt werden. Es ist nicht bekannt, wie die Methylierung vererbt wird.
\end{document}