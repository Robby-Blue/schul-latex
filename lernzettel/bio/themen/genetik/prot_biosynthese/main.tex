\documentclass{article}
\usepackage[a4paper]{geometry}
\usepackage{fancyhdr}
\pagestyle{fancy}
\lhead{Die Proteinbiosynthese}
\rhead{November 2025}
\begin{document}
\section{Die Proteinbiosynthese} 
Bei der Proteinbiosynthese wird ein Protein synthetisiert. Dies Geschieht, indem eine DNA zu einer prä-mRNA transkribiert (mit eventuellem alternativem Spleißen) wird bevor sie prozessiert wird. So kann diese sicher in das Zytoplasma gelangen, wo sie tatsächlich gebraucht wird. Dort wird diese translatiert, heißt, die Basen der Nukleotide werden in Aminosäuren encodiert. Die resultierende Aminosäurenkette ist das gewünschte Protein. 
% TODO Hyperrefs 
\end{document}