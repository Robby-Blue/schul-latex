\documentclass{article}
\usepackage{hyperref}
\usepackage[a4paper]{geometry}
\usepackage{fancyhdr}
\pagestyle{fancy}
\lhead{Die Proteinbiosynthese}
\rhead{November 2025}
\begin{document}
\section{Die Proteinbiosynthese} 
Bei der Proteinbiosynthese wird ein Protein synthetisiert. Dies geschieht, indem eine \hyperref[DNA]{DNA} zu einer prä-mRNA \hyperref[Die Transkription]{transkribiert} (mit eventuellem \hyperref[Das Alternative Spleißen]{alternativem Spleißen}) wird bevor sie prozessiert wird. So kann diese sicher in das Zytoplasma gelangen, wo sie tatsächlich gebraucht wird. Dort wird diese \hyperref[Die Translation]{translatiert}, heißt, die Basen der Nukleotide werden in Aminosäuren encodiert. Die resultierende Aminosäurenkette ist das gewünschte \hyperref[Proteine]{Protein}.
\end{document}