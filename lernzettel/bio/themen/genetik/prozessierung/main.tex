\documentclass{article}
\usepackage{hyperref}
\usepackage{csquotes}
\usepackage[a4paper]{geometry}
\usepackage{fancyhdr}
\pagestyle{fancy}
\lhead{Die Prozessierung der m-RNA}
\rhead{November 2025}
\begin{document}
\section{Die Prozessierung der m-RNA}
Die m-RNA muss bevor sie genutzt werden kann weiter prozessiert werden, um sie von einer prä-RNA (einem unreifen Primärtranskripts) zu einer fertigen m-RNA wird. 
 
\subsection{Introns und Exons}
% TODO 
 
\subsection{Ablauf}
\begin{enumerate}
 \item Zuerst läuft das \emph{capping} und das \emph{tailing} gleichzeitig ab. Dabei werden an beiden Enden Mengen an Nukleotide angehängt. Dies schützt die m-RNA davor, an den Enden abgebaut zu werden; stattdessen werden diese hinzugefügten Nukleotide abgebaut.
 \item Daraufhin kommt es zum Spleißen. Dabei werden die Introns der m-RNA abgetrennt und die Exons miteinander verbunden.
 \item Die nun \textquote{reife} m-RNA verlässt den Zellkern.
 \item Die Exons können \hyperref[Die Translation]{translatiert} werden.
\end{enumerate} 
\end{document}