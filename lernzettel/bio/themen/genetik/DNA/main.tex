\documentclass{article}
\usepackage[a4paper]{geometry}
\usepackage{fancyhdr}
\pagestyle{fancy}
\lhead{DNA}
\rhead{September 2025}
\begin{document}
\section{DNA} 
Jedes Leben beinhaltet in den Nukleinsäuren der Zellen bestimmte \emph{Erbinformationen}, welche für die Reproduktion verwendet werden, die \emph{DNA}, oder vollständig \emph{Desoxyribonukleinsäure}.
 
\subsection{Nukleotide}
Die DNA besteht aus mehreren aneinander gereihten \emph{Nukleotiden}, jeweils bestehend aus dem Zucker Desoxyribose, etwas Phosphat und eine Base.
% TODO, hier tikz, welches zwei Nukleotide zeigt, mit der C kette gekennzeichnet evtl, und den anderen bestandteilen
 
\subsection{Aufbau}
Mehrer aneinandergereihte Nukleotide bilden einen Strang. Zu jedem Strang einer DNA gehört dabei ein komplementärer Strang, welcher dem Schlüssel-Schloss-Prinzip nach zum ersten gehört. Die Schlösser und die Schlüssel sind dabei die Basen der Nukleotide: von den vier Basen, $G$ Guanin, $C$ Cytosin, $T$ Thymin und $A$ Adenin gehören immer zwei zusammen. Guanin und Cytosin haben jeweils drei Wasserstoffbrückenbindungen, wobei Thymin und Adenin mit gerademal zwei Bindungen zurecht kommen müssen.
 
Somit gehört zu jedem Nukleotide mit Guaning als Base ein komplementäres Nukleotide der jeweils anderen Strangs, welches Cytosin als Base hat. Gleiches gilt für Thymin und Adenin. 
 
Somit kann für eine Basensequenz, einer Liste der Erstbuchstaben aller Basen eines Strangs, die Basensequenz des komplementären Strangs gebildet werden, indem alle $G$s und $C$s und alle $T$s und $A$s vertauscht werden. 
 
\end{document}