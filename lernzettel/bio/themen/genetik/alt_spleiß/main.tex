\documentclass{article}
\usepackage[a4paper]{geometry}
\usepackage{fancyhdr}
\pagestyle{fancy}
\lhead{Das Alternative Spleißen}
\rhead{November 2025}
\begin{document}
\section{Das Alternative Spleißen}
Obwohl der Körper nur $23,000$ Gene hat, kann der Körper ungefähr $300,000$ Proteine herstellen. Dies funktioniert mithilfe des alternativen Spleißens.
 
\subsection{Funktionsweise}
Beim alternativen Spleißen können aus einer prä-mRNA unterschiedliche reife mRNA gebildet werden. Dies geschieht, indem während der Prozessierung die Exons in der mRNA unterschiedlich angeordnet werden oder manche komplett ausgelassen werden. Aus unterschiedlichen mRNA-Strängen folgen offensichtlich auch unterschiedliche Proteine.
 
\end{document}