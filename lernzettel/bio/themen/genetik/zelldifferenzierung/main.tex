\documentclass{article}
\usepackage[a4paper]{geometry}
\usepackage{fancyhdr}
\pagestyle{fancy}
\lhead{Zelldifferenzierung}
\rhead{November 2025}
\begin{document}
\section{Zelldifferenzierung} 
Nicht alle Zellen benötigen alle Erbinformationen; deshalb können durch die \emph{Zelldifferenzierung} aus einer Stammzelle unterschiedliche Zellen differenziert werden.
 
\subsection{Transkriptionsfaktoren}
\emph{Transkriptionsfaktoren} sind Proteine, welche als ein Faktor der Transkriptionsgeschwindigkeit arbeiten. Es wird weiter zwischen \emph{allgemeinen} und \emph{spezifischen} Transkriptionsfaktoren unterschieden.
\begin{description}
 \item[allgemeine Transkriptionsfaktoren] Immer benötigte Transkriptionsfaktoren, welche bei jeder Transkription genutzt werden. Sie binden am Promotor um der RNA-Polymerase das binden zu erleichtern.
 \item[spezifische Transkriptionsfaktoren] Transkriptionsfaktoren, welche bestimmte Transkriptionen beschleunigt oder verlangsamt. Sie können an regulatorische DNA-Sequenzen, die  \emph{Enhancer-} und \emph{Silencer}-Sequenzen, binden um die Transkription zu beschleunigen oder zu verlangsamen.
\end{description}
Nun können Stoffe an Rezeptorproteinen binden (wie z.\,B. Östrogen zu einem Östrogen-Rezeptor-Komplex), um als spezifische Transkriptionsfaktoren zu agieren. 
 
So können bestimmte Teile der DNA, je nach dem ob sie von einer Zelle benötigt werden oder nicht, (de)aktiviert werden. 
\end{document}