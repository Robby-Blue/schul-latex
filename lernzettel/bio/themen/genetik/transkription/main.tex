\documentclass{article}
\usepackage[a4paper]{geometry}
\usepackage{fancyhdr}
\pagestyle{fancy}
\lhead{Die Transkription}
\rhead{November 2025}
\begin{document}
\section{Die Transkription}
Die \emph{Transkription} ist ein Prozess, bei welcher die Informationen der DNA auf einen prä-mRNA Strang transkribiert wird, damit diese weiter verwendet werden kann. Dies ist nötig, weil die Informationen der DNA zur Proteinsynthese im Cytoplasma gebraucht werden, aber nur im Zellkern vorhanden sind.
 
\subsection{Die mRNA}
Die prä-mRNA (\emph{messenger}-Ribonukleinsäure) ist mit wenigen Unterschieden der DNA gegenüber ihr sehr ähnlich aufgebaut. Anstelle von langen Doppelsträngen besteht sie aus kurzen Einzelsträngen, welche anstelle von $T$ (Thymin) ein $U$ (Uracyl) nutzen. So kann die m-RNA als messenger zwischen dem Zellkern und der Cytoplasma arbeiten.
\subsection{Ablauf}
Die Transkription wird in drei hauptsächliche Phasen unterteilt, die \emph{Initiiation}, die \emph{Elongation} und die \emph{Termination}.
\begin{description}
 \item[Initiiation] Um die Transkription du \emph{initiieren}, muss die RNA-Polymerase auf der DNA die Startsequenz, den \emph{Promotor}, finden. Nach dem Finden des Promotors wird dort der DNA-Doppelstrang lokal in zwei Einzelstränge aufgeteilt.
 \item[Elongation] Bei der Elongation verlängert die RNA-Polymerase den neuen, transkribierten, Strang indem der codogene Strang abgelesen wird. Dabei werden dem Schlüssel-Schloss-Prinzip nach die Basen der Nukleotide an die jeweiligen Gegenstücke gebunden. Der codogene Strang verläuft dabei in $3' \rightarrow 5'$ Richtung, so dass der neue Strang in Richtung $5' \rightarrow 3'$ erstellt wird.
 \item[Termination] Wird ein Terminator gefunden, ist die Transkription fertig. Die prä-mRNA wird freigegeben, die DNA schließt sich wieder.
\end{description} 
\end{document}