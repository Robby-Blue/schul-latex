\documentclass{article}
\usepackage[a4paper]{geometry}
\usepackage{fancyhdr}
\pagestyle{fancy}
\lhead{Replikation}
\rhead{Oktober 2025}
\begin{document}
\section{Replikation}
Um einen DNA-Doppelstrang zu replizieren wird die \emph{semikonservative} Replikation genutzt. Dabei wird ein DNA-Doppelstrang in die beiden Einzelstränge geteilt bevor zu beiden Einzelsträngen der jeweils passende gegenteilige Strang gebildet werden kann.
 
\subsection{Ablauf}
Das Enzym \emph{Helicase} teilt einen Doppelstrang in die zwei Einzelstränge. Dabei werden durch die \emph{DNA-Polymerase} passende Nukleotide an die einzelnen Stränge gebunden.
 
Dabei ist aber nur ein $5' \rightarrow 3'$ Richtung eine \emph{kon­ti­nu­ier­liche} Replikation möglich, weil neue Nukleotide nur am $3'$ des vorherigen Nukleotids angefügt werden können. Der Strang, welcher durch kontinuierliche Replikation entstanden ist, ist der \emph{Leitstrang}.
 
Auf der anderen Seite gibt es noch den \emph{Folgestrang}. Hier können nur wenige Nukleutide von der Helicase aus bis zu den bereits replizierten Nukleotiden repliziert werden. Somit folgen viele kleine Bruchstücke, bestehend aus wenigen untereinander gebundenen Nukleotiden, welche aber nicht mit miteinander gebunden sind, sie \emph{Okazaki-Fragmente}. Die \emph{Ligase} veknüpft diese dann. Das ist die \emph{diskon­ti­nu­ier­liche} Replikation.
\end{document}