\documentclass{article}
\usepackage[a4paper]{geometry}
\usepackage{fancyhdr}
\pagestyle{fancy}
\lhead{Die Translation}
\rhead{November 2025}
\begin{document}
\section{Die Translation}
Nachdem die Informationen der DNA in die m-RNA umschrieben wurden und in das Zellplasma gelangt sind, müssen sie in Proteine übersetzt werden. Dies geschieht durch die Translation, in einem Ribosom. Dabei werden immer drei aufeinanderfolgende Nukleotide, ein Codon, in eine Aminosäure übersetzt. Eine Kette dieser übersetzten Aminosäuren bildet dann das decodierte Protein.
 
\subsection{Aufbau von Ribosomen}
Ribonosme haben drei relevante Stellen, die E-, die P- und die A-Stelle. Diese stehen jeweils für \emph{exit}, \emph{processing} und \emph{arrival}. 
 
\subsection{t-RNA}
Eine t-RNA ist ein aufgebundener RNA-Strang, welcher oben eine Aminosäure bindet und unten  ein dazugehöriges Anticodon, welches an ein Codon der m-RNA bindet, hat. Daraus folgt, dass wenn eine t-RNA an ein Teil einer m-RNA bindet, dass dann die Aminosäure der t-RNA auch die decodierte Aminosäure des Codons der m-RNA ist. 
 
\subsection{Ablauf} 
Eine t-RNA sitzt in der P-Stelle, so dass eine weitere t-RNA an die A-Stelle andocken kann, wessen Basen zu dne Basen der m-RNA binden können. Die Aminosäuren der t-RNA können nun aneinander binden. Die t-RNA kann an der E-Stelle das Ribosom nun verlassen, das Aminosäure wird dabei aber hinterlassen. So bildet sich mit der Zeit eine Kette. Zuletzt muss sich das Ribosom nur um ein Codon weiterbewegen, so dass an der A-Stelle Platz frei wird, welches von einer neuen t-RNA genutzt wird, wie in Schritt eins. Somit beginnt der Zyklus von vorne.
\end{document}