\documentclass{article}
\usepackage[a4paper]{geometry}
\usepackage{fancyhdr}
\pagestyle{fancy}
\lhead{Proteine}
\rhead{Oktober 2025}
\begin{document}
\section{Proteine}
Die Eigenschaften eines Proteins hängen von dessen räumlicher Struktur ab.
\begin{enumerate}
 \item Einzelne Aminosäuren bilden die \emph{Primär}struktur. 
 \item Durch Bindungen dieser kommt es zu einer Faltung bilden sich die \emph{sekundär}- und \emph{Teriär}struktur. Die Sekundärstruktur besteht dabei aus
 \begin{description}
  \item[$\alpha$ Helix] Wie eine Spirale, wie ein Helix, aufgebaut.
  \item[$\beta$ Faltblatt] Sieht wie ein Blatt aus, welches an einer Seite entlang mehrfach gefalten (und entfalten wurde).
 \end{description} 
 \item Aus mehreren Proteinen kann bei zusammewirkung eine \emph{Quartär}sturktur entstehen. 
\end{enumerate}
\end{document}