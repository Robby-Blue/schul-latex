\documentclass{article}
\usepackage[a4paper]{geometry}
\usepackage{fancyhdr}
\pagestyle{fancy}
\lhead{Räuber-Beute-Beziehungen}
\rhead{Januar 2026}
\begin{document}
\section{Räuber-Beute-Beziehungen} 
Frisst die eine Tierart die andere, so liegt eine \emph{Räuber-Beute-Beziehung} vor.
 
Diese Beziehung wirkt sich innerhalb eines Gebietes auf die Population beider Tierarten auf. Je mehr Beute es gibt, desto mehr Fressen können die Räuber finden und sich somit besser vermehren. Je mehr Räuber es gibt, desto mehr Beute fressen sie, weshalb es davon weniger gibt.
 
\subsection{Nahrungsnetze}
Eine Räuber-Beute-Beziehung kann dargestellt werden, indem ein Pfeil zwischen den beiden beteiligten Tierarten gezogen wird. Dieser zeigt von der Beute auf den Räuber. 
 
Wird dies für alle Tierarten getan, welche hierfür an einem bestimmten Ort relevant sind, so ist dies ein \emph{Nahrungsnetz}.
\end{document}