\documentclass{article}
\usepackage[a4paper]{geometry}
\usepackage{fancyhdr}
\pagestyle{fancy}
\lhead{Ökosysteme}
\rhead{Januar 2026}
\begin{document}
\section{Ökosysteme} 
Ein \emph{Ökosystem} ist die Summe von einer \emph{Biozönose} und einem \emph{Biotop} innerhalb einer Region.
\begin{description}
 \item[dynamisch] Ökosysteme verändern durch \emph{Umweltfaktoren}, sowohl intern als auch extern, konstant.
 \item[komplex] Die einzelnen Teile eines Ökosystems wirken konstant auf komplexen Art und Weise aufeinander aus.
 \item[offen] Ökosysteme haben keine klar definierten, unpassierbaren, Grenzen. Lebewesen und deren Energie können sich frei zwischen Ökosystemen bewegen.
\end{description} 
 
\subsection{Umweltfaktoren}
Umweltfaktoren sind diejenigen Faktoren, welche auf ein Ökosystem wirken. Sie können sich sowohl auf die Biozönose als auch auf das Biotop beziehen und sowohl intern and auch extern sein. 
 
\subsection{Biozönose und Biotop}
Die \emph{Biozönose} beschreibt die Lebewesen eines Ökosystems. Dazu zählen auch Pflanzen. Die Umweltfaktoren der Biozönose sind \emph{biotisch}.
 
Das \emph{Biotop} beschreibt den unlebendigen Lebensraum in welcher die Biozönose lebt, z.\,B. den Boden, den Fluss und die Steine. Die Umweltfaktoren dieser sind \emph{abiotisch}.
 
\end{document}