\documentclass{article}
\usepackage[a4paper]{geometry}
\usepackage{fancyhdr}
\pagestyle{fancy}
\lhead{Faktoren der Populationsentwicklung}
\rhead{Januar 2026}
\begin{document}
\section{Faktoren der Populationsentwicklung} 
Es gibt eine Anzahl an Faktoren für die Entwicklung der Größe einer Population eines Tieres. Diese können darin einsortiert werden, ob sie von der Dichte der Population abhängig sind oder nicht.
 
Dies kann dazu führen, dass die Population konstant in der Größe schwingt oder auch nach einer Weile eine bestimmte Größe, in welcher ein Gleichgewicht liegt, findet.
\begin{description}
 \item[dichteabhängige] Faktoren sind, offensichtlicherweise, von der Dichte der Tierart abhängig. Dazu zählen z.\,B. die Nahrungs- und Wawssermenge, die Übertragung von Krankheiten oder Parasiten oder Konkurrenz und Räuber.
 \item[dichteunabhängige] Faktoren haben mit der Dichte nichts zu tun. Dazu zählen die Faktoren wie die Temperatur, Niederschlag, Umweltkatastrophen, die Bodenqualität, und so weiter.
\end{description} 
\end{document}