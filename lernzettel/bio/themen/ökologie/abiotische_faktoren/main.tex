\documentclass{article}
\usepackage[a4paper]{geometry}
\usepackage{fancyhdr}
\pagestyle{fancy}
\lhead{Der Einfluss von abiotischen Faktoren}
\rhead{Januar 2026}
\begin{document}
\section{Der Einfluss von abiotischen Faktoren}
Abiotische Faktoren beeinflusst unterschiedliche Aspekte der dort lebenden Tiere. Die \emph{Bergmann-sche Regel} und die \emph{Allensche Regel}, zum Beispiel, beeinflussen in Abhängigkeit von der Temperatur der Klimazone eines \emph{Thermoreguliere} ein Aspekt des Körpers.
 
\subsection{Thermokonformer und Thermoregulierer}
Tiere können in \emph{Thermokonformer} und \emph{Thermoregulierer} aufgeteilt werden.
\begin{description}
 \item[Thermokonformer] müssen selbst den thermischen Umständen konform sein. Sie können selbst keinen, oder nur wenig, Einfluss auf die Körpertemperatur nehmen.
 
Sie können nur in bestimmten Regionen leben; diejenigen Regionen, welche zu kalt oder zu warm sind führen jeweils zu einer Kälte- oder Wärmestarre oder auch zu einem Kälte- oder Hitzetod.
 
Dafür benötigen sie weniger Energie. 
 \item[Thermoregulierer] können ihre eigene Körpertemperatur regulieren, über Stoffwechselprozesse und andere Prozesse. 
 
Somit ist diese von der Umgebungstemperatur unabhängig.
 
Dies benötigt mehr Energie. 
\end{description} 
 
\subsection{Die Bergmannsche Regel}
Die \emph{Bergmannsche Regel} stellt eine Kausalität zwischen der Wärme der Klimazone eines Tieres und der Körpergröße desselben Tieres auf. Werden sich verwandte Tierarten angeguckt, so wird festgestellt, dass je kälter die Klimazone ist, desto größer das Tier dieser ist.
 
Dies kann mit einfacher Geometrie begründet werden; wird ein Tier um einen Faktor $x$ größer, so wächst die Körperoberfläche um einen Faktor von $x^2$, das Körpervolumen aber um einen Faktor von $x^3$. Der Teil des Körpers, welcher die Energie des Tiers speichert, steigt bei wachsender Größe des Tiers scheller als der Teil des Körpers, welcher diese Energie in Form von Wärme an die Umwelt abgibt. % TODO: graph davon
 
Größere Tiere können mehr Energie für einen längeren Zeitraum speichern als kleinere Tiere, welches in einer kalten Region überlebensnotwendig ist. 
 
\subsection{Die Allensche Regel} 
Die \emph{Allensche Regel} stellt eine Kausalität zwischen der Wärme der Klimazone eines Tieres und der Größe der Körperanhänge desselben Tieres auf. Die Körperanhänge einer Tierart, welche in kälteren Regionen lebt, ist sind der Regel kleiner.
 
Dies liegt daran, dass die Größe der Körperanhänge proportional zur Größe der Fäche, welche Energie in Form von Wärme an die Umwelt abgibt, ist.
 
In kälteren Regionen sollte möglichst wenig Energie abgegeben, werden, welches durch kleinere Körperanhänge ermöglicht wird. Sind diese jedoch größer, so kann es einem Tier helfen die eigene Körpertemperatur in warmen Regionen zu regulieren. 
 
\end{document}