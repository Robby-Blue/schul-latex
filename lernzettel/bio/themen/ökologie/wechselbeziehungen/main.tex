\documentclass{article}
\usepackage[a4paper]{geometry}
\usepackage{fancyhdr}
\pagestyle{fancy}
\lhead{Wechselbeziehungen}
\rhead{Januar 2026}
\begin{document}
\section{Wechselbeziehungen} 
Wechselbeziehungen zwischen Lebewesen, also Interaktionen welche beide Tiere beeinflussen, können in vier Arten aufgeteilt werden: \emph{Konkurrenz}, \emph{Räuber-Beute-Beziehungen}, \emph{Symbiose} und \emph{Parasitismus}.
 
Die Auswirkunge einer Wechselbeziehung kann ganz vereinfacht durch die Symbolschreibweise ausgedrückt werden. Dabei werden zwei Tierarten benannt bevor, getrennt durch einen Schrägstrich, jedem Tier ein Plus oder ein Minus zugeordnet wird, je nach dem ob es durch die Wechselbeziehung profitiert oder Nachteile hat. 
  
\subsection{Konkurrenz}
Zwei Tiere, welche auf gleichem Raum versuchen zu leben und ähnliche Anforderungen an diesen haben, stehen in \emph{Konkurrenz} zueinander.  
 
Konkurrenz kann in kann \emph{intraspezifische} Konkurrenz, welche von zwei gleich Tieren der gleichen Art handelt, und in \emph{interspezifische} Konkurrenz, bei welcher es um zwei unterschiedliche Tierarten geht, aufgeteilt werden.  
 
Interspezifische Konkurrenz entsteht an erster Stelle durch begrenzte Ressourcen, welche zur ökologischen Nische von zwei Tierarten gehören. 
\begin{description} 
 \item[Konkurenzausschlussprinzip] Das \emph{Konkurenzausschlussprinzip} merkt an, dass interspezifische Konkurrenz langzeitig gesehen nicht möglich ist. Die Konkurrenzstärkere Art vertreibt die andere aus dem Lebensraum.
 
 \item[Konkurrenzvermeidung] Die \emph{Konkurrenzvermeidung} besagt, dass unterschiedliche Tierarten doch auch dem gleichen Lebensraum koexistieren können, wenn sie sich dabei genug vermeiden. Beispielsweise können sich zwei Vogelarten einen Wald teilen, wenn die eine Art nur in der Kronenschicht nistet, die andere aber unterhalb dieser. 
\end{description} 
 
\subsection{Räuber-Beute-Beziehungen} 
Frisst die eine Tierart die andere, so liegt eine \emph{Räuber-Beute-Beziehung} vor.
 
Diese Beziehung wirkt sich innerhalb eines Gebietes auf die Population beider Tierarten auf. Je mehr Beute es gibt, desto mehr Fressen können die Räuber finden und sich somit besser vermehren. Je mehr Räuber es gibt, desto mehr Beute fressen sie, weshalb es davon weniger gibt.
 
\subsection{Nahrungsnetze}
Eine Räuber-Beute-Beziehung kann dargestellt werden, indem ein Pfeil zwischen den beiden beteiligten Tierarten gezogen wird. Dieser zeigt von der Beute auf den Räuber. 
 
Wird dies für alle Tierarten getan, welche hierfür an einem bestimmten Ort relevant sind, so ist dies ein \emph{Nahrungsnetz}.
 
\subsection{Symbiose}
Eine \emph{Symbiose} beschreibt ein Zusammenleben von zwei unterschiedlicher Tierarten, aus welchem ein Vorteil für beide entsteht. Dabei kann dazwischen unterschieden werden, wie diese Symbiose abläuft und wie wichtig diese für die beiden Lebewesen ist. 
 
Dabei gibt es die Kategorien 
\begin{description}
 \item[Ektosymbiose] Ein Lebewesen lebt auf der Oberfläche des anderen Lebewesens.
 \item[Endosymbiose] Das Lebewesen lebt im Körper des anderens.
\end{description}
und 
\begin{description}
 \item[fakultative Symbiose] Beide Partner profitieren voneinander, sie sind aber einander nicht Überlebensrelevant.
 \item[obligate Symbiose] Die Partenr können nicht ohne einaner überleben.
\end{description}
 
\subsection{Parasitismus}
Ein \emph{Parasitismus} ist eine Wechselbeziehung zwischen zwei Tierarten, bei welcher ein Parasit die Ressourcen des Wirts für sich nutzt.
 
Dabei lässt der Parasit in der Regel dem Wirten geradeso genug Ressourcen um damit dieser überleben kann, damit der Parasit auch weiter überleben kann.
 
Ein Parasit kann im Leben zeitweise auch ohne Wirte und in Zwischenwirten leben bevor dieser sich im Endwirt fortpflanzt.
 
Hier kann zwischen \emph{Endo}parasiten, welche in anderen Lebewesen unterkommen, und \emph{Ekto}parasiten, welche auf anderen Lebewesen leben unterschieden werden.
 
\end{document}