\documentclass{article}
\usepackage[a4paper]{geometry}
\usepackage{fancyhdr}
\pagestyle{fancy}
\lhead{Konkurrenz}
\rhead{Januar 2026}
\begin{document}
\section{Konkurrenz}
Zwei Tiere, welche auf gleichem Raum versuchen zu leben und ähnliche Anforderungen an diesen haben, stehen in \emph{Konkurrenz} zueinander.  
 
Konkurrenz kann in kann \emph{intraspezifische} Konkurrenz, welche von zwei gleich Tieren der gleichen Art handelt, und in \emph{interspezifische} Konkurrenz, bei welcher es um zwei unterschiedliche Tierarten geht, aufgeteilt werden.  
 
\subsection{Interspezifische Konkurrenz}
Interspezifische Konkurrenz entsteht an erster Stelle durch begrenzte Ressourcen, welche zur ökologischen Nische von zwei Tierarten gehören.
 
\subsection{Prinzipien} 
Das \emph{Konkurenzausschlussprinzip} merkt an, dass interspezifische Konkurrenz langzeitig gesehen nicht möglich ist. Die Konkurrenzstärkere Art vertreibt die andere aus dem Lebensraum.
 
Die \emph{Konkurrenzvermeidung} besagt, dass unterschiedliche Tierarten doch auch dem gleichen Lebensraum koexistieren können, wenn sie sich dabei genug vermeiden. Beispielsweise können sich zwei Vogelarten einen Wald teilen, wenn die eine Art nur in der Kronenschicht nistet, die andere aber unterhalb dieser. 
 
\end{document}