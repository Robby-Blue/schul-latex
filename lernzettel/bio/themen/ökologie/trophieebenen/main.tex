\documentclass{article}
\usepackage{mhchem}
\usepackage[a4paper]{geometry}
\usepackage{fancyhdr}
\pagestyle{fancy}
\lhead{Trophieebenen}
\rhead{Januar 2026}
\begin{document}
\section{Trophieebenen}
Die Lebewesen eines Ökosystems können weiter in \emph{Produzenten}, \emph{Konsumenten} und \emph{Destruenten} unterteilt werden.
\begin{description} 
 \item[Produzenten] sind diejenigen Lebewesen, welche die organischen Stoffe, die Organismen zum leben brauchen, produzieren. Dies sind in der Regel Pflanzen. 
 \item[Konsumenten] sind Lebewesen, welche die organischen Stoffe der Produzenten aufnehmen.
 
Diese können weiter in ihrer \emph{Trophieebene} unterschieden werden.
 \begin{description}
  \item[Primärkonsumenten] fressen Produzenten direkt, sie sind also Pflanzenfresser.
  \item[Sekundärkonsumenten] fressen Primärkonsumenten. Sie sind Fleischfresser, welche Pflanzenfresser fressen.
  \item[Tertiärkonsumenten] fressen Sekundärkonsumenten. 
 \end{description} 
 \item[Destruenten] bauen bereits abgestorbene Biomasse zu anorganischen Stoffen, sei es \ce{CO2} oder Wasser, ab.
\end{description}
Somit entsteht ein Kreislauf. Konsumenten fressen die Produkte der Produzenten, Destruenten basieren sich auf Konsumenten und die die Ressourcen der Produzenten werden von den Destruenten aufgebaut.  
\end{document}