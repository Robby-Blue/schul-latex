\documentclass{article}
\usepackage{hyperref}
\usepackage{mhchem}
\usepackage[a4paper]{geometry}
\usepackage{fancyhdr}
\pagestyle{fancy}
\lhead{Die Primärreaktion}
\rhead{Januar 2026}
\begin{document}
\section{Die Primärreaktion}
Bei der Primärreaktion reagiert an der Membran zwischen dem Stroma und eines Thylakoidinnenraum
\[
 \ce{H2O + NADP+ + ADP + P ->[Licht] O2 + NADPH + ATP} 
\] 
 
\subsection{Ablauf} 
Trifft Licht auf das \emph{Fotosystem II} (\emph{P680}), so wird dieses angeregt. Dadurch kann dieses leichter Elektronen an den primären \emph{Elektronenakzeptor} abgeben.
 
Chlorophyll katalysiert die \emph{Fotolyse}, eine Reaktion bei welcher unter Einfluss von Licht innerhalb des Thylakoidinnenraums
\[
 \ce{2H2O ->[Licht] O2 + 4H+ + 4e-} 
\]
Die freien Elektronen werden auf das Fotosystem II übertragen. Somit füllen sie die Lücken der Elektronen, welche zuvor auf den Elektronenakzeptor übertragen wurden. Die Protonen der Fotolyse bleiben im Thylakoidinnenraum.
 
Die Elektronen des Elektronenakzeptors fließen über die Plastochinon, den Cytochrom-Komp\-lex, die Plastocyanin zum \emph{Fotosystem I} (\emph{P700}), dies ist die \emph{Elektronentransportkette}. Die Energie davon wird genutzt um Protonen in den Thylakoidinnenraum zu transportieren, so dass ein Protonengradient entsteht.
 
Trifft Licht nunauf das Fotosystem I werden wieder Elektronen auf dessen Elektronenakzeptor übertragen. Dessen Lücke wird durch Elektronen aus der Elektronentransportkette gefüllt.
 
Von diesem Elektronenakzeptor werden Elektronen über das \emph{Ferredoxin} an \ce{NADP+}-Redukt\-ase abgegeben. Dieses übertrag die Elektronen auf \ce{NADP+}, so dass sie zu \ce{NADPH} reduziert wird. 
 
Strömen die in den Thylakoidinnenraum transportierten Protonen durch die ATP-Synthese, so wird ATP synthetisiert.
 
\subsection{Vergleich} 
Damit ist die Primärreaktion der \hyperref[Die Atmungskette]{Atmungskette} sehr Ähnlich, darin dass über den Elektronentransport ein Protonengradient aufgebaut werden kann, dessen chemiosmotische Energie genutzt werden kann um ATP zu synthetisieren. Die Unterschiede liegen in der Quelle der Energie (Licht und chemisch gespeichert) und den Edukten außerhalb des ATPs.  
\end{document}