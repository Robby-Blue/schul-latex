\documentclass{article}
\usepackage{mhchem}
\usepackage[a4paper]{geometry}
\usepackage{fancyhdr}
\pagestyle{fancy}
\lhead{Äußere Einflussfaktoren}
\rhead{Januar 2026}
\begin{document}
\section{Äußere Einflussfaktoren}
Versuche zeigen dass die Fotosynthese von drei hauptsächlichen äußeren Faktoren abhängig ist. 
 
\subsection{Versuche und Variablen}
Soll die Abhängigkeit eines Prozesses von einer bestimmten Variable überprüft werden, so kann ein Versuch durchgeführt werden, in welchem diese eine \emph{unabhängige Variable} ist. Die vermutete Abhängigkeit, hier die Fotosyntheserate, wird als \emph{abhängige Variable} gemessen.
 
Dabei muss sichergestellt werden, dass bei jeder Durchführung des Versuches alle \emph{kontrollierten Variablen}, heißt, alle anderen, nicht getesteten, Außenfaktoren, gleich bleiben.
 
\subsection{Ergebniss}
Wird sich der \emph{\ce{CO2}-Gehalt des Wassers}, die \emph{Temperatur} und die \emph{Lichtstärke} angeguckt, so wird festgestellt, dass die Fotosyntheserate bei der Erhöhung einer dieser Faktoren, bis zu einer bestimmtem Sättigung, auch steigt. 
\end{document}