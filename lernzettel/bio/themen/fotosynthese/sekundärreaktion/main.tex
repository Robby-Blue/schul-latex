\documentclass{article}
\usepackage{mhchem}
\usepackage[a4paper]{geometry}
\usepackage{fancyhdr}
\pagestyle{fancy}
\lhead{Die Sekundärreaktion}
\rhead{Januar 2026}
\begin{document}
\section{Die Sekundärreaktion}
Die Sekundärreaktion, der \emph{Calvin-Zyklus}, kann in drei Schritte unterteilt werden, welche sich konstant wiederholen.
\begin{description}
 \item[Fixierungsphase] Das \ce{CO2} der Umgebung wird fixiert indem es mithilfe des Enzyms \emph{RuBisCo} an ein Stoffwechselmolekül, \emph{Ribulose-1,5-biphosphat}, gebunden wird. Daraus folgen zwei Moleküle \emph{3-Phosphoglyzerinsäure}.
 \item[Reduktionsphase] Ein Molekül 3-Phosphoglyzerinsäure wird unter Zufuhr an Energie, gewonnen durch \ce{ATP -> ADP + P} und \ce{NADPH -> NADP}, zu \emph{Glyzerinaldehyd-3-phosphat} reduziert. Dabei entsteht ein \ce{H2O}.
 \item[Regenerationsphase] Der \ce{CO2}-Akzeptor Ribulose-1,5-biphosphat wird regeneriert. Dies können die übrigen Glyzerinaldehyd-3-phosphat Moleküle unter \ce{6 ATP -> 6 ADP + P} machen.
\end{description} 
\end{document}