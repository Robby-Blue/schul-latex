\documentclass{article}
\usepackage{mhchem}
\usepackage[a4paper]{geometry}
\usepackage{fancyhdr}
\pagestyle{fancy}
\lhead{Aufbau von bifalizen Laubblättern}
\rhead{Januar 2026}
\begin{document}
\section{Aufbau von bifalizen Laubblättern}
\begin{description} 
 \item[Epidermis] Bifaziale Laubblätter nutzen als Außenwand eine obere und eine untere \emph{Epidermis}. Diese, bestehend aus Lückenlosen Zellen, schützt und stabilisiert das Blatt und reguliert den Gasaustausch.
 \item[Kutikula] Die \emph{Kutikula} ist eine wachsartige Schicht auf der Epidermis. Sie verhindert den Eintritt von Stoffen, insbesondere von Wasser.
 \item[Palisadengewebe] Das \emph{Palisadengewebe}, bestehend aus einer oder mehreren Schichten an dichten, langen Zellen, fängt das Licht auf. Hier liegen auch die meisten der Chloroplasten. Es befindet sich zwischen der oberen Epidermis und den Schwammgewebe.
 \item[Chloroplasten] \emph{Chloroplasten} stellen den Ort der Fotosynthese dar.
 \item[Schwammgewebe] Zwischen dem Palisadengewebe und der unteren Epidermis liegt im Blatt das \emph{Schwamm-gewebe}. Hier sind neben unregelmäßrigen, kleineren Zellen größere Hohlräume aufzufinden, alle durch Spaltöffnungen miteinandner verbunden. Dieses System an Hohl-räumen, das \emph{Inter-zellular-system}, transportiert \ce{CO2} und Wasser.
 \item[Spaltöffnung] In der unteren Epidermis sitzt die \emph{Spaltöffnung} (beziehungsweise das \emph{Stomata}). Sie reguliert, in Abhängigkeit von ihrer Größe, die Abgabe von Wasser als auch den Austausch von \ce{CO2} und Sauerstoff.
 \item[Leitbündel] Die \emph{Leitbündel} eines Blattes sind makroskopische Adern, welche bis in das Palisaden- beziehungsweise Schwammgewebe gehen. Sie transportieren Wasser und Nährstoffe in das Blatt rein, transportieren Fotosyntheseprodukte aus dem Blatt raus.
\end{description}
 
\subsection{Sonnen- und Schattenblätter}
Blätter können in zwei Gruppen aufgeteilt werden; Sonnenblätter und Schattenblätter.
\begin{description}
 \item[Sonnenblätter] sind dunkelgrün, dicker, kleiner. Ihre Fotosyntheserate, in Abhängigkeit zur Lichtstärke, steigt nur langsam an, erreicht aber ein hohes Maximum. Sie haben ein mehrschichtiges Palisadengewebe, mehr Stomata und eine dickere Epidermis und Cuticula, welche für die erhöhte Fotosyntheserate benötigt werden
 \item[Schattenblätter] sind hellgrün, dünner, großer. Ihre Fotosyntheserate steigt aufgrund der Größe schnell an, erreicht aufgrund des einschichtigen Palisadengewebes aber auch schnell ein niedriges Plateu.
\end{description}
 
\end{document}