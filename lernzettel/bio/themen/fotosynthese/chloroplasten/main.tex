\documentclass{article}
\usepackage[a4paper]{geometry}
\usepackage{fancyhdr}
\pagestyle{fancy}
\lhead{Aufbau der Chloroplasten}
\rhead{Januar 2026}
\begin{document}
\section{Aufbau der Chloroplasten} 
Chloroplasten haben eine äußere und eine innere \emph{Membran}. Aus diesen beiden gehen lange, gerade \emph{Stroma-Thylakoide} hervor, so dass dessen innenraum auch zwischen den beiden Membranen ist.
 
An den Seiter der Stroma-Thylakoide sind \emph{Grana-Thylakoide} aufzufinden, welche vertikal übereinander ein \emph{Granum} darstellen.
 
Desweiteren beinhalten Chloroplasten noch Fetttröpfchen, Stärkekörner, DNA, Ribosome und das Stroma. 
 
\subsection{Kompartimentierung} 
Chloroplasten sind somit in unterschiedliche Kompartimente unterteilt, wie das Stroma und den Thylakloidinnenraum. 
 
\subsection{Chlorophyll}
Chlorophyll \textit{a} und Chlorophyll \textit{b} sind für die Fotosynthese relevante Pigmente. Dies ist daraus zu schlussfolgern, dass das Blatt grün ist. Darauf folgt dass für die Fotosynthese absorbiert (genutzt) blaues und rotes Licht genutzt werden muss, wobei Cholorophyll genau dies tut.
 
Die Chlorophyll-Moleküle liegen in den Thylakoid-Membranen. Dort liegen sie in \emph{Proteinkomplexen} und bilden \emph{Lichtsammelkomplexe}. 
\end{document}