\documentclass{article}
\usepackage[a4paper]{geometry}
\usepackage{fancyhdr}
\pagestyle{fancy}
\lhead{Modellanalysen} 
\rhead{Januar 2026} 
\begin{document}  
\section{Modellanalysen}
Ein Modell hat die Aufgabe eine bestimmte Struktur oder Aufgabe der Realität vereinfacht darzustellen, so dass die Funktionsweise großteils erhalten bleibt, das Verständnis vereinfacht werden kann und neue Erkenntnisse über die Realität gewonnen werden können.
 
Es gibt drei Arten von Modellen
\begin{description}
 \item[Strukturmodelle] stellen den Aufbau einer biologischen Struktur dar.
 \item[Funktionsmodelle] stellen Funktionen oder Prozesse dar.
 \item[Theoretische Modelle] erklären komplexe Vorgänge auf eine einfache Art und Weise.
\end{description} 
 
\subsection{Analyse}
Zum verständniss eines Modells muss das biologische Analog jedes Teils, sei es eine Struktur oder ein Prozess, des Modells bestimmt werden.
 
Nachdem die Analoge bestimmt wurden, kann festgestellt werden, ob manche Analoge im Vergleich zum Original anders funktionieren oder komplett ausgelassen wurden.
 
\end{document}