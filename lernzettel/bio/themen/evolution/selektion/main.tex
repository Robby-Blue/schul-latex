\documentclass{article}
\usepackage[a4paper]{geometry}
\usepackage{fancyhdr}
\pagestyle{fancy}
\lhead{Natürliche Selektion}
\rhead{Februar 2026}
\begin{document}
\section{Natürliche Selektion}
Die natürliche Selektion ist ein wichtiger Evolutionsfaktor. Durch diese nimmt die Genfrequenz der Gen, die letztendlich für einen angepassteren Phänotypen codieren, zu. 
 
\subsection{Funktionsweise} 
Diese besagt dass Lebewesen eine \emph{reproduktive Fitness} haben, die angibt, wie gut ein Lebewesen, basierend auf dessen Genotyp, sich fortpflanzen kann. Ist ein Lebewesen besser an die eigene Umwelt angepasst, so hat es eine höhere reproduktive Fitness.
 
Diese Lebewesen können sich somit besser, mehr vermehren und ihre Genotype an den Nachwuchs weitergeben. Die Genotype der schlechter angepassten Lebewesen hingegen werden weniger häufig weitergegeben. 
 
\subsection{Arten}
Wird sich die Auswirklung der Selektion auf die Verteilung der Ausprägung eines bestimmten Merkmales angeguckt, so können diese in drei Arten aufgeteilt werden
\begin{description}
 \item[gerichtete Selektion] Die Ausprägung bewegt sich in eine spezifische Richtung. Auf der einen Seite nimmt sie ab, auf der anderen Seite nimmt sie zu.
 \item[stabilisierende Selektion] Die Ausprägung wird gezielter. Die Kanten nehmen ab.
 \item[aufspaltende Selektion] Die Ausprägung wird in zwei gespalten. Das vorherige Maximum sinkt.
\end{description}
Die meisten Fälle handeln von \emph{gerichteter} Selektion.
\end{document}