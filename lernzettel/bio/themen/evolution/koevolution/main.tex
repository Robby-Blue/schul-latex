\documentclass{article}
\usepackage[a4paper]{geometry}
\usepackage{fancyhdr}
\pagestyle{fancy}
\lhead{Koevolution}
\rhead{Januar 2026}
\begin{document}
\section{Koevolution}
\emph{Koevolution} beschreibt Prozesse der Evolution bei welcher zwei Arten aufeinander wechselwirken. So evolvieren sich beide Arten zusammen.
 
\subsection{Beispiel: Räuber und Beute}
Gibt es zwei Tierarten in einer Räuber-Beute-Beziehung, so haben diejenigen Räuber eine höhere reproduktive Fitness, die die Beute besser fangen können, also z.\,B. sich schneller fortbewegen können. Die Beute führt also auf die Räuber einen Selektionsdruck aus, schneller zu werden.
 
Andersherum haben aber auch diejenigen Beutetiere eine höhere reproduktive Fitness, welche gar nicht erste zur Beute werden, weil sie besser vor den Räubern fliehen können. Im gleichen Beispiel könnte dies auch heißen, dass auch sie sich schneller fortbewegen können. Die Räuber erzeugen also genauso einen Selektionsdruck auf die Beute aus, schneller zu werden.
 
\end{document}