\documentclass{article}
\usepackage{hyperref}
\usepackage[a4paper]{geometry}
\usepackage{fancyhdr}
\pagestyle{fancy}
\lhead{Synthetische Evolutionstheorie}
\rhead{Januar 2026}
\begin{document}
\section{Synthetische Evolutionstheorie}
Die \emph{Synthetische Evolutionstheorie} ist die zurzeit gültige Evolutionstheorie, welche die vorherigen Erkentnisse über die Evolution mit dem Wissen der Zellbiologie und der Genetik zusammensetzt.
 
\subsection{Begriffe}
\begin{description} 
 \item[Population] Eine Gruppe an artgleicher Indiviuen.
 \item[Art] Dem \emph{populationsgenetischem Artbegriff} nach eine Gemeinschaft an Lebewesen, die reproduktiv von anderen isoliert sind. Das heißt Lebewesen einer Art können untereinander aber nicht mit Tieren anderer Arten fruchtbare Nachkommen bekommen können.
 \item[reproduktive Fitness] Der Fortpflanzungerfolg einer eines Lebewesens basierend auf dessen Merkmalsausprägungen.
 \item[Genpool] Die Gesamtheit aller Genvarianten innerhalb von einer Population.
 \item[Genfrequenz] Die Häufigkeit von verschiedenen Genvarianten.
 \item[Evolutionsfaktoren] Faktoren, welche die Genfrequenz verändern.
 \item[Rekombination] Neukombination der Gene der Eltern bei der Bildung von Kindern.
 \item[Mutation] Der einzige Weg, neue Mermale in Populationen einzubringen. Siehe das Kapitel \hyperref[Mutationen]{Mutationen}. 
\end{description} 
\end{document}