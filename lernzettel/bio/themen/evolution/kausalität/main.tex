\documentclass{article}
\usepackage[a4paper]{geometry}
\usepackage{fancyhdr}
\pagestyle{fancy}
\lhead{Kausalität}
\rhead{Januar 2026}
\begin{document}
\section{Kausalität}
Die Kausalität beschreibt die Beziehung zwischen einer Ursache und einer daraus folgenden Wirkung um eine bekannte Wirkung zu erklären. Sowohl die Ursache als auch die Wirkung müssen dabei beide echte Zustände oder Ereignisse sein.
 
Dabei wird zwischen drei hauptsächlichen Arten an Erklärungen unterschieden.
\begin{description} 
 \item[finale Erklärungen] geht davon aus, dass das relevante Lebewesen ein Bewusstsein hat und auf eine Art und Weise handelt um ein bestimmtes Ziel zu erreichen. Dies ist eigentlich zu meiden.
 \item[proximate Erklärungen] beziehen sich auf Prozesse welche aufgrund von auf Gene, Hormone, Proteine, Botenstoffe, unmittelbar im Körper des Lebewesen passieren.
 \item[ultimate Erklärungen] beziehen sich auf Veränderungen, welche auf die Evolution zurückzuführen sind.
\end{description} 
\end{document}