\documentclass{article}
\usepackage[a4paper]{geometry}
\usepackage{fancyhdr}
\pagestyle{fancy}
\lhead{Artbildung}
\rhead{Februar 2026}
\begin{document}
\section{Artbildung}
Liegt eine Population einer Art vor so kann diese auf zwei unterschiedliche Art und Weisen in zwei reproduktiv isolierte Arten aufgeteilt werden.
 
\subsection{Allopatrische Artbildung}
Die \emph{allopatrische} Artbildung basiert auf räumlicher Distanz. Eine Art wird durch eine räumlich Barriere aufgeteilt, so dass es keinen Genfluss mehr gibt. Es kommt zur \emph{Seperation}.
 
Kommt es nun bei einer der Arten zu einer Mutation, so kann diese nicht zur anderen Population überfließen. In der Population, in der diese bereits ist, kann sie eine höhere reproduktive Fitness bieten, so dass nach geraumer Zeit die gesamte Population diese Mutation übernommen hat.
 
\subsection{Sympatrische Artbildung}
Aufgrund einer Mutation kann es dazu kommen, dass bestimmte die Lebewesen einer Art bei der sexuelle Reproduktion bestimmte Phänotypen des anderen Lebewesens bevorzugen. Somit haben dessen Nachkommen alle auch die gleiche Genvererbte vorliebe beziehungsweise den gleichen Genvererbten Phänotypen.
 
Lebewesen, auch der nachfolgenden Generation, mit der vorliebe werden weiterhin Lebewesen mit dem Phänotypen vorziehen und die anderen ignorieren. 
  
\end{document}