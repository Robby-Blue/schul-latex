\documentclass{article}
\usepackage{hyperref}
\usepackage{mhchem}
\usepackage[a4paper]{geometry}
\usepackage{fancyhdr}
\pagestyle{fancy}
\lhead{Die Atmungskette}
\rhead{Januar 2026}
\begin{document}
\section{Die Atmungskette} 
Die Atmungskette beschreibt einen Prozess, der sich an der inneren Membran der Mitochondrien abspielt, um das \ce{H+}-Konzentrationsgefälle, welche von der \hyperref[Die ATP-Synthase]{ATP-Synthase} benötigt wurde, aufrecht zu erhalten.
 
Dabei reagiert
\[
 \ce{2 NADH + 2H+ + O2 -> 2NAD+ + 2H2O} 
\]
 
\subsection{Explosion der Knallgasreaktion} 
Dies stellt auch die Knallgasreaktion (\ce{2H2 + O2 -> 2H2O}) dar, nur mit Nebenprodukten. Hier kommt es aber, im gegensatz zur Knallgasreaktion, zu keiner Explosion, weil die Reaktion auf mehrere Schritte aufgeteilt ist und die Aktivierungsenergie, so dass sowie zu jedem Zeitpunkt die pro Zeiteinheit freigesetzte Energie geringer ist.
  
\subsection{Ablauf} 
Über das Komplex I und das Komplex II werden jeweils \ce{NADH + H+} und \ce{FADH2} aus der Matrix, als \hyperref[Elektronenshuttle]{Elektronenshuttle} funktionierend, an das Ubichinon in der Membran weitergegeben. Dabei oxidieren beide, jeweils zu \ce{NAD+} und \ce{FAD}; Ubichinon wird reduziert.
 
Ubichinon gibt die Elektronen über Komplex III an das Cytochrom c ab, Ubichinon oxidiert dabei, das Cytochrom c wird reduziert.
 
Die Elektronen werden vom Cytochrom c aus über Komplex IV auf Sauerstoff in der Matrix übertragen; das Cytochrom c oxidiert, der Sauerstoff reduziert. Zusammen mit zwei \ce{H+}-Ionen entsteht Wasser (\ce{H2O}).
 
Mit dieser Energie pumpem Komplexe I, III und IV \ce{H+}-Ionen in den Intermembranraum. 
 
\subsection{Kopplung}
Die Atmungskette stellt somit eine energetische Kopplung zwischen des Reaktionen, welche exergonisch, also freiwillig, passiert und der Aufrechterhaltung eines hohen \ce{H+}-Ionengradienten. 
\end{document}