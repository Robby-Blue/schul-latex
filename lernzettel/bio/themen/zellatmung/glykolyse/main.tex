\documentclass{article}
\usepackage{hyperref}
\usepackage{mhchem}
\usepackage[a4paper]{geometry}
\usepackage{fancyhdr}
\pagestyle{fancy}
\lhead{Die Glykolyse}
\rhead{Januar 2026}
\begin{document}
\section{Die Glykolyse}
Die Glykolyse ist ein wichtiger Prozess der der Zellatmung. Dabei reagiert im Zytoplasma der Zelle unteranderem ein Glycose Molekül in zwei \emph{Pyruvat}-Molekülen und \ce{2 NADH + H+}.
\[
 \ce{C6H12O6 + 2NAD + 2ADP + 2Phosphat -> 2 Pyruvat + 2 NADH + H+ + 2 ATP + 2H2O} 
\]
Das dabei entstehende \ce{Pyruvat} wird bei der \hyperref[Die oxidative Decarboxylierung]{oxidativen Decarboxylierung} benötigt. Das \ce{2 NADH + H+} ist für die \hyperref[Die oxidative Phosphorylierung]{oxidative Phosphorylierung} relevant.
\end{document}