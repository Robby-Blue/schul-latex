\documentclass{article}
\usepackage[a4paper]{geometry}
\usepackage{fancyhdr}
\pagestyle{fancy}
\lhead{Elektronenshuttle}
\rhead{Januar 2026}
\begin{document}
\section{Elektronenshuttle}
Ein Elektronenshuttle wird genutzt um Elektronen und Protonen auf eine effiziente Art und Weise im Körper zu transportieren. Sie funktionieren, indem ein Stoff in einer Redoxreaktionen Elektronen und Protonen aufnimmt, welche als Teil des Stoffes einfacher bewegt werden kann.
\subsection{Redoxreaktionen}
Redoxreaktionen sind chemische Reaktionen, bei welchen sowohl eine Elektronenabgabe (Oxidation) als auch eine Elektronenaufnahme (Reduktion) stattfindet.
Dies passiert oftmals zusammen mit Protonen.
 
\end{document}