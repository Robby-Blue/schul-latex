\documentclass{article}
\usepackage{hyperref}
\usepackage{mhchem}
\usepackage[a4paper]{geometry}
\usepackage{fancyhdr}
\pagestyle{fancy}
\lhead{Zellatmung}
\rhead{Januar 2026}
\begin{document} 
Die Zellatmung ein Prozess, um aus Glucose und Sauerstoff Energie zu gewinnen.
 
Insgesamt kann die Zellatmung durch eine chemische Reaktion beschrieben werden, welche auf viele kleine Teilschritte verteilt abläuft 
\[
 \ce{C6H12O6 + 6O2 -> 6CO2 + 6H2O} 
\]
Glucose ist dabei \ce{C6H12O6}.
 
\subsection{Ablauf}
Die Zellatmung ist in \hyperref[Die vier Teilschritte der Zellatmung]{vier Teilschritte} aufgeteilt, um in die Elektronen der Glucose auf \hyperref[Elektronenshuttle]{Elektronenshuttle} zu übertragen, welche in der \hyperref[Die Atmungskette]{Atmungsektte} genutzt werden kann, um ein \ce{H+}-Ionenkonzentrationsgradienten zwischen dem Intermembranraum und der Matrix aufrecht zu erhalten, welche die \hyperref[Die ATP-Synthase]{ATP-Synthase} nutzen kann um \hyperref[ATP]{ATP} zu synthetisieren.
\end{document}