\documentclass{article}
\usepackage{mhchem}
\usepackage[a4paper]{geometry}
\usepackage{fancyhdr}
\pagestyle{fancy}
\lhead{Die ATP-Synthase}
\rhead{Januar 2026}
\begin{document}
\section{Die ATP-Synthase}
Die ATP-Synthase ist ein Enzym, welches dafür verantwortlich ist, ATP zu produzieren (synthetisieren). Die ATP-Synthase liegt im Mitochondrium.
 
\subsection{Das Mitochondrium} 
Ein Mitochondrium hat eine \emph{äußere Membran}. Im inneren dieser ist der \emph{Intermembranraum}. Im diesem ist eine zweite, \emph{innere, Membran}. Die innere Membran hat viele einfaltungen, die \emph{Cristae}. Der Raum innerhalb der inneren Membran ist die Mitochondrien\emph{matrix}. In dieser liegt das Plasmid vor.
 
Sowohl im Intermembranraum als in der Matrix liegt \ce{H+} vor. Die Konzentration dessen ist im Intermembramraum größer.
 
An der inneren Membran liegen ATP-Synthasen vor. 
\subsection{Funktionsweise der ATP-Synthase} 
Aufgrund des Konzentrationsgradienten diffundiert \ce{H+}-Atome durch die ATP-Synthase, dabei wird genug Energie geliefert, so dass \ce{ADP + P -> ATP} reagieren kann; so kann die chemiosmotische Energie die des bestehenden \ce{H+}-Konzentrationsgradienten kann genutzt werden um ATP zu synthetisieren.
\end{document}