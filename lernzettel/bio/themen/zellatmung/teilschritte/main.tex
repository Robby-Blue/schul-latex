\documentclass{article}
\usepackage{hyperref}
\usepackage{mhchem}
\usepackage[a4paper]{geometry}
\usepackage{fancyhdr}
\pagestyle{fancy}
\lhead{Die vier Teilschritte}
\rhead{Januar 2026}
\begin{document}
\section{Die vier Teilschritte der Zellatmung}
Die Zellatmung kann in vier Teilschritte aufgeteilt werden, die \emph{Glycolye}, die \emph{oxidative Decarboxylierung}, den \emph{Citratzyklus} und die \emph{oxidative Phosphorylierung}.
 
Die ersten vier Teilscheritte Produzieren jeweils die Edukte des nachkommenden und des letzten Teilschrittes und auch in Teilen ATP.
 
\subsection{Die Glycolyse} 
Die Glykolyse ist ein wichtiger Prozess der der Zellatmung. Dabei reagiert im Zytoplasma der Zelle unteranderem ein Glycose Molekül in zwei \emph{Pyruvat}-Molekülen und \ce{2 NADH + H+}.
\[
 \ce{C6H12O6 + 2NAD + 2ADP + 2Phosphat -> 2 Pyruvat + 2 NADH + H+ + 2 ATP + 2H2O} 
\]
Somit entsteht dabei \ce{Pyruvat}, welches für die oxidative Decarboxylierung benötigt wird, \ce{2 NADH + H+}, welches für die oxidative Phosphorylierung relevant ist und ATP.
 
\subsection{Die oxidative Decarboxylierung}
Bei der oxidativen Decarboxylierung gelant das eben synthetisierte Pyruvat in die Mitochondrienmatrix, wo etwas Kohlenstoff abgespalten wird; es findet die de-carboxylierung statt. Dann wird das \ce{NAD+} des Pyruvats zu \ce{NADH + H+} reduziert. Zusammen mit dem Coenzym A entsteht Acetyl-CoA. Dieses ist für den Citratzyklus relevant.
 
\subsection{Der Citratzyklus}
Der Zitratzyklus ist eine zyklische Reaktionenkette, bei welcher aus dem Acetyl-CoA und anderen Edukten \ce{CO2}, \ce{ATP}, \ce{NADH + H+} und \ce{FADH2} gewonnen werden. Die letzteren werden bei der oxidativen Phosphorylierung benötigt.
 
\subsection{Die oxidative Phosphorylierung} 
Die oxidative Phosphorylierung ist die Zusammensetzung aus der \hyperref[Die Atmungskette]{Atmungskette} und der ATP-Synthese. Somit werden die Produkte der vorherigen Teilschritte genutzt um das \ce{H+}-konzentrationsgefälle aufgebaut und genutzt um ATP zu synthetisieren. 
\end{document}