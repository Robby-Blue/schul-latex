\documentclass{article}
\usepackage[a4paper]{geometry}
\usepackage{fancyhdr}
\pagestyle{fancy}
\lhead{Die Rückkopplung}
\rhead{Januar 2026}
\begin{document}
\section{Die Rückkopplung}
Die \emph{Rückkopplung} beschreibt Strukturen und Prozesse innerhalb der Zellatmung, welche eine Beziehung zwischen der Menge an vorliegendem ATP und der Menge an neu synthetisiertem ATP erstellen; wie viel neues ATP synthetisiert werden kann hängt von der bereits im Körper vor vorliegenden ATP Menge ab.
 
\subsection{Beispiel: Phosphofructokinase} 
Das Enzym Phosphofructokinase (PFK) spielt als Katalysator eine wichtige Rolle in der Zellatmung.
 
Wird die ATP-Konzentration erhöht, steigt vorerst auch die PFK-Aktivität; die Existenz von ATP erleichtert vorerst die produktion von neuem ATP, wird es benötigt. Bei höherer ATP-Konzentration, so wirk das ATP als Hemmstoff und de PFK-Aktivität sinkt erneut, so dass nicht zu viel ATP vorliegen kann. 
\end{document}