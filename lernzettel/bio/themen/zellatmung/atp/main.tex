\documentclass{article}
\usepackage{mhchem}
\usepackage[a4paper]{geometry}
\usepackage{fancyhdr}
\pagestyle{fancy}
\lhead{ATP}
\rhead{Januar 2026}
\begin{document}
\section{ATP}
\emph{ATP} (\emph{Adenosintriphosphat}, also Adenosin und drei Teile Phosphat) wird im Körper als Energiespeicher genutzt. Es wird im Prozess der Zellatmung produziert, wird im Körper verteilt und z.\,B. von Muskeln als Energiequelle genutzt. 
 
\subsection{Funktionsweise}
Findet eine exergonische Reaktion statt, wird z.\,B. Zucker abgebaut, so wird, per Definition, Energie freigelassen. Diese Energie kann genutzt werden damit ADP (Adenosindiphosphat, also Adenosin und zwei Teile Phosphat) zusammen mit einem weiteren Teil Phosphat zu ATP reagieren.
\[
 \ce{ADP + P -> ATP}
\]
Gegenteilig kann ATP selbst exergonisch Reagieren und Energie freilassen
\[
 \ce{ATP -> ADP + P} 
\]
Mit dieser freigelassenen Energie können andere Reaktionen, z.\,B. Muskelarbeit, angetrieben werden. 
\end{document}