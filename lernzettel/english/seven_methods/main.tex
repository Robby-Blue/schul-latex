\documentclass{article}
\usepackage[a4paper]{geometry}
\usepackage{fancyhdr} 
\usepackage{csquotes}
\pagestyle{fancy}
\lhead{seven methods of killing kylie jenner}
\rhead{September 2025}
\begin{document}
\section{seven methods of killing kylie jenner}
Jasmine Lee-Jones play and book \textit{seven methods of killing kylie jenner} explores the topics of race, identity and the usage of social media.
 
The book follows two main characters \emph{Cleo} and \emph{Kara}
\begin{description}
 \item[Cleo] Cleo is a straight black woman who keeps posting about killing Kylie Jenner on Twitter.
 \item[Kara] Kara is one of Cleo's friends, a mixed lesbian woman. She is against Cleo's actions on Twitter. 
\end{description}
 
\subsection{Method 1: Poison} 
After Forbes announces that Kylie Jenner became the younger \textquote{self-made} billionaire, Cleo posts argues in a Tweet that Kylie Jenner was just a white woman born into rich America and says that she wants to kill her instead, using poison. She compares the poison to Kylie Jenners lip filler, which is regarded as beautiful, which she constrasts to plump lips on Black women which are generally considered ugly. This is a reference to \emph{blackfishing}, the act of non-black people using makeup or similar to look more black. 
 
\subsection{Method 2: Shooting} 
Cleo wants Kylie Jenner to be shot, just not with a camera. After this tweet Kara approaches her in real life and they have a discussion, with Cleo arguing that she's just venting her frustration. 
 
\subsection{Method 3: Drowning}
Cleo wants Kylie Jenner to suffer the same fate as so many refugees do, drowning in the ocean. Furthermore in real life Cleo talks about her Black boyfriend cheating on her with a white girl who fetishises his Blackness. 
 
\subsection{Method 4: Skinning} 
For the fourth method Cleo posts about wanting Kylie Jenner to be skinned alive such that she can wear Kylie's culture as a costume, mimic her race and walk around in whiteface. This is reference to \emph{blackfacing}, when a white person uses racist stereotypes to mimic black people to make fun of them.
 
Afterwards Cleo accuses Kara of not understanding her because Kara is mixed, not \textquote{fully} Black. Cleo also brings up \#wiggate, an incident at a party where she was sexually harassed, mocked and humiliated by multiple people who made jokes about her lips and lastly ripped of her wig.
 
\subsection{Method 5: Immolation}
Cleo wants Kylie Jenner to be burnt to a crisp. At the same time the hate comments she gets keep becoming worse, turning into people callingh er the N-word, posting images of monkey and images of black women being lynched.
 
Later other Twitter users find an old tweet of Cleo being homophobic after Kara came out to her. 
 
\subsection{Method 6: Disgrace}
Unrelated to literal death, Cleo wants Kylie Jenner to be shamed and in reputational ruin: not a celebrity anymore.
 
A bit later Twitter users find out her real identity. This causes Cleo to come clean and to apologise for her homophobic tweet and concedes that she never actually planned to kill Kylie Jenner. She just wanted, she says, to bring awareness to the Black experience, which brings her to the last method
 
\subsection{Method 7: Displacement} 
In the last method Cleo brings up the idea that cultural appropriation can lead to the erasure and displacement of the culture which are being appropriated. She illustrates this using Saartjie, a South African woman who was exhibited as an attraction in Europe in the 19th century because of her physical features.
 
\subsection{The post-mortem} 
The post-morthem shows Cleo and Kara symbolically burrying the burdens while smoking a joint. Saartjie's spirit appears, guiding them in dealing with their decades of experience with racism, cultural appropriation and trauma. 
 
\end{document}