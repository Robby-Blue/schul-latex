\documentclass{article}
\usepackage{csquotes}
\usepackage{hyperref}
\usepackage[a4paper]{geometry}
\usepackage{fancyhdr}
\pagestyle{fancy}
\lhead{Task Types}
\rhead{September 2025}
\begin{document}
\section{Task Types}
 
\subsection{Comprehension}
The introduction, like always, starst with general information about the text before quickly stating what the text is about in one sentence. \newline 
Afterwards you need to present the relevant content in your own words. \newline 
No conclusion is necessary, at least in the Abi. 
 
Summaries are written in the simple present. 
 
\subsection{Analysis} 
The instroduction is just a general statement as to what one expects to find out in the analysis. No \textquote{in the following}. \newline 
The main part follows using the P-E-E method. \newline 
In the conclusion you summarize the most important points.
 
An analysis, unlike all other task types, needs quotes from the text at hand. 
 
\subsection{Argumentation}
An argumentations introduction starts by catching the readers attention by explaining why the issue at hand is controversial or otherwise relevant. Afterwards one might need to explain how the topics is relevant to the given text. The introduction ends by stating your own opinion. \newline 
The main part follows using the P-E-E method. A discussion should be rather unbiased (equal amount of arguments on both sides) while a comment is expected to be rather biased (more arguments on one side). \newline 
Lastly you draw a conclusion by weighing the pros and cons, stating your opinion in the end. 
 
\subsection{Mediation} 
A mediation task contains 3 important pieces of information in its task: the situation, the \hyperref[Text Type]{text type} and the required aspects.
 
The way the introduction and conclusion are written depends on the situation and the text type while the main bodys content depends entirely on the required aspects.
\end{document}