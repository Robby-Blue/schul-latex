\documentclass{article}
\usepackage[a4paper]{geometry}
\usepackage{fancyhdr}
\pagestyle{fancy}
\lhead{Theoretische Zugriffe}
\rhead{Oktober 2025}
\begin{document}
\section{Theoretische Zugriffe}
Mit der Zeit hat sich der Krieg, in der Frage wer Kämpft, wie gekämpft wird, für den gekämpft wird, etc., durchaus verändert, deshalb wird zwischen \emph{alten} und \emph{neuen} Kriegen unterschieden.
 
Insgesamt wurde der Krieg immer weiter privatisiert, dezentralisiert und somit auch \emph{asymmetrischer} (Staat gegen nicht-Staat). Weiter ausgeführt gilt 
\begin{center}
\begin{tabular}{ |l|c|c| }
\hline
  & \textbf{alt} & \textbf{neu} \\
\hline
 \textbf{Akteure} & Staaten & Private Akteure, Terroristen \\
\hline
 \textbf{Ziele} & Staatserweiterung & Macht, Ideologie \\
\hline
 \textbf{Intensität} & größer & geringer (keine Fronten) \\
\hline
 \textbf{Dauer} & spezifisch, kürzer & unspezifischer \\
\hline
 \textbf{Finanzierung} & Staatlich & Privat, z.\,B. durch Straaftaten \\
\hline
 \textbf{Kriegsrecht} & eingehalten & vermehrt ignoriert \\
\hline
 \textbf{Kriegsführung} & Fronten & Vermehrt Guerilla \\
\hline
 \textbf{Angriffsziele} & Militär, Infrakstruktur & Inklusive ZivilistInnen \\
\hline
\end{tabular}
\end{center}
 
\end{document}