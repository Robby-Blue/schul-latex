\documentclass{article}
\usepackage[a4paper]{geometry}
\usepackage{fancyhdr}
\pagestyle{fancy}
\lhead{Die UNO}
\rhead{Oktober 2025}
\begin{document}
\section{Die UNO}
Die \emph{Vereinten Nationen}, oder auf English die \emph{UNO} oder \emph{UN}, kurz für die \emph{United Nations (Organisation)}, ist eine Organisation, welche sich die Ziele gesetzt hat den Frieden zu bewahren und sich für soziale Gerechtigkeit und Menschenrechte einzusetzen.
 
Sie wurde nach dem zweiten Weltkrieg gegründet, vor allem aufgrund von den Alliierten. Heutzutage besteht sie aus 193 Ländern. 
 
Zu der UNO gehören noch weitere Organe wie die UNESCO, UNICEF, die WTO, die WHO, der IGH und noch weitere.
 
\subsection{Die Generalversammlung}
In der \emph{Generalversammlung} sind 193 Länder vertreten, jedes mit einer Stimme. Intern gilt die Generalversammlung als Legislative, extern kann sie höchstens Empfehlungen in Form von nicht bindenden Resolutionen aussprechen. In der Regel ist eine einfache Mehrheit ausreichend.
 
Mit einer $2/3$ Mehrheit kann die Generalversammlung darüberhinaus noch Mitglieder in den Sicherheitsrat wählen. 
 
\subsection{Der Sicherheitsrat}
Der \emph{Sicherheitsrat} kann \emph{bindende Resolution} aussprechen. Dafür muss es 9 von 15 Ja-Stimmen und keinem Veto von einem ständigen Mitglied kommen.
 
Der Sicherheitsrat besteht aus 15 Mitgliedern, aufgeteilt in 5 ständige Mitglieder, den \emph{Permanent Five, P5}, und den 10 nicht ständigen Mitgliedern. Jedes Jahr werden neue nicht ständingen Mitglieder auf 2 Jahre von der Generalversammlung gewählt.
 
Die P5 setzen sich aus den Alliierten, also die USA, Großbritannien, Frankreich, China und Russland, zusammen. Weil diese Länder politisch sehr unterschiedlich sind und Resolutionen mit einem einfachen Veto dieser verhindert werden können, kommt es zu nur sehr wenigen verabschiedeten Resolutionen.
 
\subsection{Schutzverantwortung}
Die \emph{Schutzverantwortung}, auch \emph{Responsibiltiy to Protect} oder \emph{R2P} genannt, ist die Verantwortung eines jeden Staates die eigene Bevölkerung vor Völkermord, Kriegsverbrechen, ethnischer Säuberung, usw. zu beschützen.
 
Ist dies nicht möglich, soll es internationale Hilfe, sei es durch Diplomatie, Entwicklungshilfe, etc., geben. Reicht auch diese nicht aus, hat die internationale Gemeinschaft das Recht einzugreifen, sei es mit militärischer Intervention. Dies ist aber nur durch den UN-Sicherheitsrat zu legitimieren.
\end{document}