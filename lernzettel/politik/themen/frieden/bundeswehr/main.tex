\documentclass{article}
\usepackage{csquotes}
\usepackage[a4paper]{geometry}
\usepackage{fancyhdr}
\pagestyle{fancy}
\lhead{Die Bundeswehr}
\rhead{Februar 2026}
\begin{document}
\section{Die Bundeswehr}
Die Bundeswehr ist die deutsche Armee.
 
\subsection{Parlamentsarmee} 
Die Bundeswehr ist eine sogenannte \emph{Parlamentsarmee}. Das heißt, dass die Bundeswehr vom deutschen Parlament kontrolliert wird. Dies ist ein Gegensatz zu zum Beispiel der Armee der Vereinigten Staaten von Amerika, welche direkt Befehle vom Präsidenten annimmt.
 
Wird die Bundesregierung um Beteiligung gebeten, legt das Bundeskanzleramt, das Auswärtige Amt und das Bundesministerium der Verteidigung dem Bundestag ein Mandatsentwurf vor. Dieses wird vom Verteidigungsausschuss und dm Auswärtigem Ausschuss beraten und Abgestimmt. Mit einer absoluten Mehrheit kann die Bundesregierung die Bundeswehr entsenden.
 
\subsection{Wandel} 
Bis ungefähr 2022 bestanden die Aufgaben der Bundeswehr an erster Stelle außerhalb von Deutschland. Sie sollte Beispielsweise gegen Terrorismus und Piraterie kämpfen. Seitdem müsse sie aber \textquote{kriegstüchtig} werden und sich auf einen Angriff Russlands vorbereiten, innerhalb von Europa sich verteidigen.
\end{document}