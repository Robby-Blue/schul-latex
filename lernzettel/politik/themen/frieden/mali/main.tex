\documentclass{article}
\usepackage[a4paper]{geometry}
\usepackage{fancyhdr}
\pagestyle{fancy}
\lhead{Krieg in Mali}
\rhead{Oktober 2025}
\begin{document}
\section{Krieg in Mali} 
Weil Kämpfer der Tuareg einen eigenen Staat, Azawad, am südlichen Sahararand, wollen, herrscht in Mali seit 2012 Krieg. 
\begin{enumerate}
 \item Die Tuareg-Kämpfer fingen 2012 den Krieg mit ihren Erfahrungen und Waffen aus dem Libyienkrieg an.
 \item Die Tuareg-Kämpfer besetzten 2013 strategisch wichtige Städte, islamtistische Terrorgruppen, obwohl sie sich eigentlich gegenüber stehen, schließen sich ihnen an. 
 \item Nachdem die Regierung Malis nach internationaler Hilfe fragte, verdrängte Frankreich die Kämpfer wieder. Die UNO und die EU beteiligen sich auch.
 \item Im Jahre 2015 wurde ein Friedensabkommen geschlossen, welches die Gruppen in die malische Armee einbezog und machen Regionen Malis mehr Autonomie gab. 
 \item Nato und weitere beendeten in 2020 ihre Unterstützung nachdem die malische Regierung geputscht wurde.
 \item In 2021 hat die malische Regierung Wagner angefragt.
 \item Die Tuareg-Kämpfer kommen bis heute weiter voran. 
\end{enumerate} 
\end{document}