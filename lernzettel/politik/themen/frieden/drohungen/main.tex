\documentclass{article}
\usepackage[a4paper]{geometry}
\usepackage{fancyhdr}
\pagestyle{fancy}
\lhead{Bedrohungen}
\rhead{Februar 2026}
\begin{document}
\section{Bedrohungen}
\subsection{Atomwaffen}
Der geltende Abrüstungsvertrag der USA und Russland ist nun ausgelaufen, Russland testet aktiv Interkontinentalraketen. Andere Länder wie China, Indien, der Iran, Nordkorea und Israel rüsten selbst auf. Noch immer sind Atomwaffen bedrohungen.
 
\subsection{Cyberangriffe}
Mit der Digitalisierung nimmt auch konstant die Gefahr durch Cyberangriffe auf kritische Infrastruktur zu.
 
\subsection{Klimawandel}
Durch den Klimawandel kommt es vermehrt zu Ressourcenknappheiten; Fragile Staaten werden noch Fragiler. Es kommt zur Migration aus diesen heraus.
 
\subsection{Herausforderungen für die deutsche Sicherheitspolitik}
Das Weißbuch der Sicherheitspolitik nannte in 2016 die Herausforderungen für die deutsche Sicherheitspolitik.
\begin{enumerate}
 \item Überstaatlicher Terrorismus
 \item Cyberangriffe
 \item Zwischenstaatliche Konflikte
 \item Fragile Staatlichkeit, schlechte Regierungen
 \item Globale Aufrüstung, verbreitung von Massenvernichtungswaffen
 \item Probleme kritischer Infrastruktur
 \item Klimawandel
 \item Unkontrollierte Migration
 \item Pandemien, Seuchen
\end{enumerate} 
\end{document}