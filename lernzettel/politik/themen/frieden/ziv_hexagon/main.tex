\documentclass{article}
\usepackage[a4paper]{geometry}
\usepackage{fancyhdr}
\pagestyle{fancy}
\lhead{Das Zivilisatorische Hexagon}
\rhead{November 2025}
\begin{document}
\section{Das Zivilisatorische Hexagon}
Das \emph{Zivilisatorische Hexagon}, erfunden vom Friedensforscher Dieter Senghaas, beschreibt in sechs Punkten welche Bedingungen für eine friedliche, stabile, Gesellschaft benötigt werden.
 
\subsection{Analyse} 
Um mithilfe dessen kann die Friedlichkeit einer Gesellschaft einzuschätzen müssen zuerst alle relevanten Information gesammelt werden. Mit diesen werden die einzelnen Aspekte ungefähr bewertet und die Wechselwirkungen dieser betrachtet. Daraus folgt eine Gesamteinschätzug der Gesellschaft und die Bestimmung der Handlungsfelder zur Förderung des Friedens.
 
\subsection{Aspekte}
\begin{description}
 \item[Gewaltmonopol] Das Gewaltmonopol des Staates soll gewährleistet sein, private BürgerInnen sollen entwaffnet sein.
 \item[Rechtsstaatlichkeit] Der gesamte Staat muss der rechtsstaatlichen Kontrolle unterliegen. Dies stellt sicher, dass es zu keiner Diktatur kommt. Hier zählt auch das Schützen von Grundrechten, der Menschenrechten, der Gleichberechtigung aller Mensfchen, die Gewaltenteilung, freie Wahlen, usw dazu.
 \item[Demokratische Partizipation] Die politische Teilhabe eines jeden muss ermöglicht werden. Gesellschaften, in welchen nicht jeder ein Recht auf die Teilhabe hat, werden unruhig.
 \item[Konstruktive Konfliktkultur] Der öffentliche Raum muss Platz für für unterschiedliche Meinungen und Kompromisse bieten. 
 \item[Soziale Gerechtigkeit] Es soll grundlegende soziale Gerechtigkeit herrschen, inklusive der Sicherung der Grundbedürfnisse. So fühlen sich in der Regel alle in der Gesellschaft fair aufgehoben.
 \item[Interdependenzen und Affektkontrolle] Unterschiedliche Personengruppen solle voneinander abhängig sein, so dass das Angreifen einer verbündeten Gruppe auch das Angreifen eines selbst darstellt. So sollen die BürgerInnen auch zur selbstkontrolle der Affekte, der kurzzeitigen Emotionen, geführt werden. 
\end{description} 
\end{document}