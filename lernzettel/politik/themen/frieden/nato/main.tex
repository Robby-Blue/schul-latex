\documentclass{article}
\usepackage[a4paper]{geometry}
\usepackage{fancyhdr}
\pagestyle{fancy}
\lhead{Die NATO}
\rhead{Oktober 2025}
\begin{document}
\section{Die NATO}
Die \emph{NATO}, die \emph{North Atlantic Treaty Organization}, ist ein Bündnis an Staaten des Nordatlantiks. Das Ziel ist es den Frieden zu sichern.
 
\subsection{Bündnisfall}
Nach Artikel 5 der Gründungsakte der NATO beschreibt den \emph{Bündnisfall}. Ein bewaffneter Angriff auf einen Staat der Nato wird als Angriff auf alle Nato-staaten angesehen. Alle Staaten sich dadurch verpflichtet sich in untereinander zu verteidigen. 
 
\subsection{Osterweiterung}
In den vergangenen Jahrzehnte hat sich die NATO immer weiter und weiter in den Osten von Europa, in diejenigen Länder, welche mal Teil, ausgeweitet.
 
Dies kann, meinen manche, von Russland als Bedrohung angesehen werden, welches zum Sicherheitsdillema führt. Russland und die NATO sehen in sich gegenseitig eine Bedrohung, weshalb sie Aufrüsten, wobei Aufrüstungen auch Bedrohung angesehen werden.
 
Dies, auf die Spitze gebracht durch den NATO-Gipfel 2008, auf welchem der Ukraine eine Mitgliedschaft in die NATO versprochen wurde, welches aber nie durchgesetzt wurde, sei auch einer der Gründe des Russland-Ukraine Krieges.  
 
\end{document}