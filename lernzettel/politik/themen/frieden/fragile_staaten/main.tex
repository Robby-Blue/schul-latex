\documentclass{article}
\usepackage[a4paper]{geometry}
\usepackage{fancyhdr}
\pagestyle{fancy}
\lhead{Fragile Staatlichkeit}
\rhead{November 2025}
\begin{document}
\section{Fragile Staatlichkeit}
Fragile Staaten sind Staaten, welche ihrer Staatsfunktionen nur noch eingeschrängt bis garnicht nachkommen können. Somit kann der Staat den BürgerInnen nur noch wenige, eingeschränkte, Dienstleistungen, eine schlechtere Infrastruktur und keine garantierte Sicherheit bieten.
 
Zerfällt der Staat komplett, kann er keine seiner Funktionen mehr erfüllen, so wird in der Regel die Macht von anderen Parteien mit Gewalt übernommen.
 
Staaten können auf einem Kontinuum zwischen einem konsolidierten, sehr startken, Staat und einem kollabiertem Staat gesehen werden. Dazischen sind schwache Staaten, zerfallende Staaten, welche auch als fragil gelten. 
 
\subsection{Staatsfunktionen} 
\begin{description}
 \item[Sicherheitsfunktion] Der Staat nutzt das Gewaltmonopol um die physische Sicherheit der BürgerInnen zu gewährleisten. Dazu zählt sowohl die Entwaffnung der BürgerInnen als auch das Sichern der Außengrenze.
 \item[Wohlfahrtsfunktion] Der Staat utnerstützt die BürgerInnen, verteilt Ressourcen. Dazu zählen Sozial-, Bildungs-, Gesundheitspolitik, usw.
 \item[Legitimitäts- und Rechtsstaatssfunktion] BürgerInnen können sich beteiligen, durch Wahlen zum Beispiel, und es besteht ein Rechtsstaat mit einer unabhängigen Justiz. Es gibt Freiheiten und Menschenrechte.
\end{description} 
\end{document}