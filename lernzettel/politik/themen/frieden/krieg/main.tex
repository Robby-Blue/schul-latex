\documentclass{article}
\usepackage[a4paper]{geometry}
\usepackage{fancyhdr}
\pagestyle{fancy}
\lhead{Krieg}
\rhead{Oktober 2025}
\begin{document}
\section{Krieg} 
Der Zustand des Krieges stellt die Abwesenheit von Frieden, von Sicherheit, dar.
 
\subsection{Kriegsdefinition} 
Der Kriegsdefinition der AKUF folgt Krieg den Kriterien, dass es mehr als eine \emph{zentral organisierte} und \emph{bewaffnete} Streitkraft gibt, wobei mindestens eine diese eine \emph{reguläre Regierungskraft} sein muss und der bewaffnete Konflikt \emph{dauerhaft} und planmäßig, einer Strategie der beteiligten nach, stattfindet. 
 
Ein Krieg gilt als beendet, wenn es mindestens ein Jahr lang keine bewaffneten Handlungen gab.
 
\emph{Bewaffnete Konflikte}, anstelle eines Krieges, folgen wenn obige Definition nicht völlig erfüllt wird.
 
\subsection{Ursachen}
Kriege haben eine Vielzahl an Ursachen, alle stammen aber aus irgendeiner Form einer massiven Unzufriedenheit einer Personengruppe. Zu den Ursachen einer dieser Unzufriedenheiten gehören
\begin{enumerate}
 \item unbeliebte Regime und politischen Systeme
 \item verlangen nach Autonomie
 \item verlangen nach Macht eines Staates
 \item folgen des Kolonialismus 
\end{enumerate}   
\end{document}