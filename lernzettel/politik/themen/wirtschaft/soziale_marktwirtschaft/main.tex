\documentclass{article}
\usepackage{multicol}
\usepackage{csquotes}
\usepackage[a4paper]{geometry}
\usepackage{fancyhdr}
\pagestyle{fancy}
\lhead{Soziale Marktwirtschaft}
\rhead{Mai 2025}
\begin{document} 
 
\section{Soziale Marktwirtschaft}
\begin{multicols}{1}
\noindent Der in der Nachkriegszeit begründete Begriff der Sozialen Marktwirtschaft beschreibt eine Marktwirtschaft, welche vermehrt auch sozial Aspekte aufweist. \newline
Die Sozial Marktwirtschaft bezieht sich dabei auf eine Anzahl an Prinzipien
\begin{description}
 \itemsep-0.3em 
 \item[Eigentumsprinzip] Jedem ist das Recht auf Privateigentum gewährleistet 
 \item[Haftungsprinzip] Jeder trägt für den eigenen Misserfolg Verantwortung
 \item[Wettbewerbsprinzip] Es soll einen Wettbewerb geben. Ist dieser nicht aufzufinden, weil es z.\,B. Monopole gibt, kann dieser erzwungen werden. In Deutschland ist dafür das Bundeskartellamt zuständig
 \item[Sozialprinzip] Es soll auch soziale Aspekte geben, welche die Ärmsten unterstützen.
 \item[Marktkonformitätsprinzip] Eingriffe der Politik in den Markt sind zu meiden. Stattdessen soll die Politik dem Markt konform arbeiten.
\end{description}
\columnbreak 
Somit kombiniert die Soziale Marktwirtschaft mehr oder weniger das Wirtschaftswachstum der freien Marktwirtschaft und die Gerechtigkeit der Zentralverwaltungswirtschaft. Es können aber natürlich nicht alle immer alle Prinzipien eingehalten werden, weil manche zueinander in Konkurrenz stehen. \newline 
Das die Bundesrepublik Deutschland eine Soziale Marktwirtschaft ist, ist durch das Grundgesetz Artikel 20 (1) \textquote{Die Bundesrepublik Deutschland ist ein demokratischer und sozialer Bundesstaat.} und das Grundgesetz Artikel 14 (1) \textquote{Das Eigentum und das Erbrecht werden gewährleistet. [...]} zu begründen. Es ist nicht direkt in der Verfassung festgeschrieben, dass Deutschland eine Sozial Marktwirtschaft sein muss, es könnte auch jede andere Wirtschaftsordnung, welche die genannten Kriterien erfüllt, sein.
\end{multicols}
 
\end{document}