\documentclass{article}
\usepackage[inline]{enumitem}
\usepackage{multicol} 
\usepackage[a4paper]{geometry}
\usepackage{fancyhdr}
\pagestyle{fancy}
\lhead{Freie Marktwirtschaft}
\rhead{Mai 2025}
\begin{document} 
 
\section{Freie Marktwirtschaft}
\begin{multicols}{1}
\noindent Das Konzept der freien Martwirtschaft, begründet von Adam Smith, geht davon aus, dass der Mensch, der \emph{Homo Economicus}, immer rational und selbstorientiert handeln würde, weshalb bei jeder Transaktion für alle Parteien der größtmögliche Profit herauskäme, weshalb das BIP so effizient wie möglich steigen würde. Dies ist die sogenannte \emph{Unsichtbare Hand}. \newline
Adam Smith nach würden die direkten Teilnehmer des Marktes mit dessen Nöten besser auskennen als die Regierung, weshalb diese sich aus dem Markt, z.\,B. durch Subventionen, heraushalten solle. Eine Aktion sei entweder für den Markt schädlich oder würde, wenn sie tatsächlich gut ist, von diesem selbst gefunden werden. \newline
Obwohl für Adam Smith ein kleiner Staat wichtig ist, gibt er diesem doch noch drei Aufgaben, welche er zu erfüllen hat. In der freien Marktwirtschaft ist es die Aufgabe des Staates
\begin{enumerate*}
 \item Die innere Sicherheit des Staates zu gewährleisten
 \item Gerechtigkeit für alle Bürger des Staates zu gewährleisten, also eine Rechtsstaat aufzubauen
 \item Öffentliche Infrakstruktur, welche insgesamt einen hohen Wert haben, aber keinen Profit erwirtschaften können, bereitstellen
\end{enumerate*}
\end{multicols}
 
\subsection{Grundwerte}
\begin{center}
\begin{tabular}{ |l|l|l|l|l| }
\hline
  & 
 \textbf{Freiheit} &
 \textbf{Gerechtigkeit} &
 \textbf{Sicherheit} \\
\hline
 \multicolumn{1}{|c|}{\textbf{Gewichtung}} &
 Am größten & geringer & geringer \\
\hline
 \multicolumn{1}{|c|}{\textbf{Ausführung}} &
 \begin{tabular}{@{}l@{}}
  Komplette wirtschaftliche \\ Freiheit, keine Engriffe, \\
  Recht auf Privateigentum
 \end{tabular} &
 Durch Staat & Durch Staat \\
\hline
 \multicolumn{1}{|c|}{\textbf{Chancen}} &
 \begin{tabular}{@{}l@{}}
  großes Wirtschaftswachstum \\
  für alle
 \end{tabular} &
 Staatlich gegeben &
 \begin{tabular}{@{}l@{}}
  Staatlich gegeben, \\
  wirtschaftliches \\
  Wachstum 
 \end{tabular} \\
\hline 
 \multicolumn{1}{|c|}{\textbf{Gefahren}} & 
 \begin{tabular}{@{}l@{}}
  Monopolbildung, \\ Ausbeutung, \\ Armut
 \end{tabular} &
 \begin{tabular}{@{}l@{}}
  Markt arbeitet \\ dagegen, Staat \\ nicht dafür
 \end{tabular} &
  \begin{tabular}{@{}l@{}}
  Regulation \\ fehlen, Anwählte \\ teuer
 \end{tabular} \\
\hline 
\end{tabular}
\end{center}  
  
\end{document} 
 
 
