\documentclass{article}
\usepackage[inline]{enumitem}
\usepackage{multicol} 
\usepackage[a4paper]{geometry}
\usepackage{fancyhdr}
\pagestyle{fancy}
\lhead{Wirtschaftsziele}
\rhead{Mai 2025}
\begin{document} 
 
\section{Wirtschaftsziele}
Im StabG, dem Stabilitätsgesetz, wurden 1967 erstmals im magischen Viereck vier Ziele zur Förderung der Stabilität und des Wachstums der Wirtschaft fest. Später wurde dieses um zwei weitere Ziele erweitert, sodass alle heutztage relevanten Wirtschaftsziele Deutschlands im \emph{magischen Sechseck} aufzufinden sind. Diese sind
\begin{description}
 \item[Außenwirtschaftliches Gleichgewicht] Der Warenexport soll langfristig mit dem Import ungefähr ausgeglichen sein damit es zu keinem dauerhaften Netto-Abfluss der eigenen Ressourcen und keine Nettoverschuldung gegenüber anderen kommt
 \item[Stabiles Preisniveau] Die jährliche Inflationsrate soll bei unter 2 Prozent bleiben. Gemessen wird diese anhand des Verbraucherpreisindexes, dem Preis aller Waren und Dienstleistungen, die private Haushalte für den Konsum kaufen
 \item[Gerechte Einkommens- und Vermögensverteilung] Hier ist aber nicht definiert, was genau "gerecht" bedeutet. Das soll eine individuelle politische Wertentscheidung sein
 \item[Umweltschutz] Umweltschäden, welche durch wirtschaftliche Aktivitäten verursacht werden, sollen vermieden werden.
 \item[Hoher Beschäftigungsstand] Die Arbeitslosenquote soll bei $<3\%$ liegen damit die Wirtschaft ihr Wirtschaftspotenzial anschmiegt und Arbeitslosigkeit verhindert wird
 \item[Stetiges und angemessenes Wirtschaftswachstum] Anhand des realen Bruttoinlandprodukts gemessen sollen überproportionale Konjunkturschwankungen vermieden werden. Dabei wird $2$ bis $4\%$ als angemessener Wert angesehen.  
\end{description}
 
\subsection{Beziehungen} 
\begin{multicols}{1} 
\noindent Das magische Sechseck wird "magisch" genannt, weil es nie möglich ist, alle Ziele komplett zu erreichen; stattdessen stehen viele in Konflikt zueinander.
\begin{description*}
 \item[Zielkomplimentarität] bedeutet, dass das erreichen eines Ziele automatisch auch ein anderes fördert. Liegt beispielsweise ein hoher Beschäftigungsgrad vor, liegt wohl auch ein Wirtschaftswachstum vor.
 \item[Zielneutralität] heißt, dass zwei Ziele sich gegenseitig nicht beeinflussen, weder positiv noch negativ. Zum Beispiel ist die gerechtigkeit der Einkommens und Vermögensverteilung davon wie viele Warenimporte es im Vergleich zu den Exporten gibt unabhängig.
 \item[Zielkonflikte] liegen vor, wenn zwei Ziele gegeneinander arbeiten. So ist eine dauerhaft wachsende Wirtschaft auf die komplette Ausbeutung des Ressourcen der Umwelt ausgelegt und schützt diese nie.
\end{description*} 
\end{multicols} 
 
\subsection{Handlungsfelder}
Die Maßnahemen, die der Staat tätigt, um die Wirtschaftsordnung eines Landes zu gestalten sind in drei Handlungsfelder aufgeteilt 
\begin{description}
 \item[Ordnungspolitik] legt die generellen langzeitigen Rahmenbedingungen einer Wirtschaft fest, z.\,B. welche Wettbewerbsordnung oder Sozialordnung es gibt
 \item[Strukturpolitik] bestimmt mittelfristig lang die Struktur innerhalb einer bestimmten wirtschaftlichen Region oder einem wirtschaftlichen Sektor. Dazu gehört zum Beispiel die Subventionierung vom Bau bestimmer Infrastruktur oder der Forschung, welche eine Branche helfen wird 
 \item[Prozesspolitik] greift auf bestimme Prozesse ein um kurzfristig die Konjunktur zu stabilisieren durch beispielsweise Steuererleichterungen
\end{description} 
\end{document} 
 
 
 
