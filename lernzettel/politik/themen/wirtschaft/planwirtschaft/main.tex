\documentclass{article}
\usepackage[a4paper]{geometry}
\usepackage{fancyhdr}
\pagestyle{fancy}
\lhead{Zentralverwaltungswirtschaft}
\rhead{Mai 2025}
\begin{document} 
 
\section{Zentralverwaltungswirtschaft}
In einer \emph{Zentralverwaltungswirtschaft}, auch \emph{Planwirtschaft} genannt, wird von einer zentralen Verwaltung aus die gesamte Wirtschaft geplant. So könnten, in der Theorie, alle Betriebe genau aufeinander abgestimmt werden. In der realität kommt es aber durch fehlenden Wettbewerb zu fehlender Motivation, Innovation und geringem Wachstum. Die Pläne waren historisch zu lang Gedacht und konnten nicht auf spontane Änderungen reagieren. \newline
Wirklich alles, also was produziert wird, wie es produziert wird, für welchen Preis es verkauft wird, und so weiter, wird vom Staat bestimmt. 
 
\subsection{Grundwerte}
\begin{center}
\begin{tabular}{ |l|l|l|l|l| }
\hline
  & 
 \textbf{Freiheit} &
 \textbf{Gerechtigkeit} &
 \textbf{Sicherheit} \\
\hline
 \multicolumn{1}{|c|}{\textbf{Gewichtung}} &
 geringer & hoch & höher \\
\hline
 \multicolumn{1}{|c|}{\textbf{Ausführung}} &
 \begin{tabular}{@{}l@{}}
  keine wirtschaftliche \\
  Freiheit, kein Privat- \\
  eigentum 
 \end{tabular} &
 Staatlich & 
 Staatlich \\
\hline
 \multicolumn{1}{|c|}{\textbf{Chancen}} &
 \begin{tabular}{@{}l@{}}
  keine Ausbeutung \\
  mehr Gerechtigkeit 
 \end{tabular} &
 \begin{tabular}{@{}l@{}}
  Staatlich gegeben \\
  subventionen 
 \end{tabular} &
 \begin{tabular}{@{}l@{}}
  Staatlich gegeben \\
  subventionen 
 \end{tabular} \\
\hline 
 \multicolumn{1}{|c|}{\textbf{Gefahren}} & 
 \begin{tabular}{@{}l@{}}
  geringeres Wachstum \\
  geringe Innovation \\
  gerine Motivation \\ 
  fehlende Produkte \\
  fehlende Verantwortung 
 \end{tabular} &
 \begin{tabular}{@{}l@{}}
  verschwendung von \\
  subventionierter \\
  Ware 
 \end{tabular} &
 \begin{tabular}{@{}l@{}}
  fehlende Verantwortung 
 \end{tabular} \\
\hline 
\end{tabular}
\end{center}
 
\end{document} 
 
 
 
