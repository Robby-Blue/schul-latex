\documentclass{article}
\usepackage[a4paper]{geometry}
\usepackage{fancyhdr}
\pagestyle{fancy}
\lhead{Die Medialisierung der Politik}
\rhead{Oktober 2025}
\begin{document}
\section{Die Medialisierung der Politik}
Die Politik hat sich durchaus \emph{Medialisiert}, heißt, sich an die Logik der Medien angepasst.
 
Dazu zählen Unteranderem die Skandalisierung, die Personalisierung und die Emotionalisierung der Politik um Aufmerksamkeit zu bekommen, auch wenn es die Wahrheit vereinfacht, verzerrt oder auch in Teilen falsch darstellt. So sind es die kurzen Videos mit skandalösen Aussagen oder die unpolitische, symbolische, selbstdarstellung und nicht die ausführlichen politischen Analysen, welche viral gehen und im Alltag konsumiert werden.
\end{document}