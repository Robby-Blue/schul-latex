\documentclass{article}
\usepackage{hyperref}
\usepackage{csquotes}
\usepackage[a4paper]{geometry}
\usepackage{fancyhdr}
\pagestyle{fancy}
\lhead{Die Macht der Medien}
\rhead{Oktober 2025}
\begin{document}
\section{Die Macht der Medien}
Wird die Macht der Medien analysiert, so gibt es drei hauptsächliche Theorien: die \emph{Dependenzthese}, die \emph{Instrumentalisierungsthese} und die \emph{Interdependenzthese}.
\begin{description}
 \item[Dependenzthese] Die Dependenzthese beschreibt die Idee, dass die Politik eine Dependenz, eine Abhängigkeit, gegenüber den Medien haben. Somit kontrollieren diese sozusagen die Politik mit, gelten als \emph{vierte Gewalt}. Hier wird die \hyperref[Die Medialisierung der Politik]{Medialisierung} als Beweis für gesehen.
 \item[Instrumentalisierungsthese] Die Politik hat hier die Kontrolle und kann die Medien zu ihren eigenen Zwecken instrumentalisieren. Hier wird die Professionalisierung der politischen Öffentlichkeitsarbeit als Beleg gesehen.
 \item[Interdependenzthese] Dis ist die Mitte der beiden obigen Thesen, welche von einer Gleichstellung der Medien und Politik ausgeht. Somit gibt es hier ein \textquote{symbiotisches Verhältnis} zwischen den Medien und der Politik.
\end{description} 
\end{document}