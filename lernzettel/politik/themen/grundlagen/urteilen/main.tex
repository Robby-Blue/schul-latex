\documentclass{article}
\usepackage{hyperref}
\usepackage[a4paper]{geometry}
\usepackage{fancyhdr}
\pagestyle{fancy}
\lhead{Beurteilen}
\rhead{November 2025}
\begin{document}
\section{Beurteilen}
Ist ein Urteil zu einem Thema gefragt, kann das Thema mithilfe von drei hauptsächlichen Kategorien analysiert und das Urteil gefällt werden:
\begin{description}
 \item[Effizienz] Eine Maßnahme ist Effizient, wenn sie effektiv, schnell Durchsetzbar ist, bei geringen Kosten eine große Auswirkung hat, ohne Nebenwirkungen.
 \item[Legitimität] Etwas ist Legitim, wenn es generell Rechtsmäßig, der Verfassung nach, ist, nicht die Menschenrechte beeinträchtigt, transparent ist, selbstbestimmung erlaubt und so weiter. 
 \item[\mbox{\hyperref[Grundwerte]{Grundwerte}}] Entscheidungen sollten den Grundwerten folgen. Diese können aber unterschiedlich gewichtet sein.  
\end{description} 
Dabei ist zu beachten, dass die Effizienz an erster Stelle ein Tatsache, also ein Sachurteil ist während die Grundwerte eine Wertung, ein Werturteil darstellen. Die Legitimität liegt dazwischen.  
 
Anhand dieser Kategorien können Argumente gefunden, sortiert werden und ein eigenes Urteil gefunden und formuliert werden. 
 
\end{document}