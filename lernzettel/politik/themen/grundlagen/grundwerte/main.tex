\documentclass{article}
\usepackage{csquotes}
\usepackage[a4paper]{geometry}
\usepackage{fancyhdr}
\pagestyle{fancy}
\lhead{Grundwerte}
\rhead{Oktober 2025}
\begin{document}
\section{Grundwerte}
Es gibt insgesamt fünf Grundwerte, die ersten drei davon am wichtigsten, welche analysiert werden können. Zu diesen gehört
\begin{description}
 \item[Sicherheit] Zur \emph{Sicherheit} gehört die gesellschaftliche eines Gesamten als auch persönliche Sicherheit von einer jeden Person. Darunter fällt die Abwesenheit von Bedrohungen durch andere Staaten und Personen.
 \item[Freiheit] Der Grundwert der \emph{Freiheit} wird zwischen der \emph{formellen} Freiheit und der \emph{materialen} Freiheit unterschieden. Zur formellen Freiheit gehören die Rechte einer jeden Person oder Organisation, ohne staatlicher Einschränkung dem eigenen Willen nachgehen zu können. Die materiale Freiheit schränkt dies weiter ein, sie erkennt an, dass eine Person nur das alles machen kann, welches sie sich auch Leisten kann, sei es finanziell oder zeitlich oder ähnliches.
 \item[Gerechtigkeit] Hier wird zwischen der \emph{Verfahrens}gerechtigkeit und der \emph{Verteilungs}gerechtigkeit unterschieden. Die Verfahrensgerechtigkeit begründet, dass bei politischen Entscheidungen oder bei Verfahren in der Justiz gerecht gehandelt werden sollte. Die Verteilungsgerechtigkeit beschäftigt sich mit der Verteilung des Einkommens und des Vermögens und allem dies folgendem. Was hier mit \textquote{Gerecht} gemeint ist sei Ansichtssache, es gibt drei hauptsächliche Gerechtigkeitsdefinitionen:
 \begin{enumerate}
  \item das \emph{Leistungsprinzip} fordert, dass die Leistung aller Personen gemessen werden sollte und diejenigen, welche mehr Leisten, auch mehr bekommen sollten.
  \item das \emph{Egalitätsprinzip}, auch \emph{Gleichheitsprinzip} genannt, fordert, dass alle das Gleiche bekommen.
  \item das \emph{Bedarfsprinzip} fordert, dass alled das bekommen, was sie brauchen.
 \end{enumerate}
 \item[Solidarität] \emph{Solidarität} ist der Zusammenhalt einer Gruppe. Hier ist das Nutzen von Steuern zum wohl der Allgemeinheit gemeint.
 \item[Nachhaltigkeit] Die \emph{Nachhaltigkeit} beschreibt dass Ressourcen genutz werden können, auf eine Art und weise, so dass sie sich in einer angemessenen Zeit regenrieren können. Dies kann ökologisch, sozial und ökonomisch angesehen werden.  
\end{description} 
\end{document}