\documentclass{article}
\usepackage[a4paper]{geometry}
\usepackage{fancyhdr}
\pagestyle{fancy}
\lhead{Operatoren}
\rhead{November 2025}
\begin{document}
\section{Operatoren}
Bei allen Operatoren, bei allen Aufgaben, ist mit Quellenverweisen, mit Zeilenangaben, zu arbeiten. 
\subsection{Zusammenfassen}
Eine Zusammenfassung fängt mit einer Einleitung an, in welcher Fakten über den vorliegenden Text wie den die Textart, den Titel, die AutorIn, das Veröffentlichungsdatum, und so weiter, und einer kurzen Nennung vom Thema des Textes, genannt werden. Daraufhin muss der Text strukturiert, mit sprachlicher Distanz, also passenden Formulierungen und dem Konjunktiv, ohne eigener Kommentare in den wichtigsten Aussagen ($\approx 1/3$ Länge) wiedergegeben werden.
 
\subsection{Beurteilen}
Es muss ein Sachurteil gegenüber einem Sachverhalt oder Prozess gefällt werden.  
 
\subsection{Stellung nehmen}
Es soll mithilfe von Urteilskriterien ein Werturteil gefällt werden. Zu den Urteilskriterien kann der Nutzen, die Folgen, das einhalten der Grundwerte, die Transparenz des zu bewertenden Sachverhalts genutzt werden. 
 
\subsection{Erörtern}
Sich reflektiert mit einer Problemstellung auseinandersetzen und zu einem Sach- und/oder Werturteil kommen. Dabei sollen aus verschiedenen Perspektiven Pro- und Kontra-Argumente gefunden werden, Anhand von Urteilskriterien gewichtet werden und zuletzt ein eigenes Urteil gefällt werden. 
 
\subsection{Karikaturen Analysieren}
Beim Analyisieren von Karkaturen wird müssen die Daten über die Karikatur und was in der Karikatur zu sehen ist benannt werden. Daraufhin müssen die sichtbaren Symbole analysiert werden, z.\,B. wofür sie stehen. Als vorletztes muss die Karikatur an sich gedeutet werden, die Aussage und Meinung der KarikaturistIn benannt werden. Zuletzt muss eine eigene Meinung genannt werden und etwas Kritik geäußert werden.  
\end{document}