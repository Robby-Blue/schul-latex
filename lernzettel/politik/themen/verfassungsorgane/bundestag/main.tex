\documentclass{article}
\usepackage{csquotes}
\usepackage{hyperref}
\usepackage[a4paper]{geometry}
\usepackage{fancyhdr}
\pagestyle{fancy}
\lhead{Der Bundestag}
\rhead{Oktober 2025}
\begin{document}
\section{Der Bundestag}
\emph{Der Bundestag} stellt das deutsche Parlament dar, bestehend aus den Abgeordneten. Dieses hat eine Vielfalt an Aufgaben:
\begin{description}
 \item[Wahlfunktion] Der Bundestag selbst wählt weitere Ämter, wie das Amt der \hyperref[Die BundeskanzlerIn]{BundeskanzlerIn}, (zur Hälfte) das Amt der \hyperref[Die BundespräsidentIn]{BundespräsidentIn} oder auch die Hälfte der \hyperref[Das Bundesverfassungsgericht]{BundesverfassungsrichterInnen}. Andersherum kann der Bundestag die BundeskanzlerIn auch wieder abwählen, durch ein Misstrauensvotum zum Beispiel.
 \item[Gesetzgebungsfunktion] Der Bundestag muss grundlegende sowie wesentliche Entscheidungen treffen, an erster Stelle indem Gesetze verabschieded werden. Weil die Gesetze nicht im Bundestag selbst debbatiert, sondern in den Fachausschüssen bearbeitet werden, ist der Bundestag ein \emph{Arbeitsparlament}.
 \item[Kontrollfunktion] Der Bundestag hat die Aufgabe die \hyperref[Die Bundesregierung]{Bundesregierung} zu kontrollieren. Dies kann durch Anfragen an die Regierung zur öffentlichen Redestellung oder zum erlangen von Informationen oder durch eine Klage beim Bundesverfassungsgericht geschehen.
 \item[Kommunikationsfunktion] Der Bundestag agiert als \textquote{Vermittler} zwischen den BürgerInnen und dem Staate. Dazu zählt sowohl dass der Bundestag, beziehungsweise dessen Abgeordneten, den BürgerInnen zuhört als auch als \textquote{Sprachrohr} zu agieren. Hierfür sind die Plenardebatten ein Beispiel, welche nicht wirklich zum debattieren genutzt werden (siehe Oben), sondern um den Stand der Debatte an die BürgerInnen zu kommunizieren.
\end{description} 
 
\subsection{Entparlamentarisierung} 
Der Begriff der \emph{Entparlamentarisierung} beschreibt die These, dass der Bundestag viel weniger Macht habe als angenommen. Demnach würde der Bundestag nur diejenigen Gesetze, welche anderswo, z.\,B. durch die Regierung, Entschieden wurden, abstempeln, ohne zu diesen selbst etwas beizutragen. Dies stimmt nur in Teilen; der Bundestag hat aber sehr wohl noch ein Mitspracherecht, durch die obigen Funktionen und der Teilnahme an relevanten Gremien und Ausschüssen.
\end{document}