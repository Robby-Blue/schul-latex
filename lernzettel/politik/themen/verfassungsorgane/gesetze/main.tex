\documentclass{article}
\usepackage[a4paper]{geometry}
\usepackage{fancyhdr}
\pagestyle{fancy}
\lhead{Die Entstehung eines Gesetzes}
\rhead{Oktober 2025}
\begin{document}
\section{Die Entstehung eines Gesetzes}
Der Entstehungsprozess eines Gestzes ist in mehrere Schritte aufgeteilt
\begin{enumerate}
 \item Ein \emph{Gesetzesentwurf} wird in den Bundestag eingebracht, in der Regel von der Regierung. Die Fachministerien erarbeiten dies.
 \item Es kommt zur \emph{ersten Beratung} im Bundestag, jedoch ohne Beschluss, sondern nur zur Begründung des Vorhabens.
 \item Das Gesetz wird von den \emph{Ausschüssen} beraten, wobei auch externe ExpertInnen eingeladen werden können. Ein Ausschuss kann dem Bundestag eine Annahme, eine Abänderung oder eine Ablehnung des Gesetzes empfehlen.
 \item Auf Grundlage der Empfehlungen der Ausschüsse kommt es zu einer \emph{zweiten Lesung} (Beratung), bei welcher die Abgeordneten und die Fraktionen \emph{Änderungsanträge} einbringen können.
 \item Bei der darauffolgenden \emph{dritten Lesung} können nurnoch Fraktionen oder Gruppen an Abgeordneten mit $>5\%$ Änderungsanträge stellen. Anschließend kommt es zur \emph{namentlichen} und zur \emph{geheimen Abstimmung}. 
 \item Ein vom Bundestag beschlossenes Gesetz wird an den Bundesrat geleitet, welcher dem Gesetz zustimmen muss. Ist es aber ein Einspruchsgesetz, kann der Bundestag den Einspruch überstimmen.
 \item Droht das Gesetz zu scheitern, so kann ein \emph{Vermittlungsausschuss} ausgerufen werden um ein Kompromiss zwischen dem Bund und den Ländern zu finden. 
 \item Das Gesetz wird von den zuständingen MinisterInnen, der KanzlerIn und der BundespräsidentIn unterzeichnet und im Bundesgesetzblatt verkündet. 
\end{enumerate} 
\end{document}