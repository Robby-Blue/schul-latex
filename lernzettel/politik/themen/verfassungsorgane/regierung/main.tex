\documentclass{article}
\usepackage[a4paper]{geometry}
\usepackage{fancyhdr}
\pagestyle{fancy}
\lhead{Bundesregierung}
\rhead{Oktober 2025}
\begin{document}
\section{Die Bundesregierung}
Die \emph{Bundesregierung} hat zwei Hauptaufgaben:
\begin{description}
 \item[Steuerungsfunktion] Die Bundesregierung hat die Aufgabe, die Absichten der Mehrheit in Gesetzestexte zu verfassen. Dies soll dabei, so weit es geht, im Rahem des Finanziellen und des Widerspruchsfreien bleiben.
 \item[Durchführungsfunktion] Diese Gesetze soll die Bundesregierung, z.\,B. durch Verordnungen und durch organisatorische Maßnahmen, durchsetzen. 
\end{description}
Dabei muss sich jede Regierung, welche sie eine Koalition aus mehreren Parteien ist, so wie es alle Bundesregierungen bisher waren, auf einen \emph{Koalitionsvertrag} einigen, welche auch großteils eingehalten werden sollte.
 
\subsection{Prinzipien}
Die Regierung handel nach einer Handvoll an Prinzipien:
\begin{description}
 \item[Kanzlerprinzip] Die KanzlerIn bestimmt die Richtlinien und trägt die Verantwrotung über diese.
 \item[Ressortprinzip] Alle MinisterInnen leiten ihr eigenes Ressort selbstständing und unter eigener Verantwortung
 \item[Kollegialprinzip] Die MitgliederInnen beraten und entscheiden gemeinsam über Gesetzesentwürde und bei Streitfragen zwischen den MinisterInnen. 
\end{description} 
\end{document}