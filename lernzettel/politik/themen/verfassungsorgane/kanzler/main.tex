\documentclass{article}
\usepackage{csquotes}
\usepackage{hyperref}
\usepackage[a4paper]{geometry}
\usepackage{fancyhdr}
\pagestyle{fancy}
\lhead{Bundeskanzler}
\rhead{Oktober 2025}
\begin{document}
\section{Die BundeskanzlerIn} 
Zentral in der \hyperref[Die Bundesregierung]{Bundesregierung} ist die Rolle der BundeskanzlerIn. Die BundeskanzlerIn wird dabei vom Bundestag auf Vorschlag der BundespräsidentIn gewählt.
 
Eine gewählte BundeskanzlerIn hat das Recht der BundespräsidentIn vorzuschlagen MinisterInnen, zu ernennen oder zu entlassen, und bestimmt \textquote{die Richtlinien der Politik und trägt dafür die Verantwortung} (Grundgesetz Artikel 64).
 
\subsection{Eingrenzungen} Die Rolle der MinisterInnen sollten weiterhin proportional, in Bezug auf Parteiangehörigkeit, auf das Geschlecht, auf die Ost- oder Westdeutsche Herkunft, usw. bleiben. Darüberhinaus ist die BundeskanzerIn auf die Unterstützung des Bundestages angewiesen. 
\end{document}