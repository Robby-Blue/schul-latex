\documentclass{article}
\usepackage{csquotes}
\usepackage[a4paper]{geometry}
\usepackage{fancyhdr}
\pagestyle{fancy}
\lhead{Das Bundesverfassungsgericht}
\rhead{Oktober 2025}
\begin{document}
\section{Das Bundesverfassungsgericht}
Das \emph{Bundesverfassungsgericht}, kurz das \emph{BVerfG}, hat vier hauptsächliche Aufgaben:
\begin{itemize}
 \item Die Grundrechte einzelner Personen zu schützen. Glaubt eine Person, dass dessen Grundrechte durch ein Gesetz oder ein anderes Gericht eingeschränkt wurden, so kann diese, solage alle anderen Gerichte bereits wiedersprochen haben, eine \emph{Verfassungsbeschwerde} einreichen.
 \item Bei einer \emph{Normenkontrolle} wird ein verabschiedetes Gesetz auf die Verfassungsmäßigkeit überprüft. Dies wird unterschieden zwischen \emph{konktreten} und \emph{abstrakten} Normenkontrollen
 \begin{description}
  \item[konkrete] Normenkontrollen werden von einem Fachgericht eingeleitet.  
  \item[abstrakt] Normenkontrollen werden von der Bundesregierung, eines Landesregierung oder $1/4$ des Bundestages eingeleitet.
 \end{description} 
 \item Streitigkeiten zwischen zwei Verfassungsorganen, in der Regel darüber, welches Organe eine bestimmte Aufgabe hat, werden vom BVerfG geregelt.
 \item Das BVerfG kann parteien Verbieten. 
\end{itemize} 
Keine einzige der obige genannten Aufgaben kann das BVerfG von such aus nachgehen. Das BVerfG wird nur dann aktiv, wenn es durch eine Klage eines anderen dazu aufgerufen wird. 
 
\subsection{Wahl der BundesverfassungsrichterInnen} 
Der Bundestag und der Bundesrat wählen jeweils die hälfte der BundesverfassungsrichterInnen.
 
\end{document}