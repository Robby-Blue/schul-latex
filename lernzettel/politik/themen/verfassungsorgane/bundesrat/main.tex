\documentclass{article}
\usepackage[a4paper]{geometry}
\usepackage{fancyhdr}
\pagestyle{fancy}
\lhead{Der Bundesrat}
\rhead{Oktober 2025}
\begin{document}
\section{Der Bundesrat}
Der \emph{Bundesrat} geht der Aufgabe nach die Intressen der Bundesländer auch auf Bundesebene zu vertreten. Der Bundesrat besteht aus entsanden Mitglieder der Landesregierungen, in der Regel inklusive der MinisterpräsidentInnen.
 
Dabei verpflichten die Mitglieder der Länder dazu im Intresse der Länder zu handeln, auch wenn dies gegen die Parteilinie auf Bundesebene geht.
 
Die Anzahl der Mitglieder eines Landes hängt von der Einwohnerzahl desselben ab. 
 
\subsection{Rechte}
Der Bundesrat kann eigene Gesetzesvorschläge machen, zu Gesetzen Stellung nehmen oder bei \emph{zustimmungspflichtigen} Gesetzen einen Einspruch einlegen. Der Bundestag kann den Einspruch mit gleicher Mehrheit (einfache-, Zweidrittel-) wieder überstimmen. Bei Zustimmungsgesetzen kann vom Bundestag, der Regierung oder dem Bundesrat ein Vermittlungsausschuss ausgerufen werden, um das Scheitern von Gesetzen zu verhindern.
 
\subsection{Das Verstummen}
Aufgrund immer größerer Koalitionen wird es mit der Zeit schwerer im Bundesrat Mehrheiten zu finden, weshalb es vermehrt zu Enthaltungen und Uneinigkeiten kommt. So hat der Bundesrat in er $19.$ Wahlperiode nur für sieben Gesetze ein Vermittlungsausschuss ausberufen und nur ein einziges Gesetz tatsächlich blockiert.   
\end{document}