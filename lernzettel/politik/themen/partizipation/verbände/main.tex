\documentclass{article}
\usepackage[a4paper]{geometry}
\usepackage{fancyhdr}
\pagestyle{fancy}
\lhead{Intressenverbände}
\rhead{Oktober 2025}
\begin{document}
\section{Intressenverbände}
Intressenverbände sind Verbände, welche eher bestimmtere Intressen der BürgerInnen vertreten. Diese sind vom Parlament unabhängig, haben also keine direkte Einflussnahme auf die Regierung. Sie basieren auf dem Grundgesetz, Artikel 9.
 
Die Funktionen, welche die Intressenverbände erfüllen, können wie folgt beschrieben werden 
\begin{description}
 \item[Artikulation] Intressenverbände artikulieren ihre eigenen Intressen.
 \item[Aggregation] Innerhalb eines Intressenverbandes werden die Intressen der einzelnen BürgerInnen aggregiert und zusammengeführt.
 \item[Vermittlung] Intressenverbände können bei wichtigen politischen Entscheidungen beteiligt sein. Ist dies der Fall, so müssen sie ihren MitgliederInnen das Ergebnis der Verhandlung mitteilen. 
 \item[Information] In der Regel haben die MitarbeiterInnen der Intressenverbände wichtiges Fachwissen, welches bei politischen Entscheidungen relevant sein kann.
\end{description}
 
\subsection{Einflussnahme} 
% TODO: erweitern 
 
\subsection{Finanzkraft} 
Natürlich kann einfach mit Geld Einfluss genommen werden. Damit mit Einflussnahme legitim bleibt wird unter Anderem das \emph{Lobbyregister} geführt, größere \emph{Parteispenden} müssen veröffentlicht werden, \emph{Nebeneinkünfte} müssen Transparent sein und Tatigkeiten nach der Regierungsbeteiligung müssen gemeldet werden.   
\end{document}