\documentclass{article}
\usepackage[a4paper]{geometry}
\usepackage{fancyhdr}
\pagestyle{fancy}
\lhead{Intressenverbände}
\rhead{Oktober 2025}
\begin{document}
\section{Intressenverbände}
Intressenverbände sind Verbände, welche eher bestimmtere Intressen der BürgerInnen vertreten. Diese sind vom Parlament unabhängig, haben also keine direkte Einflussnahme auf die Regierung. Sie basieren auf dem Grundgesetz, Artikel 9.
 
Die Funktionen, welche die Intressenverbände erfüllen, können wie folgt beschrieben werden 
\begin{description}
 \item[Artikulation] Intressenverbände artikulieren ihre eigenen Intressen.
 \item[Aggregation] Innerhalb eines Intressenverbandes werden die Intressen der einzelnen BürgerInnen aggregiert und zusammengeführt.
 \item[Vermittlung] Intressenverbände können bei wichtigen politischen Entscheidungen beteiligt sein. Ist dies der Fall, so müssen sie ihren MitgliederInnen das Ergebnis der Verhandlung mitteilen. 
 \item[Information] In der Regel haben die MitarbeiterInnen der Intressenverbände wichtiges Fachwissen, welches bei politischen Entscheidungen relevant sein kann.
\end{description}
 
\subsection{Einflussnahme}
Verbände können Einfluss nehmen, indem die öffentliche Meinung beeinflusst wird, indem an Parteien gespendet wird oder Kontake, Informationen oder Beratung an die Ministerialbürokratie, der Bundesregierung oder dem Rest des Bundestages gegeben werden. 
 
Wie Einflussreich ein Verband ist hängt von einer Menge an Faktoren ab. Zu diesen gehören
\begin{enumerate}
 \item Ob die Intresse egoistisch oder allgemeinnützig ist. Egoistische Intressen sind in der Regel durchsetzungsfähiger.
 \item Die Konzentration der MitgliederInnen des Verbandes.
 \item Die Anzahl der MitgliederInnen im Vergleich zur potenziellen Anzahl an MitgliederInnen. Daraus folgt die Kraft Massen zu mobilisieren.
 \item Der Eindruck auf die öffentlichkeit: wie verständlich, überzeugend und sympathisch die Argumente sind.
 \item Die Finanzielle lage des Verbandes. 
\end{enumerate} 
 
\subsection{Finanzkraft} 
Natürlich kann einfach mit Geld Einfluss genommen werden. Damit mit Einflussnahme legitim bleibt wird unter Anderem das \emph{Lobbyregister} geführt, größere \emph{Parteispenden} müssen veröffentlicht werden, \emph{Nebeneinkünfte} müssen Transparent sein und Tatigkeiten nach der Regierungsbeteiligung müssen gemeldet werden.   
\end{document}