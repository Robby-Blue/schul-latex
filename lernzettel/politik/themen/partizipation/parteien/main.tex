\documentclass{article}
\usepackage[a4paper]{geometry}
\usepackage{fancyhdr}
\pagestyle{fancy}
\lhead{Parteien}
\rhead{Oktober 2025}
\begin{document}
\section{Parteien}
Politisch kann weiterhin in Parteien selbst partizipiert werden.
 
Durch Parteien werden die nächsten PolitikerInnen \emph{ausgewählt}, welche dann die Intressen der MitgliederInnen der Partei repräsentieren, also an die Politik \emph{vermitteln}, können, nachdem diese innerparteilich \emph{gefunden} wurden. Dadurch \emph{legitimieren} die das politische System. 
 
\subsection{Gesetzliche Lage}
Parteien müssen innerparteilich demokratisch organisiert sein und die Herkunft und Verwendung ihrer Mittel öffentlich darlegen. Parteien sollen zwischen den BürgerInnen und dem Parlament sitzen, so dass sie dessen Intressen diese bündeln und die BürgerInnen in sich mit einbinden, sodass eine KandidatIn aufgestellt werden kann.
 
\subsection{Parteigründungen}
Die Gründung einer Partei kann in der Regel durch die \emph{gesellschaftlichen Konfliktlinien}, die \emph{cleavages}, erklärt werden. Neue Parteien verfolgen das Ziel, eine noch nicht repräsentierte Meinung in den Bundestag zu bringen, so stehen sie, wird ihre Position zusammen mit allen anderen Parteien im Bezug auf die Konfliktlinien dargestellt, weit weg von allen anderen Parteien. So finden sie ihre eigene Niche an Meinungen und WählerInnen. 
 
Beispiele dafür sind Die Linke, welche besonders viel Acht auf ArbeiterInnenrechte legt und somit die Niche einnimmt, welche mal von der SPD besetzt war, und die AfD, welche besonders Rechtsextrem ist und somit gesondert einen Wahlerfolg bei Rechtsextremem hatte.
 
Die cleavages sind aber nicht konstant: mit der Zeit werden neue cleavages relevanter und andere werden unwichtiger. In den letzten Jahren hat die \emph{Globalisierungs}-Cleavage an relevanz bekommen. Dieses Modell hat aber auch Grenzen, wie dass es komplexe Standpunkte vereinfacht und. 
 
\subsection{Der Fall der Volksparteien}
In den letzten Jahrzenten ist die MitgliederInnenzahl der \emph{Volksparteien} --- CDU und SPD --- massiv gesunken. Stand 2022 sind beide dieser Parteien, im Vergleich zu 1990, auf circa $1/2$, wenn nicht sogar weitaus weniger, der ehemaligen MitgliederInnenzahlen gesunken. 
\end{document}