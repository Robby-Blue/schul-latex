\documentclass{article}
\usepackage{csquotes}
\usepackage[a4paper]{geometry}
\usepackage{fancyhdr}
\pagestyle{fancy}
\lhead{Wahlen}
\rhead{Oktober 2025}
\begin{document}
\section{Wahlen}
Die wohl einfachste Methode der politischen Partizipation ist die Teilnahme an einer Wahl, vorallem an einer Bundestagswahl.
 
Bei dieser kann generell jede BürgerIn wählen (und kandidieren), wobei jede Stimmabgabe ohne geheim und äußerem Druck passieren soll. Daraus soll dann direkt ein Wahlergebniss entstehen, ohne dass eine Stimm mehr zählt als die andere.
 
Mit der Wahl sollen die Meinungen der BürgerInnen repräsentiert und in der Politik integriert werden, wodurch dessen Herrschaft auch legitimiert und kontrolliert wird.
 
\subsection{NichtwählerInnen}
Ein nicht kleiner Teil der Bevölkerung, welcher mit der Zeit insgesamt wächst, entscheided sich dazu nicht zu wählen. Dazu gibt es verschiedene Gründe, wie das \textquote{unpolitisch} Sein, die Unzufriedenheit mit dem System, die Unzufriedenheit mit den Parteien oder dass momentär die Wahl ungünstig erscheint.
 
Auf der einen Seite kann angemerkt werden, dass NichtwählerInnen sich dazu entscheiden ihre Stimme für die Demokratie nicht zu nutzen, wovon in der Regel die extremen Parteien profitieren, und die oben beschriebene Legitimation zu senken, nutzen aber auf der anderen Seite auch ihr Wahlrecht, so wie es ihnen gegeben wurde.
\end{document}