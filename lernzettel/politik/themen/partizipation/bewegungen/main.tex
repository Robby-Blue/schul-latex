\documentclass{article}
\usepackage[a4paper]{geometry}
\usepackage{fancyhdr}
\pagestyle{fancy}
\lhead{Soziale Bewegungen}
\rhead{Oktober 2025}
\begin{document}
\section{Soziale Bewegungen}
Soziale Bewegungen sind Phänomene, bei welcher sich viele unterschiedliche Personen oder Gruppen mit einem gemeinsamen Ziel auf eine längere Zeit leicht organisiert arbeiten um die EntscheidungsträgerInnen zu beeinflussen, in der Regel mit dem Ziel der gesellschaftlichen Veränderung in einem bestimmten Bereich.
 
Bewegungen sind generell in Teilgruppen, die Bewegungsorganisationen, aufgeteilt, welche spezifischer, lokaler, arbeiten.
 
\subsection{Beispiele}
Zu Beispielen gehören Frauenbewegungen, ArbeiterInnenbewegungen, Umweltbewegungen (z.\,B.\texttt{Fridays for Future}) oder auch antiglobalistische Bewegungen wie \texttt{Attac} oder \texttt{Occupy Wall Street}.
 
\end{document}