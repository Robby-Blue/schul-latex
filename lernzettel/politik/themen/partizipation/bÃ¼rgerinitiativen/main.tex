\documentclass{article}
\usepackage[a4paper]{geometry}
\usepackage{fancyhdr}
\pagestyle{fancy}
\lhead{Bürgerinitiativen}
\rhead{Oktober 2025}
\begin{document}
\section{Bürgerinitiativen}
Die \emph{Bürgerinitiativen} erlauben BürgerInnen sich auf Dauer zusammen zu vereinen, um ein spezifisches, in der Regel lokales oder regionales, Intresse zu vertreten. Dies passiert durch \emph{informelle Macht}, heißt z.\,B. öffentlicher Druck.
 
\subsection{Gründung} 
In der Regel wird eine Bürgerinitiative gegründet, nachdem einfache Öffentlichkeitsarbeit und Aufrufe zur veränderung nichts gebracht haben. Nach dem Gründen wird in der Regel kontak zur Parteien gesucht und Kompromisse gefunden.
 
\subsection{Repräsentation} 
Arbeit in einer Bürgerinitiative braucht viel Zeit. Deshalb wird die oftmals überdurchschnittlich viel von denjenigen verrichtet, welche diese viele Zeit mit sich bringen können. Das sind die gut-verdienenden, Gebildeten, Alten, Kinderlosen, Männer. 
\end{document}