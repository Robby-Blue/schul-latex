\documentclass{article}
\usepackage[a4paper]{geometry}
\usepackage{fancyhdr}
\pagestyle{fancy}
\lhead{Partizipationsformen Analysieren}
\rhead{Oktober 2025}
\begin{document}
\section{Partizipationsformen Analysieren}
Um eine Form der politischen Partizipation zu analysieren wird betrachtet, in wie weit diese die Funktionen einer Partizipation realisiert.
\begin{description}
 \item[Artikulation] Die Möglichkeit, die eigene Meinung der allgemeinen Öffentlichkeit gegenüber zu artikulieren. Dabei ist an erster Stelle zu beachten, wie viele Personen erreicht werden und welche Macht diese haben.
 \item[Repräsentation] Dass die BürgerInnen, auch Minderheiten, angemessen vertreten werden. Dabei kann analysiert werden, ob BürgerInnen jedes Alters, Geschlechts, Region, Bildungsabschlusses, Einkommens, usw. vertreten sind.
 \item[Kontrolle] Ob die BürgerInnen die Möglichkeit haben, das Handeln der EntscheidungsträgerInnen zu kontrollieren. Ob Zugriff zu Informationen ermöglicht wird, welcher für die Kontrolle relevant ist, z.\,B. Personen zu fachlicher Kritik befähigt
 \item[Integration] Dass alle BürgerInnen sich bei dieser Partizipationsform beteiligen können. Dabei wird betrachtet, in wie weit soziale, ethnische, religiöse, etc Bevölkerungsgruppen im Prozess eingebunden werden.
\end{description} 
Dabei haben sowohl die Repräsentations- als auch die Artikulationsfunktion Einfluss auf die Kontroll- und die Integrationsfunktion.
 
Bei einer Anylse wird
\begin{enumerate}
 \item Die Partizipationsform benannt und beschrieben.
 \item Die obigen Funktionen analysiert.
 \item Benannt werden, welche der Funktionen die Form hauptsächlich erfüllt. 
\end{enumerate} 
\end{document}