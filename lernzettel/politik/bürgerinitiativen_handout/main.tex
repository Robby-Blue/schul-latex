\documentclass{article}
\usepackage{csquotes}
\usepackage{xcolor}
\usepackage{soulutf8}
\usepackage[a4paper]{geometry}
\usepackage{fancyhdr}
\pagestyle{fancy}
\usepackage[ngerman]{babel} 
\lhead{Bürgerinitiativen}
\rhead{April 2025}
\begin{document} 
 
\newcommand{\fsection}[2]{%
  \section*{%
    \makebox[\linewidth][l]{{#1}\hfill \texttt{\normalsize{#2}}}
  }
}
 
\fsection{Was}{nach INFO (S. 129)} 
Bürgerinitiativen ermächtigen Bürger dazu, sich zu vereinen um zusammen basisdemokratisch als sog. \emph{one purpose organization} genau \hl{ein spezifisches}, i.\,d.\,R. \hl{lokales oder regionales}, \hl{Interesse} vertreten. \newline
Sie versuchen die öffentliche Meinung durch z.\,B. \hl{Demonstrationen} oder \hl{Unterschriftensammlungen} zu beeinflussen, um so Druck auf die staatlichen Institutionen oder Parteien oder auch andere relevante Organe, wie Planungs- und Genehmigungsinstanzen Einfluss zu nehmen.
 
\fsection{Entstehung}{nach Randtext (S. 130)}
Die Entstehung einer Bürgerinitiative kann in sieben Phasen aufgeteilt werden:
\begin{enumerate}
 \item \hl{Empfinden} einer Meinung, z.\,B. finden eines Misstandes oder widerstand gegen ein Projekt
 \item Öffentlichkeitsarbeit für diesen, in Form von Zeitungsartikeln, Flugblätter, etc.
 \item Erfolgloser Ruf nach Veränderung an die Zuständigen 
 \item \hl{Gründen} der Bürgerinitiative als organisatorischer Rahmen, suche nach Mitgliedern, Experten, etc.
 \item Erster Kontakt mit Parteien
 \item \hl{Suche} nach \hl{Kompromissmöglichkeiten}
 \item Kompromissmöglichkeiten überprüfen, ob alle damit zufrieden sind
\end{enumerate} 
  
\fsection{Wer}{nach M2 (S. 129)}
Nicht jeder ist immer in der Lage, an Bürgerinitiativen teilzunehmen; sie nehmen \hl{viel Zeit} in Anspruch. Deshalb sind diejenigen, die \hl{viel frei verfügbare Zeit} haben, bei diesen \hl{überrepräsentiert}. Somit sind auch bei dieser Partizipationsform übermaßig viele Personen, denen es bereits vergleichsweise besser geht, durch u.\,a. eine \hl{gute Bildung}, \hl{Kinderlose} oder \hl{Rentner}, aufzufinden. \newline
Kritiker meinen, dass diese Partizipationsform somit nicht bei der Integration aller helfe, sondern Gegenteilig viele Personen, denen es bereits nicht gut geht, ausschließe und somit die Schere zwischen \enquote{unten} und \enquote{oben} vergrößere.
 
\end{document}
 
 
