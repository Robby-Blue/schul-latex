\documentclass{article}
\usepackage[a4paper]{geometry}
\usepackage{fancyhdr} 
\usepackage{amsmath}
\usepackage{hyperref}
\pagestyle{fancy}
\lhead{Zusammengesetzte Funktionen}
\rhead{August 2025}
\begin{document} 
\section{Zusammengesetzte Funktionen} 
Eine \emph{Zusammengesetzte Funktion} $f$ ist eine Funktion, welche durch zwei andere Funktionen $g$ und $h$ zusammengesetzt werden kann. Dabei wird unterschieden zwischen \emph{Verknüpfungen} und \emph{Verkettungen}.
 
\subsection{Verknüpfungen}
Eine verknüpfte Funktion $f$ ist eine Funktion $f(x)$ welche aus einer Operation zwischen $g$ und $h$ besteht. $f(x)$ ist somit, je nach Kontext,
\begin{align*}
 f(x) &= g(x)+h(x) = h(x)+g(x) \quad \text{oder} \\
  &= g(x) \cdot h(x) = h(x) \cdot g(x) \quad \text{oder} \\
  &= g(x) - h(x) \quad \text{oder} \\
  &= h(x) - g(x) \quad \text{oder} \\
  &= \frac{g(x)}{h(x)} \quad \text{oder} \\
  &= \frac{h(x)}{g(x)}
\end{align*}
 
\subsubsection{Limits} 
Um das Limit zu bestimmen kann die verknüfpte Funktion wieder aufgeteilt werden
\[
 \lim_{x \to x_0} g(x)+h(x) = \lim_{x \to x_0} g(x) + \lim_{x \to x_0} h(x) 
\]
Gleiches gilt für alle andere Operationen. Dabei ist relevant, welche Funktion schneller steigt. 
 
 
\subsection{Verkettungen}
Eine Verkettete Funktion $f$ ist eine Funktion der Form $f(x) = g(h(x))$. Hier kommt auch der Name der \hyperref[Ableitungen]{Kettenregel} her.
 
\subsubsection{Limits} 
Für die Limits von verketteten Funktionen gilt 
\[  
 \lim_{x \to x_0} g(h(x)) = \lim_{x \to x_1} g(x)
 \quad \text{mit} \quad
 x_1 = \lim_{x \to x_0} h(x)  
\]
Am beispiel vom Limit zu $x \to \infty$ von $g(x)=e^x$ und $h(x) = -2x$, also $f(x)=e^{-2x}$ gilt 
\[  
 \lim_{x \to \infty} -2x = -\infty
 \quad \text{also} \quad
 \lim_{x \to \infty} g(h(x)) = \lim_{x \to -\infty} e^x = 0
\]
 
\end{document}
 
 
