\documentclass{article} 
\usepackage{amsmath}
\usepackage[a4paper]{geometry}
\usepackage{fancyhdr}
\pagestyle{fancy}
\lhead{Steckbriefaufgaben}
\rhead{April 2025}
\begin{document}
\section{Steckbriefaufgaben}
Eine Steckbriefaufgabe ist eine Aufgabe, bei welcher eine (ganzrationale) Funktion mithilfe von bestimmten gegebenen Informationen bestimmt werden soll. Die Informationen können in Form von direkten gegebenen mathematischen Gleichungen (z.B. $f(16)=1$) oder als Textaufgabe mit schriftlich gegebenen Informationen (z.B. \texttt{Der Graph hat eine Nullstelle bei x=2}), welche erst selbst in eine Gleichung umgewandelt werden muss. 
 
\subsection{Bestimmen der Funktion}
Das Bestimmen einer Funktion kann in wenige, theoretisch einfache, Schritte aufgeteilt werden. 
\begin{enumerate}
 \item Sammeln aller gegebenen Informationen und das Aufschreiben dieser als Bedingungen
 \item Aufstellen der allgemeinen Form der Funktion. Gibt es $n$ Bedingungen, gibt es auch $n$ Unbekannte. Somit hat eine Funktion ohne weiteres einen Grad von $n-1$. Funktionen mit Symmetrien haben einen anderen Grad, siehe unten. 
 \item Einsetzen der Bedingungen in die allgemeine Form und deren Ableitungen und das Aufstellen eines LGS mit diesen Gleichungen.
 \item LGS lösen, Lösungen als Parameter der Funktion verwenden um die Funktion aufzustellen.
 \item \texttt{(optional)} Überprüfen, ob die Funktion die Bedingungen erfüllt.
\end{enumerate} 
 
\subsection{Bedingungen in Textaufgaben}
Ist eine Bedingung in Form einer Textaufgabe gegeben, muss aus dieser selbst eine Gleichung aufgestellt werden. Hier ist eine unvollständige Liste häufiger Formulierungen und deren zugehörige Gleichung.
\begin{center}
\begin{tabular}{ |l|c| }
\hline
 \multicolumn{1}{|c|}{\textbf{Bedingung}} & \textbf{Gleichung} \\
\hline
 \texttt{geht durch den Punkt (x|y)} & $f(x)=y$ \\
\hline
 \texttt{hat bei x eine Nullstelle} & $f(x)=0$ \\
\hline
 \texttt{hat den y-Achsenabschnitt $\texttt{y}_\texttt{abs}$} & $f(0)=y_\mathrm{abs}$ \\
\hline
 \texttt{schneidet bei x die Funktion t} & $f(x)=t(x)$ \\
\hline 
 \texttt{hat bei x die Steigung m} & $f'(x)=m$ \\
\hline
 \texttt{hat bei x eine Extremstelle} & $f'(x)=0$ \\
\hline 
 \texttt{hat bei x eine Wendestelle} & $f''(x)=0$ \\
\hline
\end{tabular}
\end{center}
 
Nicht selten sind mehrere Bedingungen in einer zusammen gebunden.
\begin{center}
\begin{tabular}{ |l|l| }
\hline
 \multicolumn{1}{|c|}{\textbf{Bedingung}} & \multicolumn{1}{c|}{\textbf{Gleichung}} \\
\hline
 \texttt{hat bei x die Tangentenfunktion t} &
  \begin{tabular}{@{}l@{}}geht bei $x$ durch $t$ und hat \\ bei $x$ die gleiche Steigung wie $t$ \end{tabular} \\
\hline
 \texttt{hat einen Extrempunkt im Urspung} &
  \begin{tabular}{@{}l@{}}geht bei $x=0$ durch $0$ und hat \\ bei $x=0$ eine Extremstelle \end{tabular} \\
\hline
 \begin{tabular}{@{}l@{}}\texttt{hat im Punkt (x|y) eine} \\ \texttt{waagerechte Tangente} \end{tabular} &
  \begin{tabular}{@{}l@{}}geht bei $x$ durch $y$ und hat \\ bei $x$ die Steigung $0$ \end{tabular} \\  
\hline
\end{tabular}
\end{center} 
 
\subsection{Grad bei Symmetrien} 
Ist die Funktion Achsensymmetrisch, kann sie keine ungeraden Exponenten mehr haben, weshalb sie einen höheren Grad braucht, um die gleiche Anzahl an Unbekannten zu haben. Somit hat sie den Grad $2(n-1)$. \newline
Ähnliches gilt für eine Punktsymmetrische Fukntion, welche den Grad $2n-1$ hat.
\end{document}
 
 
 
