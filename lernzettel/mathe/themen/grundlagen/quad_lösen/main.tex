\documentclass{article}
\usepackage{amsmath}
\usepackage[a4paper]{geometry}
\usepackage{fancyhdr}
\pagestyle{fancy}
\lhead{Nullstellen von Quadratische Gleichung finden}
\rhead{November 2025}
\begin{document}
\section{Nullstellen von Quadratische Gleichung finden}
Die Nullstellen von einer quadratischen Gleichung (eine Gleichung der Form ${ax^2 + bx + c = 0}$ oder $x^2 + px + q = 0$) zu finden ist ohne einem Taschenrechner oftmals nicht einfach möglich. Folgend sind eine Vielzahl an Algorithmen und Formeln um diese zu bestimmen.
 
\subsection{Quadratische Ergänzung}
Die Gleichung wird umgeformt, in die Form
\[
 x^2 + px = -q 
\]
Auf beiden Seiten wird $(p \cdot 0.5)^2$ addiert und zusammengefasst. Nun kann auf beiden Seiten die Wurzel genommen und nach einem $x$ umgeformt werden.
 
\subsection{$pq$-Formel}
Ist $a=1$, oder wird die Gleichung nach $a=1$ umgestellt, so kann die $pq$-Formel genutzt werden. So gilt
\[
 x = - \frac{p}{2} \pm \sqrt{\left(\frac{p}{2}\right)^2 - q}  
\] 
\subsection{Satz von Vieta}
Für jede Nullstelle gilt 
\[
 x_1 + x_2 = -p
 \qquad \text{und} \qquad
 x_1 \cdot x_2 = q 
\]
Alle Zahlenpaare, welche zu $q$ multiplizieren werden gefunden bevor untersucht wird, welche davon zu $-p$ addieren.
 
\subsection{Mitternachtsformel} 
Es gilt die Formel
\[
 x = \frac{-b \pm \sqrt{b^2-4ac}}{2a} 
\]
Beim Nachdenken wird offensichtlich, dass mithilfe von $\sqrt{b^2-4ac}$ bestimmt werden kann, wie viele Nullstellen es gibt. Ist es $<0$, so gibt es keine, ist es $=0$ gibt es eine, ist es $>0$ gibt es zwei. 
 
\end{document}