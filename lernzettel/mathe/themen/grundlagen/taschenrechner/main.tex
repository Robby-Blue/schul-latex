\documentclass{article}
\usepackage{amssymb}
\usepackage{hyperref}
\usepackage[a4paper]{geometry}
\usepackage{fancyhdr}
\usepackage{tikz} 
\pagestyle{fancy}
\lhead{Umgang mit dem Taschenrechner}
\rhead{Juli - September 2025}
\begin{document}
  
\newcommand{\return}[0]{\,%
 \begin{tikzpicture}
  \draw (0,0) -- (0,-1ex);
  \draw (0,-1ex) -- (-1ex,-1ex);
  \draw[fill] (-1.3ex,-1ex) -- (-1ex,-1.3ex) -- (-1ex,-0.7ex) -- cycle; 
 \end{tikzpicture}\,%
} 
 
\newcommand{\symbols}[0]{\,%
 \raisebox{-1.3ex}{%
  \begin{tikzpicture}[scale=1.2]
   \draw (0,0ex) -- (0,1ex);
   \draw (1ex,0ex) -- (1ex,1ex);
   \drawsquare{0.3ex}{0.3ex};
   \node at (1.7ex, 0.5ex) {$\{$};
   \drawsquare{2.2ex}{0.8ex};
   \drawsquare{2.2ex}{-0.2ex};
  \end{tikzpicture}%
 }
}
 
\newcommand{\drawsquare}[2]{
 \draw (#1,#2) -- ++(0.4ex, 0);
 \draw (#1,#2) -- ++(0, 0.4ex);
 \draw (#1,#2) ++ (0, 0.4ex) -- ++(0.4ex, 0);
 \draw (#1,#2) ++ (0.4ex, 0) -- ++(0, 0.4ex);
}
  
\newcommand{\vect}[1]{\overrightarrow{#1}}
\newcommand{\calckey}[1]{\texttt{#1}}  
\newcommand{\arrow}[0]{\textrightarrow{ }}
 
\section{Umgang mit dem Taschenrechner}
\begin{description}
 \item[LGS Lösen] Ein LGS kann entweder als Gleichungssystem mit \calckey{menu} \arrow \calckey{3} \arrow \calckey{7} \arrow \calckey{1 oder 2} oder als Matrix $m$ mit \calckey{rref(m)} gelöst werden. Mehr dazu im Kapitel \hyperref[Lineare Gleichungssysteme]{Lineare Gleichungssysteme}.
 \item[Matrix erstellen] Ein Matrix mit einer bestimmten Größe kann mit \calckey{menu} \arrow \calckey{7} \arrow \calckey{1} \arrow \calckey{1} erstellt werden. Schneller kann ein $1 \times 1$ Vektor mit \calckey{ctrl} + \calckey{(} erstellt werden.
 \item[Vektor erstellen] Ein Vektor ist im CAS eine Matrix mit nur einer Spalte oder nur einer Zeile.  
 \item[Matrix vergrößern] Ein Zeile kann mit \calckey{\return} hinzugefügt werden. Eine Spalte kann mit \calckey{shift} + \calckey{\return} hinzugefügt werden.
 \item[Skalarprodukt] Um das Skalarprodukt $\vect{a} \cdot \vect{b}$ auszurechnen wird \calckey{dotp(a,b)} (\calckey{menu} \arrow \calckey{7} \arrow \calckey{C} \arrow \calckey{3}) genutzt. Der Name kommt vom Englischen \emph{dot product}, weil das Mal-Zeichen ein Punkt ist.
 \item[Vektorenprodukt] Um das Vektorenprodukt $\vect{a} \times \vect{b}$ auszurechnen wird \calckey{crossp(a,b)} (\calckey{menu} \arrow \calckey{7} \arrow \calckey{C} \arrow \calckey{2}) genutzt. Der Name kommt vom anderen Namen des Vektorenproduktes, das \emph{Kreuzprodukt}, oder auch dem englischen \emph{cross product}, weil das Mal-Zeichen ein Kreuz ist.  
 \item[Betrag eines Vektors] Für den Betrag eines Vektors $\vect{v}$ wird \calckey{norm(v)} (\calckey{menu} \arrow \calckey{7} \arrow \calckey{7} \arrow \calckey{1}) genutzt. Der Betrag eines Vektors darf, obwohl wir beides als Beträge darstellen, nicht mit dem Betrag einer normalen Zahl verwechselt werden.
 \item[Betrag einer Zahl] Für den Betrag einer Zahl $n$ werden die Betragstriche in \symbols oder \calckey{abs(n)} genutzt. 
 \item[Binomialkoeffizient] Ein \hyperref[Binomialkoeffizient]{Binomialkoeffizient} kann \texttt{nCr(n,k)}, beziehungsweise \calckey{menu} \arrow \calckey{5} \arrow \calckey{3}
 \item[Binomialverteilung] Die Wahrscheinlichkeit, dass es bei einer \hyperref[Binomialverteilungen]{Binomialverteilung} zu genau $k$ Treffern kommt, kann im Taschenrechner durch \texttt{binomPdf(n,p,k)} beziehungsweise mit \calckey{menu} \arrow \calckey{5} \arrow \calckey{5} \arrow \calckey{A} gefunden werden. Dabei ist \texttt{k} optional; wird es weggelassen, wird $P(X=k)$ für alle $k$ in $0 \leq k \leq n$ in einer Liste ausgegeben. Die Ausgabe erfolgt ist als Dezimalzahl.
 \item[Kumulative Wahrscheinlichkeiten] Eine \hyperref[Kumulative Wahrscheinlichkeiten]{Kumulative Wahrscheinlichkeit} kann gefunden werden, indem das \texttt{Pdf} einer Wahrscheinlichkeitsverteilungsfunktion durch ein \texttt{Cdf} ersetzt wird und anstelle von einem \texttt{k} ein \texttt{a} und \texttt{b} angegeben wird. Im \texttt{menu} sind diese Fuktionen eine Option weiter unten aufzufinden.
 \item[Fakultät] Für das berechnen der Fakultät einer Zahl kann das Ausrufezeichen im Menu der \texttt{?!$\blacktriangleright$}-Taste oder das Ausrufezeichen von \texttt{menu}, \texttt{5}, \texttt{1} genutzt werden.  
\end{description} 
 
\subsection{Lists und Spreadsheets}
\begin{description}
 \item[Variablen] Eine Variable kann definiert werden, indem, wie normalerweise auch, das \texttt{:=} Syntax in einem Kästchen genutzt wird. 
 \item[Datensequenz] Eine Spalte kann mit einer Sequenz an Daten gefüllt werden, indem das Kästchen der Spaltenformel ausgewählt, aber nicht beschrieben, wird alle relevantes in \calckey{menu} \arrow \calckey{3} \arrow \calckey{1} eingetippt werden kann. Dabei können auch bereits definierte Variablen genutzt werden.
 \item[Binomialverteilung] Die Binomialverteilung kann innerhalb einer Spreadsheet, so wie normalerweise auch, mit \texttt{binomPdf(n,p,k)} berechnet werden. Hier ist dies im Menu nur unter \calckey{menu} \arrow \calckey{4} \arrow \calckey{1} \arrow \calckey{A} zu erreichen.
 \item[Histogramme] Ein Histogramm kann mit \calckey{menu} \arrow \calckey{3} \arrow \calckey{8} geöffnet werden.
\end{description}  
\end{document}