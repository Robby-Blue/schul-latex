\documentclass{article}
\usepackage[a4paper]{geometry}
\usepackage{fancyhdr}
\pagestyle{fancy}
\lhead{Zufallsgrößen}
\rhead{August - Oktober 2025}
\begin{document}
\section{Zufallsgrößen} 
Eine \emph{Zufallsgröße} $X$ ist eine Zuordnung, welche jedem Ergebnis eines zufälligen Ereignisses ein zugehöriges $x$ zuordnet, so dass die Ereignisse sozusagen durchnummeriert werden. Die dazugehörige \emph{Wahrscheinlichkeitsverteilung}, $P(X=x)$, ordnet nun jedem $x$ eine Wahrscheinlichkeit zu. \newline
Eine Zufallsgröße als Balkendiagramm aufgezeichnet, mit $x_e$ auf der x-Achse und $P(X=x_e)$ auf der y-Achse, ist ein \emph{Histogramm}.
% TODO: table für Zufallsgrößen und tikz mit Histogramm
 
\subsection{Erwartungswert}
Der \emph{Erwartungswert} einer Zufallsgröße, $E(X)$ oder $\mu$, ist das durchschnittliche Erwartete ergebnis. Um diesen zu berechnet werden alle möglichen Ergebnisse gewichtet ($x \cdot P(X=x)$) zusammen addiert. Es gilt also
\[
 \mu = E(X) = \sum_{i=1}^n x_i \cdot P(X=x_i) 
\]  
Ein Spiel ist fair, wenn niemand langzeitig Profitiert, wenn $\mu$ genau so groß ist wie der benötigt Einsatz, wie der Preis, des Spiels.
 
\subsection{Standardabweichung}
Eine \emph{Standardabweichung} $\sigma$ beschreibt die Streuung der Wahrscheinlichkeitsverteilung einer Zufallsgröße $X$ im vergleich zu $\mu$. Es gilt
\[
 \sigma = \sqrt{\sum_{i=1}^n (x_i - \mu)^2 \cdot P(X=x_i)}
\]
Neben der Standardabweichung gibt es noch die Varianz $V$, mit
\[ 
 V = \sigma^2 
\] 
\end{document}