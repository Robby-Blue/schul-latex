\documentclass{article}
\usepackage{hyperref}
\usepackage{amssymb}
\usepackage[a4paper]{geometry}
\usepackage{fancyhdr}
\pagestyle{fancy}
\lhead{Zufallsgrößen}
\rhead{August - Oktober 2025}
\begin{document}
\section{Zufallsgrößen} 
Eine \emph{Zufallsgröße} $X$ ist eine Zuordnung, welche jedem Ergebnis eines zufälligen Ereignisses ein zugehöriges $x$ zuordnet, so dass die Ereignisse sozusagen durchnummeriert werden. Die dazugehörige \emph{Wahrscheinlichkeitsverteilung}, $P(X=x)$, ordnet nun jedem $x$ eine Wahrscheinlichkeit zu. \newline
Eine Zufallsgröße als Balkendiagramm aufgezeichnet, mit $x_e$ auf der x-Achse und $P(X=x_e)$ auf der y-Achse, ist ein \emph{Histogramm}.
% TODO: table für Zufallsgrößen und tikz mit Histogramm
 
\subsection{Diskret und Stetig} 
Zufallsgrößen können entweder \emph{diskret} oder \emph{stetig} sein.
 
Eine \hyperref[Diskrete Zufallsgröße]{diskrete Zufallsgröße} wird dadurch definiert, dass sie nur für $x \in \mathbb{N}$ gilt, also nur für ganze Zahlen. Sie kann durch eine \hyperref[Binomialverteilungen]{Binomialverteilung} beschrieben. 
 
Eine \hyperref[Stetige Zufallsgröße]{stetige Zufallsgröße} wird dadurch definiert, dass sie für $x \in \mathbb{R}$ gilt, also für alle reelen Zahlen. Sie kann durch eine \hyperref[Normalverteilungen]{Normalverteilung} beschrieben werden.
 
\subsection{Kenngrößen}
Der \emph{Erwartungswert} einer Zufallsgröße, $E(X)$ oder $\mu$, ist das durchschnittliche erwartete Ergebnis. Ein Spiel ist fair, wenn niemand langzeitig Profitiert, wenn $\mu$ genau so groß ist wie der benötigt Einsatz, wie der Preis, des Spiels.
 
Eine \emph{Standardabweichung} $\sigma$ beschreibt die Streuung der Wahrscheinlichkeitsverteilung einer Zufallsgröße $X$ im vergleich zu $\mu$.
Neben der Standardabweichung gibt es noch die Varianz $V$, mit
\[ 
 V = \sigma^2 
\]
 
Wie diese beiden Kenngrößen berechnet werden, hängt davon ab, ob die vorliegende Zufallsgröße disrekt oder stetig ist. 
\end{document}