\documentclass{article}
\usepackage[a4paper]{geometry}
\usepackage{fancyhdr}
\pagestyle{fancy}
\lhead{Zufallsgrößen}
\rhead{August 2025}
\begin{document}
\section{Zufallsgrößen} 
Eine \emph{Zufallsgröße} $X$ ist eine Zuordnung, welche jedem Ergebnis eines zufälligen Ereignisses ein zugehöriges $x$ zuordnet, so dass die Ereignisse sozusagen durchnummeriert werden. Die dazugehörige \emph{Wahrscheinlichkeitsverteilung}, $P(X=x)$, ordnet nun jedem $x$ eine Wahrscheinlichkeit zu. \newline
Eine Zufallsgröße als Balkendiagramm aufgezeichnet, mit $x_e$ auf der x-Achse und $P(X=x_e)$ auf der y-Achse, ist ein \emph{Histogramm}.
% TODO: table für Zufallsgrößen und tikz mit Histogramm
\end{document}