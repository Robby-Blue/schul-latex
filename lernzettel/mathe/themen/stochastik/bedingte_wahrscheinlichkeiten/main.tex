\documentclass{article} 
\usepackage{amsmath} 
\usepackage[a4paper]{geometry}
\usepackage{fancyhdr}
\pagestyle{fancy}
\lhead{Bedingte Wahrscheinlichkeiten}
\rhead{August 2025}
\usepackage{tikz} 
\begin{document} 
\section{Bedingte Wahrscheinlichkeiten}
Eine \emph{bedingte Wahrscheinlichkeit} beschreibt die Wahrscheinlichkeit eines Ereignisses unter der Bedingung, dass ein anderes Eregniss bereits eingetroffen ist. Diese kann als $P(A|B)$, oder wie im Schulbuch als $P_B(A)$, aufgeschrieben werden. Dann gilt die Wahrscheinlichkeit 
\[ 
 P_B(A) = P(A|B) = \frac{P(A \cap B)}{P(B)}
\] 
\end{document}