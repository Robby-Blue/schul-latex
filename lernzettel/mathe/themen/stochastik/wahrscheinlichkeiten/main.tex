\documentclass{article} 
\usepackage{hyperref}
\usepackage[a4paper]{geometry}
\usepackage{fancyhdr}
\pagestyle{fancy}
\lhead{Wahrscheinlichkeiten}
\rhead{August 2025}
\begin{document}
 
\section{Wahrscheinlichkeiten} 
Die Wahrscheinichkeit dass ein Ergeignisses $E$ eintritt ist
\[
 P(E) 
\]
als Dezimalzahl zwischen $0$ und $1$ oder $0\%$ und $100\%$. Dabei kann das $E$ direkt in die Klammern geschrieben werden, oder vorher definiert werden und dann als Variabble $E$ aufgeschrieben werden. Ein $E$ kann definiert werden als
\[
 E : \texttt{Beschreibung des Ereigniss} 
\]
Die Wahrscheinlichkeit, dass entweder $E_1$ oder $E_2$ geschieht, ist $P(E_1) + P(E_2)$. Dass zwei einander bedingte Ereignisse hintereinander passieren wird durch ein $\cap$ veranschaulicht. Es gilt $P(E_1 \cap E_2) = P(E_1) \cdot P(E_2)$, solange beide Ereignisse voneinander \hyperref[Unabhängigkeiten]{Unabhängig} din.
 
\subsection{Gegenwahrscheinlichkeit}
Die Wahrscheinlichkeit, dass $E$ nicht eintritt ist immer
\[
 1 - P(E) 
\] 
 
\subsection{Bedingte Wahrscheinlichkeiten}
Eine \emph{bedingte Wahrscheinlichkeit} beschreibt die Wahrscheinlichkeit eines Ereignisses unter der Bedingung, dass ein anderes Eregniss bereits eingetroffen ist. Diese kann als $P(A|B)$, oder wie im Schulbuch als $P_B(A)$, aufgeschrieben werden. Dann gilt die Wahrscheinlichkeit 
\[ 
 P_B(A) = P(A|B) = \frac{P(A \cap B)}{P(B)}
\] 
Zwei Ereignisse $A$ und $B$ sind voneinander Stochastisch unabhängig, wenn $P_B(A)=P(A)$ oder $P(A \cap B) = P(A) \cdot P(B)$, die \emph{Multiplikationsregel}, gilt. Bei Vierfeldertafeln ist die Multiplikationsregel um einiges einfacher anzuwenden. 
\end{document}