\documentclass{article} 
\usepackage[a4paper]{geometry}
\usepackage{fancyhdr}
\pagestyle{fancy}
\lhead{Wahrscheinlichkeiten}
\rhead{August 2025}
\begin{document}
 
\section{Wahrscheinlichkeiten} 
Die Wahrscheinichkeit dass ein Ergeignisses $E$ eintritt ist
\[
 P(E) 
\]
als Dezimalzahl zwischen $0$ und $1$ oder $0\%$ und $100\%$. Dabei kann das $E$ direkt in die Klammern geschrieben werden, oder vorher definiert werden und dann als Variabble $E$ aufgeschrieben werden. Ein $E$ kann definiert werden als
\[
 E : \texttt{Beschreibung des Ereigniss} 
\]
Die Wahrscheinlichkeit, dass entweder $E_1$ oder $E_2$ geschieht, ist $P(E_1) + P(E_2)$. Dass zwei einander bedingte Ereignisse hintereinander passieren, ist das Produkt der beiden einander unabhängigen Ereignisse.
 
\subsection{Gegenwahrscheinlichkeit}
Die Wahrscheinlichkeit, dass $E$ nicht eintritt ist immer
\[
 1 - P(E) 
\] 
 
\end{document}