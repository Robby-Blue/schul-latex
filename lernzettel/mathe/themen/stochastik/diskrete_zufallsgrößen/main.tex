\documentclass{article}
\usepackage[a4paper]{geometry}
\usepackage{fancyhdr}
\pagestyle{fancy}
\lhead{Diskrete Zufallsgröße}
\rhead{Oktober 2025}
\begin{document}
 
\section{Diskrete Zufallsgröße}
\subsection{Erwartungswert}
Um den Erwartungswert einer diskreten Zufallsgröße zu berechnen werden alle möglichen Ergebnisse gewichtet ($x \cdot P(X=x)$) zusammen addiert. Es gilt also
\[
 \mu = E(X) = \sum_{i=1}^n x_i \cdot P(X=x_i) 
\]
 
\subsection{Standardabweichung}
Für die Standardabweichung einer diskreten Zufallsgröße gilt immer
\[
 \sigma = \sqrt{\sum_{i=1}^n (x_i - \mu)^2 \cdot P(X=x_i)}
\]
Also folgt die Varianz
\[ 
 V = \sigma^2 = \sum_{i=1}^n (x_i - \mu)^2 \cdot P(X=x_i)
\]  
 
\end{document}