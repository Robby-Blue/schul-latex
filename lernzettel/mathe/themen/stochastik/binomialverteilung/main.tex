\documentclass{article}
\usepackage{hyperref}
\usepackage{amsmath} 
\usepackage{amssymb}
\usepackage[a4paper]{geometry}
\usepackage{fancyhdr}
\pagestyle{fancy}
\lhead{Binomialverteilungen}
\rhead{September 2025}
\begin{document}
\section{Binomialverteilungen}
Eine Binomialverteilung ist die Wahrscheinlichkeitsverteilung einer Bernoulli-Kette. Für eine Binomialverteilung gibt es eine Funktion $P(X=k)$, welche jeweils angibt, wie wahrscheinlich es ist, dass es bei einem Bernoulli Experiment mit $n$ Wiederholungen zu genau $k$ treffern kommt. Im Taschenrechner gilt 
\[
 P(X=k)=\texttt{binomPdf(n,p,k)}
\]
Anstelle vom Ausschreiben von \texttt{binomPdf} kann auch \texttt{menu}, \texttt{5}, \texttt{5}, \texttt{A} genutzt werden. Dabei ist \texttt{k} optional; wird es weggelassen, wird $P(X=k)$ für alle rellen Zahlen $k$ mit $0 \leq k \leq$ in einer Liste ausgegeben.
 
Soll eine \hyperref[Kumulative Wahrscheinlichkeiten]{kumulative Wahrscheinlichkeit} berechnet werden, wird im Taschenrechner \texttt{binomCdf(n,p,a,b)} genutzt, welches Menu unter \texttt{menu}, \texttt{5}, \texttt{5}, \texttt{B} verfügbar ist.
 
\subsection{Bernoulli Ketten}
Eine Bernoulli Kette beschreibt ein \emph{Bernoulli Experiment}, ein Experiment, welches
\begin{enumerate}
 \item Zwei mögliche Ergebnisse hat
 \item Eine gleichbleichebende Wahrscheinlichkeit hat
 \item In seinen Einzelexperimenten unabhängig ist
 \item Eine feste Anzahl an Wiederholungen, $n$, hat 
\end{enumerate} 
Die Wahrscheinlichkeit einer Bernoulli-Kette ist
\[
 P(X=k) =
 \begin{pmatrix} n \\ k \end{pmatrix} \cdot
 p^k \cdot (1-p)^{n-k}
\]
Diese lässt sich so erklären, dass es $\begin{pmatrix} n \\ k \end{pmatrix}$ Möglichkeiten gibt, dass es zu genau $k$ Treffern kommt, wobei, wenn sich ein Baumdiagramm vorgestellt wird, jeder Weg zu treffen eine Wahrscheinlichkeit von $p^k \cdot (1-p)^{n-k}$ hat.
 
\subsection{Erwartungswert und Standardabweichung}
Bei einer Binomialverteilung gilt
\[
 \mu = n \cdot p 
 \qquad \text{und} \qquad
 \sigma = \sqrt{n \cdot p \cdot (1-p)} 
\]
\end{document}