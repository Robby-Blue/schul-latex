\documentclass{article}
\usepackage{amsmath}
\usepackage{hyperref}
\usepackage[a4paper]{geometry}
\usepackage{fancyhdr}
\pagestyle{fancy}
\lhead{Konfidenzintervalle}
\rhead{Oktober 2025}
\begin{document}
\section{Konfidenzintervalle}
Ist das $p$ eines Gesamten nicht bekannt, so kann mit der relativen Häufigkeit $h$ einer Stichprobe ein \emph{Konfidenzintervall} aufgestellt werden, in welches mit einer Wahrscheinlichkeit von $95\%$ (oder einem anderen \emph{Sicherheitsfaktor}) das $p$ des Gesamten einhält.
 
\subsection{Algebraisch}
Wird ein \hyperref[Prognosen]{$95\%$-Prognoseintervall} aufgestellt und durch $n$ geteilt, folgt 
\[
 h = p \pm 1,96 \sqrt{\frac{p \cdot (1-p)}{n}}
\]
Das $h$ folgt aus $\dfrac{k}{n}$, beschreibt also die relative Häufigkeit. Aus einer Stichprobe kann nun ein $h$ gefunden werden, welches in der obigen Gleichung nach $p$ aufgelöst werden kann, welche als Grenzen des Intervalls dienen.
 
Zu einer Wahrscheinlichkeit von $95\%$ kann nun gesagt werden, dass dieses Intervall das echte $p$ beinhaltet.
\end{document}