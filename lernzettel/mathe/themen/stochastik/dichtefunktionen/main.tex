\documentclass{article}
\usepackage{amsmath} 
\usepackage{amssymb}
\usepackage[a4paper]{geometry}
\usepackage{fancyhdr}
\pagestyle{fancy}
\lhead{Dichtefunktionen}
\rhead{Oktober 2025}
\begin{document}
\section{Dichtefunktionen}
Eine Dichtefuntkion ist eine Funktion, welche die Dichte der Wahrscheinlichkeit einer stetigen Zufallsgröße angibt. Somit gibt das Integral zwischen zwei Stellen eine kumulative Wahrscheinlichkeit an, es git also immer.
\[
 P(a \le x \le b) = \int_a^b f(x)\,\text{d}x 
\] 
Damit eine Funktion $f$ eine Dichtefunktion sein kann, muss die Gesamtwahrscheinlichkeit $100\%$, also die Fläche unter Kurve $1$, sein und es kann zu keinem Punkt eine negative Wahrscheinlichkeit existieren. Mathematisch ausgedrückt gilt also 
\[
 f(x) > 0 \text{ für } x \in \mathbb{R}
 \qquad \text{und} \qquad  
 \int_\infty^\infty f(x)\,\text{d}x = 1 
\]  
\end{document}