\documentclass{article}
\usepackage[a4paper]{geometry}
\usepackage{fancyhdr}
\pagestyle{fancy}
\lhead{Prognosen}
\rhead{Oktober 2025}
\begin{document}
 
\section{Prognosen} 
\begin{minipage}{\dimexpr\linewidth-6cm}
Wurde ein $\mu$ und ein $\sigma$ gefunden, so können Prognosen darüber aufgestellt werden, wie wahrscheinlich es ist, dass eine Stichprobe in einem bestimmten Prognoseintervall, in einem bestimmten Radius um $\mu$, liegt. Dieser Radius wird als ein vielfaches von $\sigma$ angegeben.
\[
 P(\mu-\sigma \cdot k \le X \le \mu+\sigma \cdot k) 
\]
Bei diskreten Verteilungen, wie der Binomialverteilung, muss das $\sigma \cdot k$ abgerundet werden, weil die Intervallenden von disrekten Verteilungen ganze Zahlen sein müssen. 
\end{minipage}
\hfill 
\begin{minipage}{6cm}
\begin{center}
\begin{tabular}{ |c|c| }
\hline
 $\mathbf{k}$ & \textbf{Wahrscheinlichkeit} \\
\hline
 1 & 0.68 \\
\hline
 1.64 & 0.9 \\
\hline
 1.96 & 0.95 \\
\hline
 2 & 0.955 \\
\hline
 2.58 & 0.99 \\
\hline
 3 & 0.997 \\
\hline
\end{tabular}
\end{center}
\end{minipage} 
 
\vspace*{0.3em} 
\noindent Hier gilt die $\sigma$-Regel: die Wahrscheinlichkeit, dass das Ergebnis in diesem Intervall liegt hängt von dem gegebenen $k$ ab. Bei Normalverteilten Verteilungen gelt ungefähr die Werte der obigen Tabelle, falls $\sigma > 3$.
 
\subsection{Verträglichkeit} 
Eine Stichprobe ist mit einem $p$ Verträglich, wenn die Stichprobe dann im $95\%$-Prognoseintervall liegt. Liegt sie außerhalb von diesem, liegt eine \emph{signifikante} Abweichunge vor. Liegt eine Stichprobe sogar außerhalb des $99\%$-Intervalls, ist diese eine \emph{hochsignifikante} Abweichung.
 
\end{document}