\documentclass{article}   
\usepackage{amsmath} 
\usepackage[a4paper]{geometry}
\usepackage{fancyhdr}
\pagestyle{fancy}
\lhead{Vierfeldertafel}
\rhead{August 2025}
\usepackage{tikz} 
\begin{document} 
\section{Vierfeldertafel}
\begin{minipage}{\dimexpr\linewidth-8cm} 
 Eine Vierfeldertafel ist eine art Tabelle, welche jeweils in ihren Zeilen und Spalten die wahrscheinlichkeit eines bestimmtes Ereignisses, dessen Gegenwahrscheinlichkeit und die Wahrscheinlichkeit dass eine Kombination beider Ereginisse darstellt.
 
\end{minipage}
\hfill
\begin{minipage}{8cm}
 \center
 \begin{tabular}{ |c|c|c|c| }
  \hline
        & $A_1$ & $A_2$ & $\Sigma$ \\
  \hline
  $B_1$ & $P(A_1 \cap B_1)$ & $P(A_2 \cap B_1)$ & $P(B_1)$ \\
  \hline
  $B_2$ & $P(A_1 \cap B_2)$ & $P(A_2 \cap B_2)$ & $P(B_2)$ \\
  \hline
  $\Sigma$ & $P(A_1)$ & $P(A_2)$ & 1 \\
  \hline
 \end{tabular}
\end{minipage}
In der obersten Reihe und linkesten Spalte ist jeweils der Name des Ereignisses und dessen Gegenerigniss aufgelistet. In der untersten Reihe und der rechtesten Spalte sind die Wahrscheinlichkeiten der dazugehörigen Ereignisse aufgelistet. Diese repräsentieren die Summen der einzelnen Teilwahrschdeinlichkeiten, so dass ganz unten rechts die Wahrscheinlichkeit steht, dass irgendetwas passiert, also immer $1$. \newline
In der Mitte stehen die jeweiligen Teilwahrscheinlichkeiten, dass eine Kombination der jeweiligen oberen und linkeren Ereigniss stattfindet. 
\end{document}