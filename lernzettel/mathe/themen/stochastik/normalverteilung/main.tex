\documentclass{article}
\usepackage{amssymb}
\usepackage{amsmath}
\usepackage[a4paper]{geometry}
\usepackage{fancyhdr}
\pagestyle{fancy}
\lhead{Normalverteilung}
\rhead{Dezember 2025}
\begin{document}
\section{Normalverteilung} 
Eine Normalverteilung ist eine Dichtefunktion. Sie folgt der Form
\[
 \varphi(x) = \frac{1}{\sigma \sqrt{2\pi}} \cdot \text{e}^{-\dfrac{(x-\mu)^2}{2\sigma^2}}
\]
Somit ist sie zur Vertikalen $x=\mu$ Symmetrisch. Bei $x=\mu \pm \sigma$ sind Wendestellen.
 
\subsection{Verteilungsfunktion einer Normalverteilung} 
Die Verteilungsfunktion gibt die kumulative Wahrscheinlichkeit einer Normalfunktion von $-\infty$ bis $x$ an, so dass
\[
 \Phi(x) = P(X \leq x) = \int_{-\infty}^x \varphi(t) \,\text{d}t 
\] 
 
\subsection{Inverse Normalfunktion}
Die Inverse Normalfunktion agiert als das Gegenteil einer Verteilungsfunktion. Sie kann genutzt werden, um die Stelle zu finden, an welcher die kumulative Wahrscheinlichkeit einen bestimmeten Wert erreicht. 
\[
 \Phi(x) = A
 \Longleftrightarrow 
 \texttt{invNorm(A, $\mu$, $\sigma$)} = x
\] 
\texttt{invNorm} ist unter \texttt{menu}, \texttt{5}, \texttt{5}, \texttt{3} auffindbar.
\end{document}