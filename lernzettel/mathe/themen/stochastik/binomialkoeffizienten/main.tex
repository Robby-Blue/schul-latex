\documentclass{article}
\usepackage{amsmath}
\usepackage[a4paper]{geometry}
\usepackage{fancyhdr}
\pagestyle{fancy}
\lhead{Binomialkoeffizienten}
\rhead{September 2025}
\begin{document}
 
\newcommand{\bink}[2]{
\begin{pmatrix}
 #1 \\
 #2 \\
\end{pmatrix} 
} 
 
\section{Binomialkoeffizienten}  
Ein Binomialkoeffizient $n$ über $k$, aufgeschrieben als
\[ 
\bink{n}{k} 
= \frac{n!}{k! \cdot (n-k)!} 
\]
kann genutzt werden, um zu bestimmen, wie viele Möglichkeiten es gibt, von insgesamt $n$ Element nur $k$ Elemente auszuwählen, wobei ihre Reihenfolge irrelevant ist. Die obige Fomel kann gekürzt werden zu,
\[
 \frac{n \cdot \ldots \cdot (n-k+1)}{k!}
\]  
Als weitere Tricks gelten immer
\[
\bink{n}{k} = \bink{n}{n-k}
\quad \text{und} \quad
\bink{n}{1} = n 
\]
Ansonsten gilt im Taschenrechner \texttt{nCr(n,k)}, beziehungsweise \texttt{menu}, \texttt{5}, \texttt{3} 
 
\subsection{Fakultät}
Die Fakultät einer natürlichen Zahl $n$, genannt $n$ Fakultät, aufgeschrieben als $n!$, ist immer $n \cdot (n-1) \cdot (n-2) \cdot \ldots \cdot 1$. Aus einfachem wegstreichen folgt dass
\[
 \frac{n!}{(n-1)!} = n
\] 
\end{document}