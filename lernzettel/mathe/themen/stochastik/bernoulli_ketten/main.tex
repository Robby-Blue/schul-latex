\documentclass{article}
\usepackage{amssymb}
\usepackage{amsmath}
\usepackage[a4paper]{geometry}
\usepackage{fancyhdr}
\pagestyle{fancy}
\lhead{Bernoulli Ketten}
\rhead{September 2025}
\begin{document}
\section{Bernoulli Ketten}
Wine Bernoulli Kette beschreibt ein \emph{Bernoulli Experiment}, ein Experiment, welches
\begin{enumerate}
 \item Zwei mögliche Ergebnisse hat
 \item Eine gleichbleichebende Wahrscheinlichkeit hat
 \item In seinen Einzelexperimenten unabhängig ist
 \item Eine feste Anzahl an Wiederholungen, $n$, hat 
\end{enumerate} 
Die Wahrscheinlichkeit einer Bernoulli-Kette ist
\[
 P(X=k) =
 \begin{pmatrix} n \\ k \end{pmatrix} \cdot
  p^k \cdot (1-p)^{n-k}
\]
Im Taschenrechner gilt \texttt{binomPdf(n,p,k)} beziehungsweise mit \texttt{menu}, \texttt{5}, \texttt{5}, \texttt{A}. Dabei ist \texttt{k} optional; wird es weggelassen, wird $P(X=k)$ für alle $\{k \in \mathbb{N}_0 \mid 0 \leq k \leq n\}$ in einer Liste ausgegeben
\end{document}