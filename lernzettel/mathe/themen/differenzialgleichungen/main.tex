\documentclass{article}
\usepackage{amsmath}
\usepackage[a4paper]{geometry}
\usepackage{fancyhdr}
\pagestyle{fancy}
\lhead{Differenzialgleichungen}
\rhead{März 2025}
\begin{document}
\section{Definition} 
Differenzialgleichungen (DGLs) sind Gleichungen, welche sowohl eine Funktion als auch dessen Ableitung beinhaltet.
\[
 f'(x) = ... + ... \cdot f(x)
\]
Die Lösung einer Differenzialgleichung ist eine Funktion selbst. DGLs beschreiben Systeme, bei welchen die momentane Änderungsrate von dem momentanen Bestand abhängig ist.
 

\section{Aufstellen}
\subsection{Sachaufgaben} 
DGLs können für Sachkontexte aufgestellt werden, indem basierend auf der Beschreibung der Zusammenhang zwischen dem Bestand und der Änderungsrate dargestellt wird. \newline
Ist beispielsweise in einer Aufgabe die Änderungsrate eines Bestandes als prozentuale Steigung von $p$ pro Zeiteinheit angegeben, kann $f'(x) = f(x) \cdot p$ beschrieben werden. Weitere Faktoren, welche die Änderungsrate beeinflussen, wie beispielsweise eine konstante absolute Zunahme von $m$, können einfach hinzugefügt werden, zu $f'(x)=m + f(x) \cdot p$.
 
\subsection{Mit $f(x)$} 
Ist $f(x)$ gegeben und kommt in ihrer eigenen Ableitung $f'(x)$ vor, kann sie dort durch die Funktion ersetzt werden. Dies ist insbesondere bei Funktionen, welche in ihrer Ableitung vorkommen, wie der e-Funktion, hilfreich. \newline
Ist beispielsweise $f(x)=\mathbf{e^{5x}}$, so ist $f'(x)=5 \cdot \mathbf{e^{5x}}$. Wird dieses $e^{5x}$ nun erneut durch das $f(x)$ ersetzt, folgt $f'(x)=5 \cdot f(x)$ als DGL.
 
\section{Lösungen} 
Eine Funktion löst ein DGL, wenn sie die Gleichung für alle $x$ erfüllt. 
Ob eine gegebene Funktion dies tut, kann überprüft werden, indem sie und ihre Ableitungen in die DGL eingesetzt werden, die Gleichung vereinfacht wird und überprüft wird, ob sie weiterhin erfüllt ist.
\end{document}
 
