\documentclass{article}
\usepackage[a4paper]{geometry}
\usepackage{fancyhdr}
\pagestyle{fancy}
\lhead{Parametergleichungen}
\rhead{März 2025}
\begin{document}
 
\newcommand{\vect}[1]{\overrightarrow{#1}} 
 
\section{Parametergleichungen}
Eine Parametergleichung ist eine Gleichung, welche mithilfe eines Parameter verschiedene Punkte, wie beispielsweise alle Punkte, welche auf einer bestimmten Gerade liegen, darstellt. \newline
Die Schreibweise der Parametergleichung lautet 
\[
 \texttt{name}: \vect{x} = \texttt{gleichung}
\] 
Der Name kommt dabei auf die Parametergleichung an sich an, normalerweise ein $g$ für Geraden und ein $E$ für Ebenen. 
\end{document}
 
