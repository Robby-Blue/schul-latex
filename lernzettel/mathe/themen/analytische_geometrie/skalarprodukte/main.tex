\documentclass{article}
\usepackage{amsmath} 
\usepackage[a4paper]{geometry}
\usepackage{fancyhdr}
\pagestyle{fancy}
\lhead{Skalarprodukte}
\rhead{April 2025}
\begin{document}
  
\newcommand{\norm}[1]{\big| {#1} \big|}  
\newcommand{\vect}[1]{\overrightarrow{#1}} 
 
\section{Skalarprodukte}
Das Skalarprodukt von zwei Vektoren beschreibt die Summe aller Produkte der jeweiligen Koordinatenpaare der Vektoren. Es gilt
\[
 \vect{a} \cdot \vect{b} =
 a_1 \cdot b_1 + a_2 \cdot b_2 \qquad
 \text{mit} \quad
 \vect{a} = 
 \begin{pmatrix}
  a_1 \\
  a_2 \\
 \end{pmatrix}
 \, \text{und} \, 
 \vect{b} = 
 \begin{pmatrix}
  b_1 \\
  b_2 \\
 \end{pmatrix}
\]
 
\subsection{Orthogonalität}
\begin{description} 
 \item[Vektoren] Zwei Vektoren sind zueinander orthogonal, heißt senkrecht, wenn sie ein Skalarprodut von $0$ haben. \newline
 Dies kommt daher, dass die Enden von zwei orthogonalen Vektoren $\vect{a}$ und $\vect{b}$ miteinander Verbunden werden können, um mit dem $\vect{ab} = \vect{b} - \vect{a}$ ein rechtes Dreieck zu bilden. Somit gilt dem Satz von Pythagoras nach  
 \[  
  \norm{\vect{a}}^2 + \norm{\vect{b}}^2 = \norm{\vect{b} - \vect{a}}^2
 \]
 Dies kann nach $a_1 \cdot b_1 + a_2 \cdot b_2 = 0$ aufgelöst werden. % TODO hier tikz einfügen
 \item[Geraden] Zwei Geraden verlaufen zueinander, wenn ihre jeweiligen Richtungsvektoren zueinander orthogonal sind. % TODO müssen sie sich scheiden?
 \item[Gerade und Ebene] Eine Gerade verläuft zu einer Ebene orthogonal, wenn der Richtungsvektor der Gerade zu beiden Richtungsvektoren der Ebene orthogonal ist.  
\end{description}
 
\end{document} 
 
