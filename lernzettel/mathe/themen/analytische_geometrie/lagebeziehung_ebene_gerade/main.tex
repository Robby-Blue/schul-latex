\documentclass{article}
\usepackage[a4paper]{geometry}
\usepackage{fancyhdr} 
\usepackage{tikz} 
\pagestyle{fancy}
\lhead{Lagebeziehungen Ebene und Gerade}
\rhead{Juli 2025}
\begin{document}
 
\newcommand{\vect}[1]{\overrightarrow{#1}}
\newcommand{\drawplane}[0]{ 
 \begin{scope} 
  \clip(-0.7,0) rectangle (3.7,3);
  \fill[blue!20] (1.2-0*1.2-1.4,1-0*0.7+0.3) -- (1.2+1*1.2-1.4,1+1*0.7+0.3) -- (1.2+1*1.2+1.4,1+1*0.7-0.3) -- (1.2-0*1.2+1.4,1-0*0.7-0.3) -- cycle;
  \draw (1.5, 1.6) node[black] {$\mathrm{E}$}; 
 \end{scope}
} 
 
\newcommand{\drawuv}[0]{ 
 \draw[->,thick,blue] (0.3,1.35) -- ++(1.2*0.5,0.7*0.5) node[above left] {$\vect{u}$};
 \draw[->,thick,red] (0.3,1.35) -- ++(1.4*0.5,-0.3*0.5) node[below left] {$\vect{v}$}; 
} 
  
\section{Lagebeziehungen Ebene und Gerade}
Die Langebeziehung zwischen einer Ebene und einer Gerade beschreibt wie diese zueinander im Raum stehen. Dabei gibt es drei Möglichkeiten.
\begin{description}
 \item[Schneidung] Wenn eine Ebene und eine Gerade genau einen Schnittpunkt haben, so schneiden sie sich. Dies passiert zwangsweise, solange der Normalvektor $\vect{n}$ von $\mathrm{E}$ zu der Gerade $\mathrm{g}$ nicht orthogonal ist.
 \item[ineinander] Sind alle Punkte der Gerade $\mathrm{g}$ auch in der Ebene $\mathrm{E}$, so liegt $\mathrm{g}$ in $\mathrm{E}$. Dies passiert, wenn es unendlich viele Schnittpunkte oder mindestens ein Schnittpunkt und eine orthogonalität zwischen $\mathrm{n}$ und $\mathrm{g}$ vorliegt. 
 \item[echt parallel] Eine Gerade $\mathrm{g}$ ist zu der Ebene $\mathrm{E}$ echt parallel, wenn es keinen einzigen gemeinsamen Schnittpunkt gibt. Dabei sind zwangsweise, aber nicht hinreichend, $\vect{n}$ und $\mathrm{g}$ orthogonal.
\end{description}
 
\begin{center}
 \begin{tabular}{c c c}
  \begin{tikzpicture}
   \drawplane
   \drawuv 
     
   \draw (2, 1.5) node {$\times$};
   \draw[thick,blue] (1.6,0.1) -- ++(0.4, 1.4);
   \draw[->,thick,black] (2,1.5) -- ++(0.4, 1.4) node[right] {$g$}; 
   \draw[->] (0, 0) -- (3, 0); 
   \draw[->] (0, 0) -- (0, 3);
  \end{tikzpicture}
  &
  \begin{tikzpicture}
   \drawplane 
 
   \draw[->,thick,black] (0.5,1.5) -- ++(1.4*1.5, -0.3*1.5) node[above] {$g$}; 
   \draw[->] (0, 0) -- (3, 0); 
   \draw[->] (0, 0) -- (0, 3);
  \end{tikzpicture} 
  &
  \begin{tikzpicture}
   \drawplane 
     
   \draw[->,thick,black] (0.3,2.3) -- (2.7, 2.3) node[below left] {$g$}; 
   \draw[->] (0, 0) -- (3, 0); 
   \draw[->] (0, 0) -- (0, 3);
  \end{tikzpicture} 
  \\
  Schneidung & ineinander & echt parallel \\
  $\vect{n}$ und $\mathrm{g}$ nicht orthogonal &
  $\vect{n}$ und $\mathrm{g}$ orthogonal, &
  $\vect{n}$ und $\mathrm{g}$ orthogonal, \\
  & alle Punkte von $\mathrm{g}$ in $\mathrm{E}$ & kein gemeinsamer Punkt
 \end{tabular} 
\end{center} 
 
\end{document} 
 
 
