\documentclass{article}
\usepackage{tikz} 
\usepackage{wrapfig}
\usepackage{amsmath} 
\usepackage[a4paper]{geometry}
\usepackage{fancyhdr}
\pagestyle{fancy}
\lhead{Wachstum}
\rhead{März 2025}
\begin{document}
\noindent Wachstumsfunktionen beschreiben zeitlich einen kontextuellen Bestand und dessen Wachstum.
\section{Exponentielles Wachstum}
\begin{wrapfigure}{l}{5cm}
  \centering
  \begin{tikzpicture}     
    \draw[->, thick, domain=-2:1.3, samples=50] 
              plot (\x, {e^(\x)});
       
    \draw[->] (-2, 0) -- (2, 0); 
    \draw[->] (0, -0.5) -- (0, e^1.3);
   
    \draw (0, 0) node[below left] {0};    
  \end{tikzpicture}
\end{wrapfigure} 
Mit dem Anfangsbestand $A$, dem Wachstumsfaktor $b$ und bei Bedarf der Wachstumskonstante $k=\ln{b}$ gilt
\[ 
 A \cdot b^x 
 \quad \text{oder} \quad 
 A \cdot e^{k \cdot x}  
\]
Solange ${b > 1 \implies k > 0}$ ist die Funktion streng monoton steigend. Bei einer prozentualen Steigung von $s$ pro Zeiteinheit ist ${b=1+s}$, wobei $s$ als Dezimalzahl ausgedrückt wird. Daraus kann dann auch $k$ bestimmt werden. \newline
Liegt beispielsweise eine Steigung um $15\%$ pro Zeiteinheit vor, ist ${b=1 + 0,15=1,15 \implies k=\ln{1,15} \approx 0,14}$. 
\newline 
 
\subsection{Exponentieller Zerfall}
\begin{wrapfigure}{r}{5cm}
  \centering
  \begin{tikzpicture}     
    \draw[->, thick, domain=-1.3:2, samples=50] 
              plot (\x, {e^(-\x)});
       
    \draw[->] (-2, 0) -- (2, 0); 
    \draw[->] (0, -0.5) -- (0, e^1.3);
   
    \draw (0, 0) node[below left] {0};    
  \end{tikzpicture}
\end{wrapfigure} 
Ist hingegen das $b < 1 \implies k < 0$, fällt die Funktion streng monoton und es liegt ein exponentieller Zerfall vor. $b$ und $k$ ist nun jeweils der Abnahmefaktor und die Zerfallskonstante. Bei gegebener prozentualer Zerfallsrate pro Zeiteinheit ist ähnlich wie oben vorzugehen, nur mit einer Subtraktion anstelle von einer Addition zu $1$, also gilt ${b=1-s}$. \newline
Liegt beispielsweise eine Verlust von $15\%$ pro Zeiteinheit vor, ist ${b=1-0,15=0,85 \implies k=\ln {0,85} \approx -0,16}$.  
\section{Wachstumsgeschwindigkeiten}
Für jede Funktion, die ein exponentielles Wachstum beschreibt, $f(x)$, beschreibt die erste Ableitung $f'(x)$ die Wachstumsgeschwindigkeit. \newline
Gegenteilig ist die Stammfunktion von der Ableitung der Bestandsfunktion die Bestandsfunktion selbst, sodass die Änderung des Bestandes als bestimmtes Integral der Wachstumsgeschwindigkeit angesehen werden kann. Beschreibt $f(x)$ die Wachstumsgeschwindigkeit, so stellt
\[
 \int_{t_1}^{t_2} f(x) \,\mathrm{d}x 
\]
die Änderung des Bestandes zwischen den Zeitpunkten $t_1$ und $t_2$ dar. 
 
\end{document}
 
 
 
 
 
 
 
 
 
 
