\documentclass{article}
\usepackage{amsmath} 
\usepackage[a4paper]{geometry}
\usepackage{fancyhdr}
\pagestyle{fancy}
\lhead{Lineare Gleichungssysteme}
\rhead{März 2025}
\begin{document}
\section{Definition}
Eine lineares Gleichungssystem (LGS) ist eine Menge an linearen Gleichungen, mithilfe welcher nach den Werten von mehreren Variable aufgelöst werden kann. Die Gleichungen werden dabei in der folgenden Form, untereinander mit Strichen links und rechts aufgeschrieben
\[
\begin{vmatrix}
 r_{11} \cdot x + r_{12} \cdot y + ... = c_1 \\
 r_{21} \cdot x + r_{22} \cdot y + ... = c_2 
\end{vmatrix}
\] 
Ein LGS kann auch als Matrix aufgeschrieben werden, welche eine Zeile pro Gleichung und eine Spalte mehr als Variablen hat. In die ersten Spalten werden die konstanten, welche als Faktoren der Gleichungen genutzt werden eingetragen. Die letzte Spalte wird mit den konstanten hinter dem Gleich eingesetzt.
\[
\begin{bmatrix}
 r_{11} & r_{12} & c_1 \\
 r_{21} & r_{22} & c_2 \\
\end{bmatrix}
\] 
 
\section{Lösen}
\subsection{Schriftlich}
\subsection{CAS}
% mit interpretation der Lösungsmatrix 
 
\end{document}
 
 
 
 
