\documentclass{article}
\usepackage{enumerate}
\usepackage[ngerman]{babel} 
\usepackage{amsmath}
\usepackage{tcolorbox}
\tcbuselibrary{skins} 
\usepackage{tikz}
\usepackage[a4paper]{geometry}
\usepackage{fancyhdr}
\pagestyle{fancy}
\lhead{Mathe Präsentation \textit{Lösung}}
\rhead{zum 25. Februar 2026}
 
\newcommand{\task}[2]{%
\begin{center} 
 \begin{tcolorbox}[width=0.95\linewidth,
  enhanced,
  title=\textbf{#1)}, 
  attach boxed title to top left={
   xshift=0.75cm,
   yshift=-\tcboxedtitleheight/2,
   yshifttext=-\tcboxedtitleheight/2,
  }, 
  boxed title style={
    sharp corners,
    colback=white,
    colframe=black
  },
  coltitle=black, 
  colframe=black, colback=white,
  drop shadow=black!50!white,
  sharp corners, 
  boxrule=1pt, boxsep=8pt,
  left=0pt,right=0pt,top=0pt,bottom=0pt]
  #2
 \end{tcolorbox}
\end{center} 
} 
 
\begin{document}
\noindent \texttt{Jahr 2025, eA, Aufgaben 1B und 2A} 
\section*{Aufgabe 1B}
\[
 f(x) = 0,025x^4-0,3x^2+0,9 
\]
\task{a}{
\begin{center}
 \begin{tikzpicture}
  \draw[->] (0, 0) -- (0, 1.5) node[below right] {$y$};
  \draw[->] (-2.8, 0) -- (2.8, 0) node[above left] {$x$};
  \draw[blue, domain=-2.55:2.55, samples=20] plot (\x, {0.025*(\x)^4-0.3*(\x)^2+0.9});
  \node at (-2.55, 0) {$\times$}; 
  \node at (2.55, 0) {$\times$};
 
  \draw[red] (-1, 0) -- (-1, 0.625); 
  \draw[red] (-1, 0.625) -- (1, 0.625); 
  \draw[red] (1, 0) -- (1, 0.625);  
 
  \draw[<->, red] (-1, -0.5) -- (1, -0.5);   
  \node at (0, -0.5) [below] {$2\,m$};  
  \node at (-1, 0) [below] {$-1$}; 
  \node at (0, 0) [below] {$0$};
  \node at (1, 0) [below] {$1$};
 \end{tikzpicture} 
\end{center}
Die maximale Höhe ist die Höhe des niedrigsten Punktes, bei $-1$ und $1$. 
\[
 f(1) = 0,625 
\]} 
 
\task{b}{
Damit ist gemeint, dass die Fledermausgaube mindestens $5$ mal und höchstens $8$ mal so lang sein soll wie sie hoch ist. Hat sie eine Breite $b$ und eine Höhe $h$ so muss
\[
 5 < \frac{b}{h} < 8 
\]
Die Breite ist die Länge der einen Nullstelle zur anderen.
\begin{align*} 
 f(x) &= 0 \\
 &\Rightarrow x_{1,2} = \pm 2,45 \\
 &\Rightarrow b = 2 \cdot 2,45 
\end{align*} 
Die Höhe der Fledermausgaube ist die Höhe am Höhepunkt bei $x=0$
\[
 h = f(0) = 0,9 
\]
Somit liegt das Verhältnis nun bei
\[
 \frac{2 \cdot 2,45}{0,9} \approx 5.44 
\]
Weil $5 < 5.44 < 8$ wird die Vorgabe eingehalten. 
}
\task{c}{Wird nach der maximalen Steigung gesucht ist die entweder offensichtlich (das Maximum, also die Ableitung gleich null, von $f'(x)$= oder bei der Wendestelle, also bei
\[
 f''(x) = 0 \Rightarrow x_{1,2} \approx \pm 1,4142 
\]
Nun ist dessen Steigung
\begin{align*} 
 f'(-1,4142) &= 0,56 \\
 &\Rightarrow \tan^{-1}(0,4) = 29,5^\circ 
\end{align*} 
}
\task{d}{
Vertikal werden insgesamt $20\,cm$, $0,2$, entfernt. Also endet das Fenster auch sobald
\begin{align*} 
 f(x) - 0,2 &= 0 \\
 &\Rightarrow x_{1,2} \pm 1,78
 \quad \text{und} \quad
 x_{3,4} \pm 2,97  
\end{align*}
Mit dem $10\,cm$ horizontalen Steg, auf beiden Seiten auf jeweils $5\,cm$ aufgeteilt, liegt das Fenster nun im Intervall $[0,05;1,78]$.
 
Der Flächeninhalt ist also
\[ 
 2 \cdot \int_{0,05}^{1,78} f(x)-0,2\,\text{d}x = 1,473
\] 
}
\task{e}{
Weil manche Teile des Gaube schneller sinken als eine gerade Linie von $(0\,\vert\,0,9)$ zu $(2,45\,\vert\,0)$ würde, wären Teile des Fensters außerhalb der Gaube.
 
Damit es geradeso passt musst das Dreieck eine Tangente zur Kante der Gaube sein. Es muss eine Gerade gefunden werden, welche durch $(0\,\vert\,0,9)$ geht und eine Tangente an einem $(x\,\vert\,f(x))$ hat.
\begin{align*}
 f'(x) \cdot x + 0,9 &= f(x) \\
 &\Rightarrow x_1 = 0
 \quad \text{und} \quad
 x_{2,3} = \pm 2 
\end{align*} 
Die Tangente muss bei $x=2$ liegen. Wird diese Steigung genutzt, so endet das Fenster bei der Nullstelle
\[
 f'(2) \cdot x + 0,9 = f(x) \Rightarrow x = 2,25
\]
Mit $A = 1/2 \cdot b \cdot h$ und zwei Fensterhälften ist der Flächeninhalt $2,025$. 
}
\task{f}{
\begin{enumerate}[I]
 \item Eine Funktion $d$ wird gebildet, welche die Differenz zwischen $f$ und $g$ darstellt.
 \item Die Extremstellen der Funktion $d$, also der Differenz von $f$ und $g$ werden in der Linken hälfte gefunden.
 \item Es wird überprüft, ob die Extremstellen Hochpunkte oder Tiefpunkte sind.
 \item Die Differenzen an den Extremstellen werden berechnet. 
\end{enumerate}
$x_2$ wurde als Hochpunkt bestimmt, heißt es stellt die größte Differenz zwischen $f$ und $g$ dar. Dort gibt es eine Differenz der beiden von $0,06\,m$, also $6\,cm$.} 
\newpage 
\section*{Aufgabe 2A} 
\end{document}