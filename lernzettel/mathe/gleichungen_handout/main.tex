\documentclass{article}
\usepackage{xcolor} 
\usepackage{amsmath}
\usepackage[a4paper]{geometry}
\usepackage{fancyhdr}
\pagestyle{fancy}
\lhead{Koordinatengleichungen bestimmen}
\rhead{Mai 2025}
\begin{document}
 
\newcommand{\vect}[1]{\overrightarrow{#1}}
\newcommand{\vectp}[1]{\vect{\mathrm{#1}}}
 
\section*{Anleitung}
Um eine gegebene Parametergleichung in eine Koordinatengleichung umzuwandeln muss sie nur in die Form diese umgeformt werden.
\[
 \vect{x} \cdot \vect{n} = \vectp{OA} \cdot \vect{n} 
\]
Dies kann in drei einfachen Schritten passieren 
\begin{description}
 \item[1. Normalvektor bestimmen] Zuerst muss der Normalvektor $\vect{n} = \vect{u} \times \vect{v}$ ausgerechnet werden.
 \item[2. Linke Seite] Nun muss die linke Seite der gleichung bestimmt werden. Dafür muss einfach nur das Skalarprodukt des eben bestimmten $\vect{n}$ und einem $\vect{x}$ gefunden werden. Weil $\vect{x} = \begin{pmatrix} x_1 \\ x_2 \\ x_3 \end{pmatrix}$ bereits bekannt ist, müssen nur die jeweiligen $x$-Variablen mit ihren Faktoren des $\vect{n}$ multipliziert und dann addiert werden.
 \item[3. Rechte Seite] Zuletzt muss nur noch die rechte Seite bestimmt werden, indem $\vectp{OA} \cdot \vect{n}$ ausgerechnet wird.
\end{description} 
  
\section*{Beispiel}
Gegeben ist die Parametergleichung
\begin{equation}
 E: \vect{x} =
 \colorbox{blue!50}{$\begin{pmatrix} -2 \\ 4 \\ 1 \end{pmatrix}$}
 + r \cdot \colorbox{red!50}{$\begin{pmatrix} 3 \\ 0 \\ 1 \end{pmatrix}$}
 + s \cdot \colorbox{green!50!black!50}{$\begin{pmatrix} 3 \\ -2 \\ 2 \end{pmatrix}$}
\end{equation}
 
\begin{description}
 \item[1. Normalvektor bestimmen] Um $\vect{n}$ zu bestimmen, muss das Vektorenprodukt der beiden Richtungsvektoren, hier \colorbox{red!50}{$\begin{pmatrix} 3 \\ 0 \\ 1 \end{pmatrix}$} und \colorbox{green!50!black!50}{$\begin{pmatrix} 3 \\ -2 \\ 2 \end{pmatrix}$}, gefunden werden, sodass
 \begin{equation}
  \vect{n} =
   \colorbox{red!50}{$\begin{pmatrix} 3 \\ 0 \\ 1 \end{pmatrix}$}
   \times \colorbox{green!50!black!50}{$\begin{pmatrix} 3 \\ -2 \\ 2 \end{pmatrix}$}
   = \colorbox{yellow}{$\begin{pmatrix} 2 \\ -3 \\ -6 \end{pmatrix}$}
 \end{equation}
 
 \item[2. Linke Seite bestimmen] Die linke Seite ist einfach $\vect{x} \cdot \vect{n}$, also
 \begin{equation}
  \colorbox{purple!50}{$\begin{pmatrix} x_1 \\ x_2 \\ x_3 \end{pmatrix}$}
  \cdot \colorbox{yellow}{$\begin{pmatrix} 2 \\ -3 \\ -6 \end{pmatrix}$}
  = \colorbox{yellow}{$2$} \colorbox{purple!50}{$x_1$} +
  \colorbox{yellow}{$(-3)$} \colorbox{purple!50}{$x_2$} +
  \colorbox{yellow}{$(-6)$} \colorbox{purple!50}{$x_3$}
 \end{equation}
 
 \item[3. Rechte Seite bestimmen] Die rechte Seite der Gleichung ist nur $\vectp{OA} \cdot \vect{n}$, also
 \begin{equation}
  \colorbox{blue!50}{$\begin{pmatrix} -2 \\ 4 \\ 1 \end{pmatrix}$}
  \cdot \colorbox{yellow}{$\begin{pmatrix} 2 \\ -3 \\ -6 \end{pmatrix}$}
  = \colorbox{cyan!50}{$-22$}
 \end{equation} 
\end{description} 
Somit ist die gesamte Koordinatengleichung
\begin{equation} 
 \colorbox{yellow}{$2$} \colorbox{purple!50}{$x_1$} +
 \colorbox{yellow}{$(-3)$} \colorbox{purple!50}{$x_2$} +
 \colorbox{yellow}{$(-6)$} \colorbox{purple!50}{$x_3$}
 = \colorbox{cyan!50}{$-22$}
\end{equation} 
 
\setcounter{equation}{0}
\section*{Herleitung}
Dieser Teil ist nicht direkt relevant, um eine Parametergleichung in eine Koordinatengleichung umzuwandeln, kann aber dabei hilfreich sein, den Prozess besser nachvollziehen zu können. \newline
Ist eine Ebene in der Parametergleichung
\begin{equation}
 E: \vect{x} = \vectp{OA} + r \cdot \vect{u} + s \cdot \vectp{v} 
\end{equation}
gegeben, kann diese zuerst durch bereits bekannte Wege in die Normalengleichung
\begin{equation}
 (\vect{x} - \vect{p}) \cdot \vect{n} = 0
 \quad \text{mit} \quad
 \vect{x} = \begin{pmatrix} x_1 \\ x_2 \\ x_3 \end{pmatrix} 
\end{equation}
umgewandelt werden. Dabei ist $\vect{n}$ der Normalenvektor der Ebene, ein Vektor, welcher zu beiden Richtungsvektoren ortoghonal ist, also $\vect{n} = \vect{u} \times \vect{v}$ und der Stützvektor $\vect{p}$ ist einfach nur der Stützvektor $\vectp{OA}$.
Somit gilt 
\begin{equation}
 (\vect{x} - \vectp{OA}) \cdot \vect{n} = 0 
 \quad \text{mit} \quad
 \vect{n} = \vect{u} \times \vect{v}
\end{equation}
Wird in der Gleichung die Klammer ausmultipliziert, um die Gleichung zu vereinfachen folgt  
\begin{equation}
 \vect{x} \cdot \vect{n} - \vectp{OA} \cdot \vect{n} = 0 
 \quad \text{mit} \quad
 \vect{n} = \vect{u} \times \vect{v}
\end{equation}
Wird auf beiden Seiten $\vectp{OA} \cdot \vect{n}$ addiert, folgt das Ergebnis 
\begin{equation}
 \vect{x} \cdot \vect{n} = \vectp{OA} \cdot \vect{n} 
 \quad \text{mit} \quad
 \vect{n} = \vect{u} \times \vect{v}
\end{equation}
 
 
\end{document}
 
 
